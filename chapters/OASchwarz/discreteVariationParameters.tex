\begin{appendix}
\section{Discrete variation of parameters}
\label{sec:OASchwarz_appendix_discreteVariationParameters}
We look for the solutions of the Fourier transform of equation
\eqref{eq:OASchwarz_DiscreteStationaryState_homogeneousEq1}:
\begin{subequations}
	\begin{align}
	(i\omega + if) \widehat{u}_{m+\frac{1}{2}} -
		\nu_j \frac{\widehat{\phi}_{m+1} -
		\widehat{\phi}_m}{h_j}
		&=U_0
	\hspace{1.2cm} \mbox{in}\;
	(\omega_{\min},\omega_{\max})
		\label{eq:OASchwarz_appendix_eqFourier}\\
		u_m(t=0) &= U_0(z_m),   \hspace{1.8cm}
		\forall m, M_o \leq m \leq M_a \\
		\widehat{u}(H_j, \omega) &= 0,
		\hspace{2.1cm}
		\omega \in [\omega_{\min},\omega_{\max}]
		\label{eq:OASchwarz_appendix_bdCond}\\
		\frac{3\alpha^e}{2} 
		\left( \widehat{u}_{1/2} - \widehat{u}_{-1/2}
		+ \frac{{\cal O}}{3} \overline{
		\widehat{u}_{1/2}(-\omega) -
		\widehat{u}_{-1/2}(-\omega)} \right) &=
		\nu_{\rm a}\widehat{\phi}_{\rm a, 0},
		\hspace{1.12cm} t \in [0,T]\\
		\rho_{\rm o} \nu_{\rm o}\widehat{\phi}_{\rm o, 0}
		&= \rho_{\rm a} \nu_{\rm a}\widehat{\phi}_{\rm a, 0},
		\hspace{1.0cm} t \in [0,T]
		\end{align}
		\end{subequations}
% The set of solutions of the homogeneous version of
% \eqref{eq:OASchwarz_appendix_eqFourier} is given in the
% convergence analysis: $\widehat{\phi}_{\rm o}(-m h_{\rm o})
% 	= A (\lambda_{\rm o}+1)^m$
% and
% $\widehat{\phi}_{\rm a}(m h_{\rm a}) = B (\lambda_{\rm a}+1)^m$ 
% with $\lambda_j = \frac{1}{2}\left(\chi_j - \sqrt{\chi_j} \sqrt{\chi_j + 4}\right)$, 
% $\chi_j=\frac{i (\omega+f) h_j^2}{\nu_j}$ and $m$ the space index.
\begin{itemize}
\item if $\omega=-f$:
	\begin{equation}
		\widehat{\phi}_{a, m+1} - \widehat{\phi}_{a,m} =
		-\frac{h_a}{\nu_a} U_0(m h_a), ~~~~~~ m \geq 0
	\end{equation}
		$\widehat{\phi}_{a,m}= {C}_1^{-f} - \frac{h_a}{\nu_a}
	\sum_{i=0}^{m-1} U_{0, i+1/2}$.
	Since $U_{a, m+1/2}- U_{a, m-1/2} =
	h_a \phi_{a, m}$ we obtain with the Dirichlet
	boundary condition:
	\begin{equation}
		U_{a,m+1/2} = -h_a\sum_{i=m+1}^{H_a/h_a}
		\widehat{\phi}_{a,i} =(m h_a - H_a){C}_1^{-f}+
		\frac{h_a^2}{\nu_a}\sum_{i=m+1}^{H_a/h_a}
		\sum_{k=0}^{i}U_{0,k+1/2}
	\end{equation}
	for $m<0$, we get:
	\begin{equation}
		U_{o,m+1/2} = h_o\sum_{i=H_o/h_o}^{m}
		\widehat{\phi}_{o,i} =(H_o - (m+1) h_o){C}_2^{-f}+
		\frac{h_o^2}{\nu_o}\sum_{i=H_o/h_o}^{m}
		\sum_{k=i}^{-1}U_{0,k+1/2}
	\end{equation}
	and $(\widehat{\phi}_{a,0}, \widehat{\phi}_{o,0}) =
		({C}_1^{-f},{C}_2^{-f})$.
	\begin{remark}
		having infinite domains with constant space step
		leads to $U_{m\rightarrow \infty}=0 \Rightarrow
		\phi_{(m\rightarrow \infty)}=0$, which overdetermines
		the system. This is why we keep $H_j$ here.
	\end{remark}
	Note that with the help of the interface condition
	$\rho_o \nu_o \phi_o = \rho_a \nu_a \phi_a$,
	the jump of solutions
	$\widehat{u}_{1/2} - \widehat{u}_{-1/2}$ can be written
		$ \Delta H \times {C}_1^{-f} +S(U_0)$ where
	$\Delta H = (|H_o|\epsilon \frac{\nu_o}{\nu_a} - H_a)$.
\item if $\omega \neq -f$: \\
We extend the well-known method of variation of parameters to the
discrete case to compute a solution of
\eqref{eq:OASchwarz_appendix_eqFourier}.
	let $B_m^1, B_m^2$ and $\lambda_u$ such that
	the solutions of the homogeneous version of
	\eqref{eq:OASchwarz_appendix_eqFourier} are
		$\widehat{u} = B^1_m \lambda_u^m +
			B^2_m \lambda_u^{-m}$.
	Equation \eqref{eq:OASchwarz_appendix_eqFourier}
	can be rewritten as
	\begin{equation}
		(\widehat{u}_{m+1} - \widehat{u}_{m})
		+\chi_2 \widehat{u}_{m}
		- (\widehat{u}_{m} - \widehat{u}_{m-1})
		= U_0
	\end{equation}
	Injecting $\widehat{u} = B^1_m \lambda_u^m +
	B^2_m \lambda_u^{-m}$ and rearranging terms gives
\begin{equation}
	\begin{aligned}
	&B^1_m \underbrace{\left(
		\lambda_u^{m+1} - \lambda_u^m + \chi_2 \lambda_u^m
		- (\lambda_u^{m} - \lambda_u^{m-1})
		\right)}_{=0} ~~~~~~~+ B^2_m \times 0 \\
	&+ \lambda_u^{m+1} (B_{m+1}^1 - B_m^1)
		- \lambda_u^{m} (B_{m}^1 - B_{m-1}^1)
		~~~~+ \lambda_u^{-m-1} (B_{m+1}^2 - B_m^2)
		- \lambda_u^{-m} (B_{m}^2 - B_{m-1}^2)
		\\
	&+ (\lambda_u^m - \lambda_u^{m-1})(B_{m}^1 - B_{m-1}^1)
		~~~~~~~~~~~~~~+ (\lambda_u^{-m} - \lambda_u^{-m+1})
				(B_{m}^2 - B_{m-1}^2)\\
		&~~~~~~~~~~~~= U_0
	\end{aligned}
\end{equation}
A particular solution of \eqref{eq:OASchwarz_appendix_eqFourier}
can be taken as
\begin{equation}
\begin{cases}
	\lambda_u^m(B_m^1 - B_{m-1}^1)
	+ \lambda_u^{-m}(B_m^2 - B_{m-1}^2)&= 0 \\
	(\lambda_u^m - \lambda_u^{m-1})(B_m^1 - B_{m-1}^1)
	+ (\lambda_u^{-m} - \lambda_u^{-m+1})(B_m^2 - B_{m-1}^2)
	&= {U_0}
\end{cases}
\end{equation}
which has a unique solution:
\begin{equation}
	\begin{pmatrix}
B_m^1 - B_{m-1}^1\\
B_m^2 - B_{m-1}^2
	\end{pmatrix}
	=
	\begin{pmatrix}
		\lambda_u^m & \lambda_u^{-m} \\
		(\lambda_u^m - \lambda_u^{m-1})  & 
		(\lambda_u^{-m} - \lambda_u^{-m+1})
	\end{pmatrix}^{-1}
	\begin{pmatrix}
		0 \\ U_0
	\end{pmatrix}
\end{equation}
The determinant of the inversed matrix is
$\lambda_u^{-1} - \lambda_u$ which is not zero since
$\omega\neq -f$.
Let us note 
$\begin{pmatrix}
	S_m^1(U_0) + {C}_3\\
	S_m^2(U_0) + {C}_4
\end{pmatrix}:=\begin{pmatrix}
B_m^1\\
B_m^2
\end{pmatrix}$.
Let $M_a=\frac{H_a}{h_a}$.
The Dirichlet boundary condition is equivalent to
$B_{M_a}^1 (\lambda_u)^{2M_a} = - B_{M_a}^2$ which determines
${C}_4$:
\begin{equation}
- {C}_4 = \left(S_{M_a}^1(U_0) + {C}_3\right)
	(\lambda_u)^{2M_a} + S_{M_a}^2(U_0)
\end{equation}
\begin{equation}
	\begin{aligned}
		\widehat{u}_{\frac{1}{2}} &= {C}_3 (1 - \lambda_u^{2M_a})
	+ \underbrace{S_0^1+S_0^2 - S_{M_a}^1 \lambda_u^{2M_a}
		-S_{M_a}^2}_{:=S_{u, a}^{m=1/2}(U_0)} \\
		\chi_a \widehat{u}_{\frac{1}{2}} &= u_{\frac{3}{2}} - u_{\frac{1}{2}}
		- \phi_0 + \frac{h^2_a}{\nu_a} U_0
	\end{aligned}
\end{equation}
Those equations give
\begin{equation}
	\begin{aligned}
		\widehat{\phi}_{a,0} ={C}_3 \left(
			\underbrace{\lambda_u-\lambda_u^{2M_a-1}
			- (\chi_a+1)(1-\lambda_u^{2M_a})}_{:=K_{1,a}}
			\right)
		-\left(\underbrace{(\chi_a + 1) S_{u,a}^{m=1/2}
		- S_{u,a}^{m=3/2}
		- \frac{h^2}{\nu} U_0}_{:=S_{\phi,a}^{m=0}}\right)
	\end{aligned}
\end{equation}
We obtain similar equations in the ocean part:
\begin{equation}
	\begin{aligned}
		\widehat{u}_{-\frac{1}{2}} &= {C}_5 (1 -
			\lambda_{u_o}^{2M_o})
			+ S_{u, o}^{m=-1/2}(U_0) \\
		\widehat{\phi}_{o,0} &={C}_5 K_{1,o}
		-S_{\phi,o}^{m=0}
	\end{aligned}
\end{equation}
The continuity of the flux at interface
$\rho_o \nu_o \phi_o = \rho_a \nu_a \phi_a$ lets us rewrite
\begin{equation}
{C}_5 = {C}_3 \times \epsilon
	\frac{\nu_a}{\nu_o} \frac{K_{1,a}}{K_{1,o}} +
	\frac{S_{\phi,o}^0 - S^0_{\phi, a}}{K_{1,o}}
\end{equation}
The jump between the solutions is
\begin{equation}
	\widehat{u}_{\frac{1}{2}} - \widehat{u}_{-\frac{1}{2}}
	= {C}_3 K_{2} + \Delta S
\end{equation}
where $K_{2} = 1 - \lambda_{u_a}^{2M_a} - (1 - \lambda_{u_o}^{2M_o})
\epsilon \frac{\nu_a}{\nu_o} \frac{K_{1,a}}{K_{1,o}}$.
The friction law writes
\begin{equation}
	\underbrace{(\nu_a K_{1,a} - \frac{3\alpha^e K_{2}}{2})}_{
		K_3(\omega)
	}{C}_3
	-\underbrace{
		\alpha^e \frac{{\cal O} \overline{K_2}(-\omega)}{2}}_{
		\overline{K_4(-\omega)}}
	\times \overline{{C}_3(-\omega)}
	= S_{\rm friction}
\end{equation}
where $S_{\rm friction} =
	\frac{3\alpha^e}{2} \left(
	\Delta S + \frac{\cal O}{3}
	\overline{\Delta S(-\omega)}\right)
	+\nu_a S_{\phi,a}^{m=0}$.
The linearized problem is well posed if
\begin{equation}
	\begin{pmatrix}
		K_3(\omega)& - \overline{K_4(-\omega)}\\
		-K_4(\omega) & \overline{K_3(-\omega)}
	\end{pmatrix}
\end{equation}
is inversible for $\omega \notin \{f, -f\}$ and
the matrix
\begin{equation}
	\begin{pmatrix}
		\nu_a - \frac{3 \alpha^e}{2}\Delta H &
		\overline{K_4(f)} \\
		-\frac{\alpha^e}{2}\overline{\cal O} \times \Delta H&
		\overline{K_3(f)}
	\end{pmatrix}
\end{equation}
is inversible.
\end{itemize}
Let us approximate some of the terms for sufficiently big $M_a, M_o$:
\begin{itemize}
	\item $K_1 \approx \lambda_u - 1 - \chi_j \approx \chi_j/2$
		\myTD{verifier ça}
	\item $K_2 \approx 1 (- \epsilon\frac{\nu_a}{\nu_o}
		\frac{K_{1,a}}{K_{1,o}})$
	\item $K_3 \approx \nu_a \chi_a / 2 - 3 \alpha^e / 2$
	\item $K_4 \approx \overline{\alpha^e {\cal O}(-\omega)} / 2$
\end{itemize}
Now, $\left|
	\begin{pmatrix}
		K_3(\omega)& - \overline{K_4(-\omega)}\\
		-K_4(\omega) & \overline{K_3(-\omega)}
	\end{pmatrix}\right|\approx
	|\alpha^e|^2 \overline{ {\cal O}(-\omega)}
	{\cal O}(\omega) / 4
	$
	and $\overline{ {\cal O}(-\omega)}
	{\cal O}(\omega) = \frac{-\omega + f}{-\omega-f} \times
	\frac{\omega + f}{\omega-f} = -1$
\end{appendix}
