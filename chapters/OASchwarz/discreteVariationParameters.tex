\section{Appendix: well-posedness of the linearized quadratic friction case}
\label{sec:OASchwarz_appendix_discreteVariationParameters}
We look for the solutions of the Laplace transform
(with the frequency variable $s=\sigma + i\omega, \sigma>0$) of our
coupled problem (the steady state has been subtracted from $u$):
\begin{subequations}
	\label{eq:OASchwarz_appendix_discreteVariationParameters}
	\begin{align}
	(s + if) \widehat{u}_{m+\frac{1}{2}} -
		\nu_j \frac{\widehat{\phi}_{m+1} -
		\widehat{\phi}_m}{h_j}
		&=U_0
	\hspace{1.2cm} \omega \in
	(\omega_{\min},\omega_{\max})
		\label{eq:OASchwarz_appendix_eqFourier}\\
		u_m(t=0) &= U_0(z_m),   \hspace{1.8cm}
		\forall m, M_o \leq m \leq M_a \\
		\widehat{u}(H_j, s) &= 0,
		\hspace{2.1cm}
		\omega \in (\omega_{\min},\omega_{\max})
		\label{eq:OASchwarz_appendix_bdCond}\\
		\frac{3\alpha^e}{2} 
		\left( \widehat{u}_{1/2} - \widehat{u}_{-1/2}
		+ \frac{{\cal O}}{3} \overline{
			\widehat{u}_{1/2}(\overline{s}) -
		\widehat{u}_{-1/2}(\overline{s})} \right) &=
		\nu_{\rm a}\widehat{\phi}_{\rm a, 0},
		\hspace{1.12cm}
		\omega \in (\omega_{\min},\omega_{\max})
		\label{eq:OASchwarz_appendix_linearizedCond}
		\\
		\rho_{\rm o} \nu_{\rm o}\widehat{\phi}_{\rm o, 0}
		&= \rho_{\rm a} \nu_{\rm a}\widehat{\phi}_{\rm a, 0},
		\hspace{1.0cm}
		\omega \in (\omega_{\min},\omega_{\max})
		\label{eq:OASchwarz_appendix_continuityFlux}
		\end{align}
		\end{subequations}
where ${\cal O} = \frac{U_{\rm a}^e - U_{\rm o}^e}{\overline{U_{\rm a}^e
	- U_{\rm o}^e}}$.
% The set of solutions of the homogeneous version of
% \eqref{eq:OASchwarz_appendix_eqFourier} is given in the
% convergence analysis: $\widehat{\phi}_{\rm o}(-m h_{\rm o})
% 	= A (\lambda_{\rm o}+1)^m$
% and
% $\widehat{\phi}_{\rm a}(m h_{\rm a}) = B (\lambda_{\rm a}+1)^m$ 
% with $\lambda_j = \frac{1}{2}\left(\chi_j - \sqrt{\chi_j} \sqrt{\chi_j + 4}\right)$, 
% $\chi_j=\frac{i (\omega+f) h_j^2}{\nu_j}$ and $m$ the space index.
\begin{remark}
	To show the well-posedness of the non-linear
	problem, it is necessary to add external data
	that does not depends on $\widehat{u}$
	in \eqref{eq:OASchwarz_appendix_bdCond} and
	\eqref{eq:OASchwarz_appendix_linearizedCond}.
	It only complicates the derivation and does not need
	any special treatment: we hence omit them.
\end{remark}
To show that there exists a unique solution to
\eqref{eq:OASchwarz_appendix_discreteVariationParameters},
we first compute the jump $\widehat{u}_{1/2} - \widehat{u}_{-1/2}$
then prove that the friction law always has a unique solution.
\subsection{Jump of the solution
	$\widehat{u}_{1/2} - \widehat{u}_{-1/2}$}
We first compute the jump of the solution
$\widehat{u}_{1/2} - \widehat{u}_{-1/2}$.
\begin{remark}
	If a Fourier transform was used
	(corresponding to $\sigma=0$), the case $\omega=-f$ would
	require a particular attention as
	$\lambda_u^{-1} - \lambda_u$ would attain 0.
\end{remark}
We extend the well-known method of variation of parameters to the
discrete case to compute a solution of
\eqref{eq:OASchwarz_appendix_eqFourier}.
Let $B_m^1, B_m^2$ and $\lambda_u$ such that
the solutions of \eqref{eq:OASchwarz_appendix_eqFourier} are
$\widehat{u}_{a,m+\frac{1}{2}} = B^1_m \lambda_u^m + B^2_m \lambda_u^{-m}$, where
$\lambda_u=1+\frac{\chi_2}{2}
-\frac{1}{2}\sqrt{\chi_2}\sqrt{\chi_2+4}$ and
$\chi_j = h_j^2 \frac{s + if}{\nu_j}$.
Note that it it similar to the solution of the homogeneous
equation but $B^1, B^2$ now depend on $m$.
	Equation \eqref{eq:OASchwarz_appendix_eqFourier}
	can be rewritten as
	\begin{equation}
		(\widehat{u}_{m+\frac{3}{2}} - \widehat{u}_{m+\frac{1}{2}})
		-\chi_2 \widehat{u}_{m+\frac{1}{2}}
		- (\widehat{u}_{m+\frac{1}{2}} - \widehat{u}_{m-\frac{1}{2}})
		= -\frac{h_2^2}{\nu_2} U_0
	\end{equation}
	Injecting $\widehat{u}_{m+\frac{1}{2}} = B^1_m \lambda_u^m +
	B^2_m \lambda_u^{-m}$ and rearranging terms gives
\begin{equation}
	\begin{aligned}
	&B^1_m \underbrace{\left(
		\lambda_u^{m+1} - \lambda_u^m - \chi_2 \lambda_u^m
		- (\lambda_u^{m} - \lambda_u^{m-1})
		\right)}_{=0} ~~~~~~~+ B^2_m \times 0 \\
	&+ \lambda_u^{m+1} (B_{m+1}^1 - B_m^1)
		- \lambda_u^{m} (B_{m}^1 - B_{m-1}^1)
		~~~~+ \lambda_u^{-m-1} (B_{m+1}^2 - B_m^2)
		- \lambda_u^{-m} (B_{m}^2 - B_{m-1}^2)
		\\
	&+ (\lambda_u^m - \lambda_u^{m-1})(B_{m}^1 - B_{m-1}^1)
		~~~~~~~~~~~~~~+ (\lambda_u^{-m} - \lambda_u^{-m+1})
				(B_{m}^2 - B_{m-1}^2)\\
		&~~~~~~~~~~~~= -\frac{h_2^2}{\nu_2}U_0
	\end{aligned}
\end{equation}
A particular solution of \eqref{eq:OASchwarz_appendix_eqFourier}
can be taken as
\begin{equation}
\begin{cases}
	\lambda_u^m(B_m^1 - B_{m-1}^1)
	+ \lambda_u^{-m}(B_m^2 - B_{m-1}^2)&= 0 \\
	(\lambda_u^m - \lambda_u^{m-1})(B_m^1 - B_{m-1}^1)
	+ (\lambda_u^{-m} - \lambda_u^{-m+1})(B_m^2 - B_{m-1}^2)
	&= -\frac{h_2^2}{\nu_2}{U_0}
\end{cases}
\end{equation}
which can be inverted:
\begin{equation}
	\label{eq:OASchwarz_appendix_deltaB}
	\begin{pmatrix}
B_m^1 - B_{m-1}^1\\
B_m^2 - B_{m-1}^2
	\end{pmatrix}
	=
	\begin{pmatrix}
		\lambda_u^m & \lambda_u^{-m} \\
		(\lambda_u^m - \lambda_u^{m-1})  & 
		(\lambda_u^{-m} - \lambda_u^{-m+1})
	\end{pmatrix}^{-1}
	\begin{pmatrix}
		0 \\ \frac{h_2^2}{\nu_2}-U_0
	\end{pmatrix}
\end{equation}
The determinant of the inverse matrix in
\eqref{eq:OASchwarz_appendix_deltaB} is
$\lambda_u^{-1} - \lambda_u$ which is never zero because
$\sigma > 0$.
\begin{remark}
	A particular solution was chosen here: as in the continuous case
	of the variation of parameters, the difference between this
	solution and another one solves the homogeneous equation.
	The degrees of freedom $B_0^1$ and $B_0^2$ actually represent
	the whole space of solutions.
\end{remark}
\par
The sum of \eqref{eq:OASchwarz_appendix_deltaB} from $1$ to $m$
gives a relation between $B^1_m, B^2_m$ and
$B^1_0, B^2_0$ with the parameters $\lambda_u, \frac{h_2^2}{\nu_2}U_0$ and $m$.
Showing that $B^1_0, B^2_0$ are uniquely determined is hence
sufficient to show that there is a unique solution $\widehat{u}$
in the atmosphere.

Let us note 
$\begin{pmatrix}
B_m^1\\
B_m^2
\end{pmatrix}
:= \begin{pmatrix}
	 B^1_0 + S_m^1(U_0)\\
	 B^2_0 + S_m^2(U_0)
\end{pmatrix}$.
The vector $\begin{pmatrix} S_m^1(U_0)\\ S_m^2(U_0) \end{pmatrix}$
is the sum of the right hand side of
\eqref{eq:OASchwarz_appendix_deltaB} from $1$ to $m$. It makes
the link between $B^1_0, B^2_0$ and $B^1_m, B^2_m$ but will not
affect the presence of a unique solution for $B^1_0, B^2_0$.
\begin{remark}
	All the variables $S$ in the following will not be used
	to determine the existence of a unique solution
	$\widehat{u}$.
\end{remark}
The Dirichlet boundary condition at $H_a = (M_a + \frac{1}{2})h_a$
is equivalent to
$B_{M_a}^1 (\lambda_u)^{2M_a} = - B_{M_a}^2$ which determines
$B^2_0$:
\begin{equation}
- B^2_0 = \left(S_{M_a}^1(U_0) + B^1_0\right)
	(\lambda_u)^{2M_a} + S_{M_a}^2(U_0)
\end{equation}
We hence only need the sole value of $B^1_0$ to characterize
$\widehat{u}$. In particular, the variable $\widehat{u}_{\frac{1}{2}}$
used in the linearized transmission condition reads
\begin{equation}
	\widehat{u}_{\frac{1}{2}} = B^1_0 (1 - \lambda_u^{2M_a})
	+ \underbrace{S_0^1+S_0^2 - S_{M_a}^1 \lambda_u^{2M_a}
		-S_{M_a}^2}_{:=S_{u, a}^{m=1/2}(U_0)}
\end{equation}
To obtain the derivative at interface $\widehat{\phi}_{a,0}$
\eqref{eq:OASchwarz_appendix_eqFourier} is used
at the first grid level:
\begin{equation}
	\chi_a \widehat{u}_{\frac{1}{2}} =
	\widehat{u}_{\frac{3}{2}} - \widehat{u}_{\frac{1}{2}}
	- h_a \widehat{\phi}_{a,0} + \frac{h^2_a}{\nu_a} U_0
\end{equation}
Those equations give
\begin{equation}
	\begin{aligned}
		\widehat{\phi}_{a,0} =& B^1_0 \underbrace{\frac{1}{h_a}
		\left(
			\lambda_u-\lambda_u^{2M_a-1}
			- (\chi_a+1)(1-\lambda_u^{2M_a})
			\right)}_{:=K_{1,a}}\\
		&-\underbrace{\frac{1}{h_a}\left((\chi_a + 1) S_{u,a}^{m=1/2}
		- S_{u,a}^{m=3/2}
		- \frac{h_a^2}{\nu_a} U_0\right)}_{:=S_{\phi,a}^{m=0}}
	\end{aligned}
\end{equation}
We obtain similar equations in the ocean part:
\begin{equation}
	\begin{aligned}
		\widehat{u}_{-\frac{1}{2}} &= A^1_0 (1 -
			\lambda_{u_o}^{2M_o})
			+ S_{u, o}^{m=-1/2}(U_0) \\
		\widehat{\phi}_{o,0} &= - A^1_0 K_{1,o}
		-S_{\phi,o}^{m=0},
		~~~~~~ K_{1,o} = \frac{1}{h_o}
		\left(
			\lambda_{u_o}-\lambda_{u_o}^{2M_o-1}
			- (\chi_o+1)(1-\lambda_{u_o}^{2M_o})
		\right)
	\end{aligned}
\end{equation}
The continuity of the flux at interface
$\rho_o \nu_o \phi_o = \rho_a \nu_a \phi_a$ lets us rewrite
\begin{equation}
A^1_0 = - B^1_0 \times \epsilon
	\frac{\nu_a}{\nu_o} \frac{K_{1,a}}{K_{1,o}} -
	\frac{S_{\phi,o}^0 - \epsilon \frac{\nu_a}{\nu_o}
	S^0_{\phi, a}}{K_{1,o}}
\end{equation}
and the jump between the solutions is
\begin{equation}
	\widehat{u}_{\frac{1}{2}} - \widehat{u}_{-\frac{1}{2}}
	= B^1_0 K_{2} + \Delta S
\end{equation}
where $K_{2} = 1 - \lambda_{u_a}^{2M_a} + (1 - \lambda_{u_o}^{2M_o})
\epsilon \frac{\nu_a}{\nu_o} \frac{K_{1,a}}{K_{1,o}}$
and $\Delta S = S_{u, a}^{m=\frac{1}{2}} - S_{u, o}^{m=-\frac{1}{2}}
+(1 - \lambda_{u_o}^{2M_o})\frac{S_{\phi,o}^0 - \epsilon \frac{\nu_a}{\nu_o}
	S^0_{\phi, a}}{K_{1,o}}$.
\subsection{Inverting the friction law}
After substitution of $\widehat{u}_{\frac{1}{2}}
- \widehat{u}_{-\frac{1}{2}}$ and $\phi_{{\rm a}, 0}$ in
the friction law \eqref{eq:OASchwarz_appendix_linearizedCond}, we
obtain a linear equation involving $B^1_0(s)$ and
$\overline{B^1_0(\overline{s})}$:
\begin{equation}
	\label{eq:OASchwarz_appendix_frictionLaw}
	\underbrace{(\nu_a K_{1,a} - \frac{3\alpha^e K_{2}}{2})}_{
		K_3(s)
	}B^1_0(s)
	-\underbrace{
		\alpha^e \frac{{\cal O} \overline{K_2}(\overline{s})}{2}}_{
		\overline{K_4(\overline{s})}}
	\times \overline{B^1_0(\overline{s})}
	= S_{\rm friction}
\end{equation}
where $S_{\rm friction} =
	\frac{3\alpha^e}{2} \left(
	\Delta S + \frac{\cal O}{3}
	\overline{\Delta S(\overline{s})}\right)
	+\nu_a S_{\phi,a}^{m=0}$.
The complex conjugate of \eqref{eq:OASchwarz_appendix_frictionLaw}
taken at $\overline{s}$ gives
\begin{equation}
\label{eq:OASchwarz_appendix_frictionLaw2}
	- K_4(s)B^1_0(s) +
	\overline{K_3(\overline{s}) B^1_0(\overline{s})} =
	\overline{S_{\rm friction}(\overline{s})}
\end{equation}
Finally, $B^1_0$  (and $\widehat{u}_j$) can be uniquely determined
if the system
(\ref{eq:OASchwarz_appendix_frictionLaw}-\ref{eq:OASchwarz_appendix_frictionLaw2})
has a unique solution, i.e. if the following matrix is invertible:
\begin{equation}
	\label{eq:OASchwarz_appendix_matrix_omega}
	\begin{pmatrix}
		K_3(s)& - \overline{K_4(\overline{s})}\\
		-K_4(s) & \overline{K_3(\overline{s})}
	\end{pmatrix}
\end{equation}
\begin{figure}
    \centering
    \includegraphics[scale=0.85]{isDetZero.pdf}
    \caption{Absolute value of the determinant of
	\eqref{eq:OASchwarz_appendix_matrix_omega} depending on
	the frequency variable. With our parameters,
	if $\sigma>0$ the matrix is invertible $\forall \omega$.}
    \label{fig:OASchwarz_isDetZero}
\end{figure}
Figure \ref{fig:OASchwarz_isDetZero} shows that the determinant of
this matrix is not zero for $\sigma>0$. For the frequencies
$\omega=f$ and $\omega=-f$ the determinant is close to zero, and
it seems to be asymptotically proportional to $\sqrt{\sigma}$.
\par
The Laplace transform can be uniquely inverted
(e.g. \citep{cohen_inversion_2007}) because
$\widehat{u}_j(s)$ does not grow faster than exponentially
for $|s|\rightarrow \infty$.
Indeed, the asymptotic values of $K_i$ for $|s|\to\infty$ are
\begin{equation*}
	\begin{aligned}
		K_{1,j} \sim -\frac{\chi_j}{h_j}, ~~~~~~~&
		K_{3} \sim -\frac{\nu_a \chi_a}{h_a}\\
		K_{2} \rightarrow 1 - \epsilon \frac{h_a}{h_o},
		~~~~~~~&
		K_{4} \rightarrow \alpha^e
		\frac{\cal O}{2}(1 - \epsilon \frac{h_a}{h_o})
	\end{aligned}
\end{equation*}
One can see that $K_3 \rightarrow \infty$ whereas $K_4$ tends to
a constant. Since $\widehat{u}$ is given by the inverse of
\eqref{eq:OASchwarz_appendix_matrix_omega}, it tends to zero as
$s\rightarrow \infty$.
Figure \ref{fig:OASchwarz_isDetZero} also
shows that the determinant of
\eqref{eq:OASchwarz_appendix_matrix_omega}
increases with $\omega^\beta$ (for some $\beta\approx 2$) for
high $|s|$.
