\begin{appendix}
\section{Discrete variation of parameters}
	\label{sec:OASchwarz_appendix_discreteVariationParameters}
We look for the solutions of the Fourier transform of equation
\label{eq:OASchwarz_DiscreteStationaryState_homogeneousEq1}:
\begin{subequations}
	\begin{align}
	(i\omega + if) \widehat{u}((m+\frac{1}{2}) h_j) -
		\nu_j \frac{\widehat{\phi}((m+1)h_j) -
		\widehat{\phi}(m h_j)}{h_j}
		&=U_0
	\hspace{1.4cm} \mbox{in}\;\widetilde{\Omega}_j \times
	(\omega_{\min},\omega_{\max})
		\label{eq:OASchwarz_appendix_eqFourier}\\
		U_j(z,t=0) &= U_0(z),   \hspace{2.8cm}
		\forall z \in \widetilde{\Omega}_j  \\
		\widehat{u}(x\rightarrow \pm \infty, \omega) &= 0,
		\hspace{3.1cm}
		\omega \in [\omega_{\min},\omega_{\max}]
		\label{eq:OASchwarz_appendix_bdCond}
		\end{align}
		\end{subequations}
The set of solutions of the homogeneous version of
\eqref{eq:OASchwarz_appendix_eqFourier} is given in the
convergence analysis: $\widehat{\phi}_{\rm o}(-m h_{\rm o})
	= A (\lambda_{\rm o}+1)^m$
and
$\widehat{\phi}_{\rm a}(m h_{\rm a}) = B (\lambda_{\rm a}+1)^m$ 
with $\lambda_j = \frac{1}{2}\left(\chi_j - \sqrt{\chi_j} \sqrt{\chi_j + 4}\right)$, 
$\chi_j=\frac{i (\omega+f) h_j^2}{\nu_j}$ and $m$ the space index.
We extend the well-known method of variation of parameters to the
discrete case to compute a solution of
\eqref{eq:OASchwarz_appendix_eqFourier}. 
Assuming $B_m$ depends on $m$ (the case $z<0$ is identical
to the case $z>0$ detailed here),
\begin{itemize}
\item if $\omega=-f$: $\lambda_o=0$ and
	\begin{equation}
		B_{m+1} - B_m = -\frac{h_j}{\nu_j} U_0(m h_j)
	\end{equation}
		$\widehat{\phi}(m h_2)= \frac{h_2}{\nu_2}
		\sum_{i=m}^{\infty} U_0(i h_2)$ is a solution of
		\eqref{eq:OASchwarz_appendix_eqFourier}
		with $\omega=-f$ \myTD{condition of convergence
		of series},
		to the condition that this solution is compatible
		with \eqref{eq:OASchwarz_appendix_bdCond}:
		since $U_j((m+1/2)h_j) - U_j((m-1/2)h_j) =
		h_j \phi_j(m h_j)$,
		the condition is
	\begin{equation}
		\label{eq:OASchwarz_appendix_constraints}
		u(\frac{h_2}{2}) +
		h_2 \sum^\infty_{m=1} \widehat{\phi}_2(m h_2)
		= u(-\frac{h_1}{2})-
		h_1 \sum^{-1}_{-\infty} \widehat{\phi}_1(m h_1) = 0
	\end{equation}
	There are two equalities in
	\eqref{eq:OASchwarz_appendix_constraints}: they
	allow to prescribe $u$ everywhere in the domain.
	We will end up with a constraint on $U_0$ when
	introducing the linearized interface condition.
\item if $\omega \neq -f$:
	let $B^1, B^2$ and $\lambda_u$ such that
	the solutions of the homogeneous version of
	\eqref{eq:OASchwarz_appendix_eqFourier} are
		$\widehat{u} = B^1_m \lambda_u^m +
			B^2_m \lambda_u^{-m}$.
	Equation \eqref{eq:OASchwarz_appendix_eqFourier}
	can be rewritten as
	\begin{equation}
		(\widehat{u}_{m+1} - \widehat{u}_{m})
		+\chi_2 \widehat{u}_{m}
		- (\widehat{u}_{m} - \widehat{u}_{m-1})
		= U_0
	\end{equation}
	Injecting $\widehat{u} = B^1_m \lambda_u^m +
	B^2_m \lambda_u^{-m}$ and rearranging terms gives
\begin{equation}
	\begin{aligned}
	&B^1_m \underbrace{\left(
		\lambda_u^{m+1} - \lambda_u^m + \chi_2 \lambda_u^m
		- (\lambda_u^{m} - \lambda_u^{m-1})
		\right)}_{=0} ~~~~~~~+ B^2_m \times 0 \\
	&+ \lambda_u^{m+1} (B_{m+1}^1 - B_m^1)
		- \lambda_u^{m} (B_{m}^1 - B_{m-1}^1)
		~~~~+ \lambda_u^{-m-1} (B_{m+1}^2 - B_m^2)
		- \lambda_u^{-m} (B_{m}^2 - B_{m-1}^2)
		\\
	&+ (\lambda_u^m - \lambda_u^{m-1})(B_{m}^1 - B_{m-1}^1)
		~~~~~~~~~~~~~~+ (\lambda_u^{-m} - \lambda_u^{-m+1})
				(B_{m}^2 - B_{m-1}^2)\\
		&~~~~~~~~~~~~= U_0
	\end{aligned}
\end{equation}
A particular solution of \eqref{eq:OASchwarz_appendix_eqFourier}
can be taken as
\begin{equation}
\begin{cases}
	\lambda_u^m(B_m^1 - B_{m-1}^1)
	+ \lambda_u^{-m}(B_m^2 - B_{m-1}^2)&= 0 \\
	(\lambda_u^m - \lambda_u^{m-1})(B_m^1 - B_{m-1}^1)
	+ (\lambda_u^{-m} - \lambda_u^{-m+1})(B_m^2 - B_{m-1}^2)
	&= {U_0}
\end{cases}
\end{equation}
which has a unique solution:
\begin{equation}
	\begin{pmatrix}
B_m^1 - B_{m-1}^1\\
B_m^2 - B_{m-1}^2
	\end{pmatrix}
	=
	\begin{pmatrix}
		\lambda_u^m & \lambda_u^{-m} \\
		(\lambda_u^m - \lambda_u^{m-1})  & 
		(\lambda_u^{-m} - \lambda_u^{-m+1})
	\end{pmatrix}^{-1}
	\begin{pmatrix}
		0 \\ U_0
	\end{pmatrix}
\end{equation}
The determinant of the inversed matrix is
$\lambda_u^{-1} - \lambda_u$ which is not zero since
$\omega\neq -f$.

\myTD{Normalement ca met une contrainte du style $U_0$ doit
décroitre plus vite que $\lambda_u^m$ pour avoir les conditions
aux bords}
\end{itemize}
\end{appendix}
