\subsubsection{Existence of solutions of the
non-linear semi-discrete in space problem}
In \citep{chacon-rebollo_existence_2014},
the existence of unsteady solutions of
non-linear turbulent models for oceanic surface mixing layers is
proven.
The proof relies on three main points:
\begin{enumerate}
	\item the existence of a steady-state;
	\item the well-posedness of the linearized problem
	around the equilibrium;
	\item the use of the inverse function theorem;
\end{enumerate}
The method can be generalized to other types of
non-linearities. In particular, we show the existence and
uniqueness of problem \eqref{eq:SWRbulk}
(ignoring the Schwarz iteration indices) close to the
equilibrium state.
\begin{enumerate}
	\item The existence of a steady state is discussed in
	Section \ref{sec:OASchwarz_DiscreteStationaryState}.
	\item Studying the well-posedness of the linearized
	problem is close to study the convergence of the SWR
	algorithm.
	First, we consider the following system obtained
	by substracting the stationary state to
		the linearized version of \eqref{eq:SWRbulk}:
	\begin{subequations}
	\label{eq:OASchwarz_DiscreteStationaryState_homogeneousEq}
		\begin{align}
		(\partial_t + if) &U_j - \nu_j\partial_z \phi_j=0, 
			\hspace{3.4cm} \mbox{in}\;\widetilde{\Omega}_j \times (0,T)
	\label{eq:OASchwarz_DiscreteStationaryState_homogeneousEq1}\\
		U_j(z,0) &= U_0(z),   \hspace{4.8cm}  \forall z \in \widetilde{\Omega}_j  \\
		U_j(H_j, t) &= 0, \hspace{5.1cm}  t \in [0,T]\\
		\nu_{\rm a}\phi_{\rm a}(0,t) &= \frac{3\alpha^e}{2} 
\left( \delta U_{\rm a} \right. - \delta U_{\rm o}
			\left.+ \frac{{\cal O}}{3} \overline{
	\delta U_{\rm a} - \delta U_{\rm o}} \right), 
		\hspace{0.12cm} t \in [0,T]\\
		\rho_{\rm o} \nu_{\rm o}\phi^k_{\rm o}(0,t) &= \rho_{\rm a}
		\nu_{\rm a}\phi^k_{\rm a}(0,t), \hspace{3.98cm} t \in [0,T]
		\end{align}
		\end{subequations}
	A Fourier transform of
	\eqref{eq:OASchwarz_DiscreteStationaryState_homogeneousEq1}
	can be solved with the discrete variation of parameters
	(see appendix) and we
	obtain under strict conditions on $U_0$:
	\begin{equation}
		\widehat{U}_1 &= A\lambda_a^m P_1(U_0, \omega) \\
		\widehat{U}_2 &= B\lambda_o^{-m} P_2(U_0, \omega)
	\end{equation}
	where P \myTD{will be} defined in
	\ref{sec:OASchwarz_appendix_discreteVariationParameters}.
	To obtain only one degree of freedom in each domain,
	the Dirichlet boundary condition at $x\rightarrow \pm \infty$
	was used.
	The linearized problem is well posed iff
	the two interface conditions determine $A$ and $B$.
	In the particular frequency of $\omega=-f$,
	$\lambda_a=\lambda_o=1$. The two interfaces conditions
	lead to a very strong constraint on $U_0$.
	\myTD{développer}
	The other frequencies do not impose such constraint on
	$U_0$.
\end{enumerate}
