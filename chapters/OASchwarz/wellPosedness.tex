\subsubsection{Existence of solutions of the
non-linear semi-discrete in space problem}
\cite{lions_mathematical_1995} proved the existence
of global in time weak solutions of the linear case in
two dimensions with more sophisticated dynamics.
\myTD{vérifier que ca correspond au cas constant ?
les viscosités sont pas constantes il me semble ?}
We focus here on showing the existence of strong
unsteady solutions of the semi-discrete in space,
non-linear problem.
\myTD{Note: je parle pas de fibrillation parceque c'est vraiment
relié au schéma de turbulence, rien à voir avec ici.}
\\
In \citep{chacon-rebollo_existence_2014},
the existence of unsteady solutions of
non-linear turbulent models for oceanic surface mixing layers is
proven.
The proof relies on three main points:
\begin{enumerate}
	\item the existence of a steady-state;
	\item the well-posedness of the linearized problem
	around the equilibrium;
	\item the use of the inverse function theorem;
\end{enumerate}
The method can be generalized to other types of
non-linearities. In particular, we show the existence and
uniqueness of problem \eqref{eq:SWRbulk}
(ignoring the Schwarz iteration indices) close to the
equilibrium state.
\begin{enumerate}
	\item The existence of a steady state is discussed in
	Section \ref{sec:OASchwarz_DiscreteStationaryState}.
	\item The well-posedness is discussed in Appendix
	\ref{sec:OASchwarz_appendix_discreteVariationParameters}.
	\item The use of the inverse function theorem can be
		done in four steps:
	\begin{enumerate}
		\item concatenate the state vectors
			in a single vector $\mathbf{U}=
			\{u_a, u_o, \left.\phi_a\right|_{z=0}\}
			\in \mathbf{\cal U}$
		where $\mathbf{\cal U} = L^2([0,T])^{M_a+M_o+1}$.
		\item
	Define a mapping
	$\mathbf{\Phi}: \mathbf{\cal U}\rightarrow \mathbf{Y}$
	such that
\begin{equation}
	\label{eq:OASchwarz_wellPosedness_Phi}
\begin{aligned}
	\mathbf{\Phi}(\mathbf{U}) =
	\{&(\partial_t + if) u_a - \nu_a\partial_z \phi_a - g_a, \\
	&(\partial_t + if) u_o - \nu_o\partial_z \phi_o - g_o, \\
	&u_a(H_a, t) - u^\infty_a, ~~ u_o(H_o, t) - u^\infty_o, \\
	&\nu_{\rm a}\left.\phi_{\rm a}\right|_{z=0} - \alpha
	\left( U_{\rm a}\left(\frac{h_{\rm a}}{2},
	t\right) - U_{\rm o}\left(-\frac{h_{\rm o}}{2},
	t\right)\right) \\
	&\left.u_a\right|_{t=0} - U^e_{a}, ~~
	\left.u_o\right|_{t=0} - U^e_{o}\}
\end{aligned}
\end{equation}
where $\left.\phi_j\right|_{z\neq 0} = \partial_z u_j$
and $\partial_z$ is to be undertood as the finite difference
operator which is applied only where it makes sense:
for instance the first line of \eqref{eq:OASchwarz_wellPosedness_Phi}
is not applied for $z=H_a$.
Let us draw some important remarks about $\mathbf{\Phi}$:
\begin{itemize}
 \item The equation $\phi_o = \epsilon \frac{\nu_a}{\nu_o}\phi_a$ is
	 implicit;
 \item $\mathbf{\Phi}(\mathbf{U}^e)=0$
 	where $\mathbf{U}^e$
 	is the steady-state;
 \item the codomain $\mathbf{Y}$ is
 	\begin{equation}
		\mathbf{Y}=L^2([0,T])^{M_a-1}
			\times L^2([0,T])^{M_o-1}
 			\times L^2([0,T])^{3}
			\times \mathbb{R}^{M_a+M_o}
 	\end{equation}
\item 
finding $\mathbf{\Phi}^{-1}(y)$ is equivalent
to solving the nonlinear semi-discrete problem
\eqref{eq:SWRbulk} if the component of $y$ corresponding
to the interface condition is zero.
The idea of the proof is that if $\mathbf{\Phi}$ is invertible
around $\mathbf{U}^e$ then the nonlinear semi-discrete problem
\eqref{eq:SWRbulk} is invertible.
\myTD{Note: normalement bien posé ça veut dire que la solution
	dépend continuement des variables d'entrées:
	faudra le rajouter quelque part dans l'appendice
		et à la fin de cette démo.}
\end{itemize}
		\item Show that $\mathbf{\Phi}$ is $C^1$ in a
			neighborhood of $\mathbf{U}^e$.
			First, note that the mapping
			$x \mapsto |x|x$ is analytic in a
			ball that does not contain zero.
			Since the non-linear friction condition
			is the only source of
			non-linearity in $\mathbf{\Phi}$,
			it directly follows that
			$\mathbf{\Phi}$ is $C^1$ and that
			its differential
			$D\mathbf{\Phi}(\mathbf{U}^e)$ is given
			by the linearized problem (for more
			details see
			\cite{chacon-rebollo_existence_2014}).
		\item Show that $D\mathbf{\Phi}(\mathbf{U}^e)$
		is an isomorphism:
		$D\mathbf{\Phi}(\mathbf{U}^e)$ can
		be inverted by solving the linearized problem
		with additional input data.
		It is \myTD{not completely} shown
		that it is well-posed in appendix
		\ref{sec:OASchwarz_appendix_discreteVariationParameters}.
	\end{enumerate}
\end{enumerate}
