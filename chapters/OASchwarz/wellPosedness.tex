\subsubsection{Existence of solutions of the
non-linear semi-discrete in space problem}
\myTD{Fibrillations biblio?}
\\
\myTD{Lions: rééférence dans le cas couplé (voir mail)}
\\
\myTD{Renault, Lemarié pour bien posé}
In \citep{chacon-rebollo_existence_2014},
the existence of unsteady solutions of
non-linear turbulent models for oceanic surface mixing layers is
proven.
The proof relies on three main points:
\begin{enumerate}
	\item the existence of a steady-state;
	\item the well-posedness of the linearized problem
	around the equilibrium;
	\item the use of the inverse function theorem;
\end{enumerate}
The method can be generalized to other types of
non-linearities. In particular, we show the existence and
uniqueness of problem \eqref{eq:SWRbulk}
(ignoring the Schwarz iteration indices) close to the
equilibrium state.
\begin{enumerate}
	\item The existence of a steady state is discussed in
	Section \ref{sec:OASchwarz_DiscreteStationaryState}.
	\item The well-posedness is discussed in Appendix
	\ref{sec:OASchwarz_appendix_discreteVariationParameters}.
	\item The use of the inverse function theorem can be
		done in four steps:
	\begin{enumerate}
		\item concatenate $u_a, \phi_a, u_o, \phi_o$ in
		a single vector $\mathbf{U}\in \mathbf{\cal U}$
		where
		\begin{equation}
			\mathbf{\cal U} = L^2([0,T])^{M_o} \times
			L^2([0,T])^{M_o} \times L^2([0,T])^{M_a}
			\times L^2([0,T])^{M_a}
		\end{equation}
		\item
		Define a mapping
		$\mathbf{\Phi}: \mathbf{\cal U}\rightarrow \mathbf{Y}$
		such that finding $\mathbf{\Phi}^{-1}(y)$ is equivalent
		to solving the nonlinear semi-discrete problem
			\eqref{eq:SWRbulk}.
		\myTD{define $\mathbf{\Phi}$ explicitly}
		\item Show that $\mathbf{\Phi}$ is $C^1$.
			First, note that the mapping
			$x \mapsto |x|x$ is analytic.
			\myTD{Faudrait clarifier du coup:
			on fait de l'analyse réelle ou complexe ?
			comment on traduit le théorème de
			l'inverse en analyse complexe ?}
		\item Show that $D\mathbf{\Phi}(\mathbf{U}^e)$
			is an isomorphism \myTD{de où vers Y?}.
		$D\mathbf{\Phi}(\mathbf{U}^e)$ can
		be inverted by solving the linearized problem which
		is well-posed.
	\end{enumerate}
\end{enumerate}
