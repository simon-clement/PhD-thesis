\section{Detailed convergence study of the Linearized case}
\label{appendix:OASchwarz_LinearizedAnalysis}
We interest ourselves to the linearization around a stationary state 
$U^e_j, \phi^e_j$.

\begin{equation*}
\begin{array}{rcll}
	(\partial_t +if) u^{{k}}_{j,m+\frac{1}{2}} &=&
	\nu_j\frac{\phi^{{k}}_{j,m+1} - \phi^{{k}}_{j,m}}{h} \\
	\left.u^{{k}}_j\right|_{t=0} &=& 0 \\
	\left.u^{{k}}_j\right|_{z=H_j} &=& 0\\
	\rho_{\rm o} \nu_{\rm o} \phi^{{k}}_{{\rm o}, 0} &=&
	\rho_{\rm a} \nu_{\rm a} \phi^{{k}}_{{\rm a}, 0} \\
	\nu_a \phi_{{\rm a},0}^{{k}}
	&=& C_D \left|\Delta U^e\right| \left(\frac{3}{2} \Delta u^{
		{k-1}}
	+ \frac{1}{2} \frac{\Delta U^e}{\overline{\Delta U^e}}
		\overline{\Delta u^{
			{{k-1}}}}\right) \\
		&&-{\theta \; C_D|\Delta U^e|
		\left( u_a^{{k}} -
		u_a^{{k-1}}\right)}
\end{array}
\end{equation*}
The three first equations are used in a Fourier transform or
z-transform to obtain
\begin{equation}
\label{eq:FourierTransform}
\begin{array}{rcll}
	(\emphase{s} +if) \widehat{u}^{{k}}_{j,m+\frac{1}{2}} &=&
	\nu_j\frac{\widehat{\phi}^{{k}}_{j,m+1}
	- \widehat{\phi}^{{k}}_{j,m}}{h} \\
	\left.\widehat{u}^{{k}}_j\right|_{z=H_j} &=& 0\\
\end{array}
\end{equation}
where $s$ is the frequency variable.
Using the form of $\phi_{j, m \neq 0} = \frac{u_{j, m+1/2} -
u_{j, m-1/2}}{h}$
we obtain for $m>0$
\begin{equation}
\label{eq:diff_finies_on_u}
	(\chi_j+2) \widehat{u}^{{k}}_{j,m+\frac{1}{2}} =
	\widehat{u}_{j, m+3/2} + \widehat{u}_{j, m-1/2},
	~~~~~~ \chi_j = h_j^2\frac{s + if}{\nu}
\end{equation}
We use the ansatz
\begin{equation*}
\begin{aligned}
\widehat{u}_{a,m+1/2} &= A_k^a (1+\lambda_a)^m
+ \widetilde{A}_k^{a}(1+\widetilde{\lambda_a})^{m} \\
\widehat{u}_{o,m-1/2} &= A_k^o (1+\lambda_o)^{m}
+ \widetilde{A}_k^{o}(1+\widetilde{\lambda_o})^{m}
\end{aligned}
\end{equation*}
where $\lambda_j, \widetilde{\lambda_j}$ can be obtained by injecting
the ansatz in \eqref{eq:diff_finies_on_u};
\begin{equation*}
\begin{aligned}
\forall m>0,~~~ &A_k^j\left(1+\lambda_j\right)^{m-1}
\left((\chi_j+2) (1+\lambda_j) - (1+\lambda_j)^2 - 1\right)\\
=&
\widetilde{A}_k^j\left(1+\widetilde{\lambda_j}\right)^{m-1}
\left((\chi_j+2) (1+\widetilde{\lambda_j}) - (1+\widetilde{\lambda_j})^2 - 1\right) = 0
\end{aligned}
\end{equation*}
The solutions of this equation are
$\lambda_j=\widetilde{\lambda_j}=-1$ (corresponding to
$\widehat{u}_j=0$) and
$\lambda_j =
\frac{1}{2}\left(\chi_j - \sqrt{\chi_j}\sqrt{\chi_j+4}\right),
\widetilde{\lambda_j}=\chi_j - \lambda_j$. One can prove
that $(1+\lambda_j)(1+\widetilde{\lambda_j}) = 1$.

The boundary condition at $H_j = (M_j + \frac{1}{2}) h_j$ gives that
$\widetilde{A}_k^j = - A_k^j \left(1+\lambda_j\right)^{2 M_j}$.
Finally, the jump of the solution across interface is
\begin{equation*}
\widehat{u}_{a, 1/2} - 
\widehat{u}_{o, -1/2} = A_k^a - A_k^o
- A_k^a\left(1+\lambda_a\right)^{2M_a}
+ A_k^o\left(1+\lambda_o\right)^{2M_o}
	:= A_k^a I_a^{-} - A_k^o I_o^{-}
\end{equation*}
and we also define $I_j^{+} = 2-I_j^{-}$. All the variables
$I_j^{\pm}\rightarrow 1$ with the infinite domain hypothesis.
The fluxes inside the domains are computed
with the finite differences:
\begin{equation}
	\label{eq:OASchwarz_appendix_phiaDiscrete}
\phi_{a,m+1} = A_k^a \frac{\lambda_a}{h_a}
	(1+\lambda_a)^m + \widetilde{A}_k^a \frac{\widetilde{\lambda_a}}{h_a}
	(1+\widetilde{\lambda_a})^m
\end{equation}
\begin{equation}
	\label{eq:OASchwarz_appendix_phioDiscrete}
\phi_{o,m-1} = -A_k^o \frac{\widetilde{\lambda_o}}{h_o}
	(1+\lambda_o)^m - \widetilde{A}_k^o \frac{{\lambda_o}}{h_o}
	(1+\widetilde{\lambda_o})^m
\end{equation}
And the fluxes at interface are obtained 
by using \eqref{eq:FourierTransform}:
\begin{equation*}
\begin{aligned}
h_a\phi_{a,0}&= h_a\phi_{a,1} - \chi_a \widehat{u}_{a, 1/2}
= A_k^a \left(\lambda_a - \chi_a\right) +
	\widetilde{A}_k^a\left(\widetilde{\lambda_a} - \chi_j\right)
	= A_k^a ({\lambda_a} I_a^{+} - \chi_a)
\\
h_o\phi_{o,0}&= h_o\phi_{o,-1} + \chi_o \widehat{u}_{o, -1/2}
	= A_k^o \left(-\widetilde{\lambda_o} + \chi_o\right) +
	\widetilde{A}_k^o\left(-{\lambda_o} + \chi_o\right)
	=A_k^o(({\lambda_o}-\chi_o)I_o^{+} + \chi_o)
\end{aligned}
\end{equation*}
\begin{remark}
It can be checked that $\phi_{a,0}$ and $\phi_{o,0}$
correspond to the values obtained with respectively
$m=-1$ and $m=1$ in
\eqref{eq:OASchwarz_appendix_phiaDiscrete} and
\eqref{eq:OASchwarz_appendix_phiaDiscrete}. This
is not obvious since they do not correspond to a finite
difference approximation $\frac{u_{j, \frac{1}{2}} -
u_{j, -\frac{1}{2}}}{h_j}$.
It is hence correct if $\omega \neq -f$ to start from
$\phi_j(\pm m h_j)= X_k(\lambda_j+1)^m$ in Sections
	\ref{sec:conv-lin} and \ref{sec:OASchwarz_DiscreteStationaryState}.
\end{remark}
We are now ready to study the transmission conditions.
The continuity of the flux leads to
\begin{equation}
    h_o \widehat{\phi}_{o,0} = h_a \widehat{\phi}_{a,0}
    \epsilon\frac{\nu_a h_o}
    {\nu_o h_a}
    \Rightarrow
	A_k^o = A_k^a \frac{\lambda_a I_a^{+} - \chi_a}
	{(\lambda_o - \chi_o)I_o^{+} + \chi_o}
    \epsilon\frac{\nu_a h_o}
    {\nu_o h_a}
\end{equation}
We inject this equality in the jump of the solution across
the interface and obtain
\begin{equation}
    \widehat{u}_{a,\frac{1}{2}}^k - \widehat{u}_{o,-\frac{1}{2}}^k
    = A_k^a \mu(s)
\end{equation}
where
\begin{equation}
	\mu(s) = I_a^{-} - I_o^{-}
\frac{\lambda_a I_a^{+} - \chi_a}
	{(\lambda_o - \chi_o)I_o^{+} + \chi_o}
    \epsilon\frac{\nu_a h_o}
    {\nu_o h_a}
\end{equation}
The second interface transmission condition which
is linearized gives
\begin{equation}
\frac{\nu_{\rm a}}{h_a}A_k^a \left(\lambda_a I_a^{+} - \chi_a \right)
	=  \alpha^e \left(\frac{3}{2}A_{k-1}^a \mu(s)
+\theta ( \widehat{u}_{\rm a}^{k} -  \widehat{u}_{\rm a}^{k-1})
+ \frac{1}{2}\frac{\Delta U^e}{\overline{\Delta U^e}}
\overline{A_{k-1}^a(\overline{s}) \mu(\overline{s})}
\right)
\end{equation}
where
\begin{equation}
	\widehat{u}_{\rm a}^{k} -  \widehat{u}_{\rm a}^{k-1} = (A_k^a - A_{k-1}^a)I_a^{-}
\end{equation}
We finally obtain
\begin{equation}
    A_k^a = a_1 A_{k-1}^a + a_2 \overline{A_{k-1}^a}
\end{equation}
where
\begin{equation}
    a_1 = \alpha^e\frac{\frac{3}{2}\mu(s) - \theta I_a^{-}}
	{\frac{\nu_a}{h_a}\left(\lambda_a I_a^{+} - \chi_a \right)
	- \alpha^e \theta I_a^{-}}, ~~~~~~~
        a_2 = \alpha^e\frac{\frac{1}{2}\frac{\Delta U^e}{\overline{\Delta U^e}}
 \overline{\mu(\overline{s})}}
	{\frac{\nu_a}{h_a}\left(\lambda_a I_a^{+} - \chi_a \right)
	- \alpha^e \theta I_a^{-}}
\end{equation}
