\chapter{Ocean-atmosphere vertical column modelling}
\label{ch:airseaSCM}
\minitoc
We present in this chapter the concepts and equations in
the continuous level which
we will use in the rest of this thesis.
In particular, in Section \ref{sec:airseaSCM_primitiveEquations}
the turbulent primitive equations are derived then
the surface layer particularities are described in Section
\ref{sec:airseaSCM_SL}.
Finally, the models used in this thesis are summed up
in Section \ref{sec:airseaSCM_hierarchy}
before giving the context and objectives of this thesis
(Section \ref{sec:airseaSCM_context_objectives}).
\section{Derivation of the primitive equations with turbulence}
\label{sec:airseaSCM_primitiveEquations}
This section aims to present the so-called
primitive equations, which are used to model
the inner parts of both the atmosphere and ocean.
\subsection{Non-turbulent primitive equations}
We start from the Navier-Stokes momentum equation in a rotating frame:
\begin{equation}
\label{eq:airseaSCM_momentumNavierStokes}
\rho \left(\partial_t + \mathbf{U} \cdot \nabla \right) \mathbf{U}
= - \nabla p + \nu^m \Delta \mathbf{U} - \rho \mathbf{g}
- 2 \mathbf{\Omega} \times (\rho \mathbf{U})
\end{equation}
where the symbols are given in Table
\ref{tab:airseaSCM_primitiveEquationsSymbols}
\begin{table}
\centering
\begin{tabular}{c|c|c}
Symbol & Quantity & Unit \\
\hline
$\rho$& {Density} & ${\rm kg}.{\rm m}^{-3}$ \\
$p$   & {Pressure} & ${\rm kg}.{\rm m}^{-1}.{\rm s}^{-2}$ \\
$\mathbf{U}$ & {Velocity}& ${\rm m}.{\rm s}^{-1}$ \\
$\nu^m$ & {Molecular viscosity}& ${\rm m}^{-1}.{\rm s}^{-2}$ \\
$\mathbf{g}$ & {Gravity acceleration}& ${\rm kg}.{\rm m}.{\rm s}^{-2}$ \\
$\mathbf{\Omega}$ & {Angular speed}& ${\rm rad}.{\rm s}^{-1}$ \\
$\nabla$ & {Spatial Gradient} & ${\rm m}^{-1}$\\
$\Delta$ & {Spatial Laplacian} &${\rm m}^{-2}$
\end{tabular}
\caption{Symbols used in Section
\ref{sec:airseaSCM_primitiveEquations}. Bold letters
indicate that the quantity lies in $\mathbb{R}^3$.}
\label{tab:airseaSCM_primitiveEquationsSymbols}
\end{table}
and the \textit{continuity equation} which ensures
conservation of mass:
\begin{equation}
\label{eq:airseaSCM_conservationMass}
\partial_t \rho = - \nabla \cdot (\rho \mathbf{U})
\end{equation}
\myTD{J'ai paraphrasé la thèse de Sophie ici}
It is now possible to simplify the two equations
\eqref{eq:airseaSCM_momentumNavierStokes} and
\eqref{eq:airseaSCM_conservationMass} with
some usual approximations:
\begin{itemize}
\item \textit{Spherical geoid approximation}:
we assume that the earth is spherical and that the
gravity acceleration is given by
$\mathbf{g} =\begin{pmatrix}
0\\ 0 \\ g
\end{pmatrix}$ where $g=9.81 {\rm m}.{\rm s}^{-2}$.
\item \textit{Traditional approximation}:
We neglect the horizontal terms of
$\mathbf{\Omega}$: we assume that
$\mathbf{\Omega} =\begin{pmatrix}
0\\ 0 \\ f/2
\end{pmatrix}$ where $f$ depends on the latitude.
\item \textit{Hydrostatic fluid}:
the pressure gradient balances the gravity force:
$\partial_z p = -\rho g$. We neglect the vertical
acceleration due to pressure and gravity.
This approximation is usually done for large time
scales like the one used in climate simulations.
\item \textit{Boussinesq approximation}:
the density $\rho$ is close to a constant $\rho_0$.
The variation of density $\widetilde{\rho} =
\rho - \rho_0$ is neglected except when computing
the pressure gradient $\partial_z p = - \rho g$.
\myTD{citer Boussinesq}
Despite this approximation being very rough
for the atmosphere, we use it for both domains
to keep some symmetry of our equations.
\end{itemize}
We hence obtain the \textit{non-turbulent}
primitive equations:
\begin{equation}
	\label{eq:airseaSCM_nonTurbulentPrimitiveEq}
\begin{cases}
	\nabla \cdot \mathbf{U} &= 0 \\
	\partial_t u + \nabla \cdot (\mathbf{U} u) &=
	- \frac{\partial_x p}{\rho_0} + \nu^m \Delta u
	+ f v \\
	\partial_t v + \nabla \cdot (\mathbf{U} v) &=
	- \frac{\partial_y p}{\rho_0} + \nu^m \Delta v
	- f u \\
	\partial_z p &= -\rho g
\end{cases}
\end{equation}
where we noted $\mathbf{U} = \begin{pmatrix}u\\v\\w\end{pmatrix}$.
$\rho$ is given by an equation of state, which will only depend on
the temperature in our case. \myTD{Ajouter une équation de température !}
\subsection{Reynolds decomposition}
The solution of the system \eqref{eq:airseaSCM_nonTurbulentPrimitiveEq}
contains very small scales which cannot be solved numerically for
large domains such as the ocean or the atmosphere. It is hence usual
to use the \textit{Reynolds decomposition}, which consists in
separatinig the variables into an "average" part which can be
represented by the numerical schemes and a fluctuating part which
will be parameterized with turbulent closure schemes
(detailed later in \S \ref{sec:airseaSCM_turbulentClosure}).
\par
For $X=u, v, w, p$ or $\theta$, we note
\begin{equation}
	X = \langle X\rangle + X'
\end{equation}
where $\langle \cdot \rangle$ represents a \textit{statistical}
average which satisfies:
\begin{equation}
\begin{aligned}
	\langle X' \rangle &= 0 \\
	\langle \partial_\beta X \rangle &=
	\partial_\beta \langle  X \rangle, ~~~\beta=x,y,z,t \\
	\langle\langle \cdot \rangle\rangle &= \langle \cdot \rangle \\
	\langle \langle X \rangle Y\rangle &= \langle X \rangle\langle Y \rangle
\end{aligned}
\end{equation}
using those properties, the Reynolds decomposition of 
\eqref{eq:airseaSCM_nonTurbulentPrimitiveEq} gives:
\begin{equation}
	\label{eq:airseaSCM_TurbulentPrimitiveEq}
\begin{cases}
	\nabla \cdot \langle\mathbf{U}\rangle &= 0 \\
	\partial_t \langle u \rangle+ \nabla \cdot
	(\langle\mathbf{U}\rangle \langle u\rangle) &=
	- \frac{\partial_x \langle p\rangle}{\rho_0} +
	\nu^m \Delta \langle u\rangle
	+ f \langle v\rangle - \nabla \cdot \langle
	\mathbf{U}' u'\rangle\\
	\partial_t \langle v\rangle + \nabla \cdot
	(\langle \mathbf{U}\rangle \langle v\rangle) &=
	- \frac{\partial_y \langle p\rangle}{\rho_0} +
	\nu^m \Delta \langle v\rangle
	- f \langle u\rangle  - \nabla \cdot \langle
	\mathbf{U}' v'\rangle\\
	\partial_z\langle p\rangle &= -\rho g
\end{cases}
\end{equation}
However, with this decomposition we obtain more unknown than
equations: this is known as the turbulent closure problem.
\subsection{Turbulent closure and kinetic energy}
\label{sec:airseaSCM_turbulentClosure}
We now provide additional relations between
$\langle \mathbf{U}' u'\rangle, \langle \mathbf{U}' v'\rangle,\langle \mathbf{U}' w'\rangle,
\langle \mathbf{U}' \theta'\rangle$ and the average quantities
$\langle \mathbf{U}\rangle, \langle \theta \rangle, \langle p \rangle$.
\paragraph{Boussinesq hypothesis}
From the observation that the turbulent term acts like a diffusion
term oriented "down-gradient", the turbulent terms are approximated
with the help of a "turbulent viscosity" $\nu \in \mathbb{R}^3$:
\begin{equation}
\langle \mathbf{U}' u'\rangle = - \nu \cdot \nabla u'
\end{equation}
\myTD{Bon ici c'est vraiment pas au point}
The turbulent viscosity can be defined in a lot of different ways.
An important quantity often used is the Turbulent Kinetic Energy (TKE):
\begin{equation}
	e = \frac{1}{2} \left(\langle (u')^2 \rangle + \langle (v')^2 \rangle
	+ \langle (w')^2 \rangle\right)
\end{equation}
in the turbulent closure model $k-\epsilon$,
the turbulent viscosity is then defined by
\begin{equation}
	\nu = C_m l_m \sqrt{e}
\end{equation}
where $C_m$ is a constant and $l_m$ is a length scale.
\myTD{equation en 3D plutôt, et voir quelles notations j'utilise:
ça serait bien d'avoir les mêmes dans toute la thèse !
A la limite je pourrais dire que K c'est une viscosité non constante
et $\nu$ une viscosité constante}
\begin{equation}
\label{eq:airseaSCM_TKE_evolution}
    \begin{aligned}
    \partial_t e =
    \underbrace{\partial_z \left(K_e
    \partial_z e\right)}_{\text{diffusion}}
    + \underbrace{K_u ||\partial_z u||^2}_{\text{shear}} 
    - \underbrace{K_{\theta} N^2 }_{\text{buoyancy}}
    - \underbrace{c_{\epsilon}
    \frac{e^{3/2}}{l_{\epsilon}(z)}}_{\text{dissipation}}
    \end{aligned}
\end{equation}
\myTD{etre clair dans l'intro de la TKE, et de la "production de TKE par le shear"}
\section{The surface layer}
\label{sec:airseaSCM_SL}
The ocean and atmosphere have not been distinguished in equations so far.
We will use a subscript "o" for the ocean quantities and "a" for the
atmosphere quantities.
Even if we assumed that \eqref{eq:airseaSCM_TurbulentPrimitiveEq}
describes both of them, the transition between the ocean and the
atmosphere is not straightforward to handle.
\myTD{Un mot sur l'interface plane ici pitet}
We call \textit{surface layer} the area close to the ocean surface
such that it responds to a change at the surface with a short
timescale.
We first restrict ourselves to the atmosphere then
present the oceanic surface layer in \S \ref{sec:airseaSCM_twoSided}.
\subsection{Law of the wall and Monin-Obukhov Similarity Theory}
For the atmosphere, the ocean surface can be seen as a
rough surface : the wind cannot slip on the ocean which
means that the velocity of the wind is zero
relatively to the surface.
\par
\myTD{copié-collé du chapitre 4:}
\par
In fluid dynamics, the presence of a rough surface makes strong
gradients appear:
due to the no-slip boundary condition
the scales of motion close to the surface are much smaller than
elsewhere in the domain and it is numerically
intractable in most applications
to refine the mesh enough for those small scales.
The models hence exclude from the computational domain
a part of the surface layer, using an adapted boundary
condition.
%
\paragraph{Neutral case}
We first present the case without stratification:
\citep{karman_mechanische_1930} noticed that the
fluids close to rough surfaces present similarities and
propose a universal function based on dimensional analysis.
Von Kàrmàn also states that "the characteristic length
of the flow pattern is proportional to
[the distance to the wall]".

\myTD{On pourra citer \citep{schlichting_boundary_1960} sur le
choix de $z_0$}
In the surface layer, the size of the turbulent eddies at height $z$
is proportional to the distance to the surface $z$
(e.g. \cite{kawai_wall-modeling_2012}).
The turbulent viscosity
linearly scales with $u_\star z$, where the coefficient of
proportionality is the Von Karman constant $\kappa = 0.40$.
\par
The surface layer has two main characteristics:
\begin{itemize}
	\item it is well mixed. The governing equation
		is usually stationary because the surface layer
		immediately adjusts to the external parameters.
		As a consequence (and because the Coriolis effect
		and the geostrophic forcing are neglected),
		the flux $K \partial_z u$
		is a constant along the vertical axis.
	\item The vertical profile of $K$ strongly depends 
		on $z$.
\end{itemize}
\myTD{dire que M-O
n'est pas universellement admis (mais qu'il est utilisé presque partout,
voir \citep{basu_cautionary_2017})}
\myTD{Petit schéma}
\myTD{IL FAUT ENCORE INTRODUIRE LES EQUATIONS DU BULK !!}
\subsection{Two-sided bulk and entire coupled system}
\label{sec:airseaSCM_twoSided}
In Chapter \ref{ch:OceanND} we will also consider the
\textit{oceanic surface layer}.
The sea surface temperature and the surface currents are often
evaluated below the surface.
Some efforts have been made to correct the satellite measurement.
\myTD{citer des trucs que cite charles ?}.
In \citep{pelletier_two-sided_2021} a bulk formulation which
uses Monin-Obukhov Similarity Theory (MOST) in the oceanic
surface layer is derived.
We follow this idea \myTD{et on en fait quoi ?}

\myTD{Finally, the coupled system with the surface layer is...}
\section{A hierarchy of models}
\label{sec:airseaSCM_hierarchy}
\subsection{Reaction-diffusion equations coupling
with heterogeneous diffusion}
\subsection{Ekman problem with a friction law}
\label{sec:airseaSCM_hierarchy_Ekman}
\myTD{introduire et détailler le principe de $u$ complexe}
\subsection{Ekman problem Neutral or Stratified with bulk and turbulent kinetic energy}
\myTD{Donner des noms: Reaction-diffusion / Ekman /
turbulent Ekman ?}
\section{Context and objective of this thesis}
\label{sec:airseaSCM_context_objectives}
