\clearpage
\phantomsection
\pdfbookmark[0]{Abstract (Français/Anglais)}{abstract}
\thispagestyle{empty}
\newgeometry{left=1.2cm,top=3cm, bottom=0.1cm, right=2cm}
\begin{tabular}{c|c}
{\begin{minipage}{18em}
Les modèles numériques de prévision de l'océan et de l'atmosphère
sont essentiels pour la compréhension des phénomènes géophysiques
qui y sont liés.
Le couplage de ces modèles dans les simulations
joue un rôle clé pour une large gamme d'échelles temporelles
(cycle diurne, cyclone tropicaux, climat...)
où il est nécessaire de représenter les interactions entre l'océan et
l'atmosphère.
\par
	\vspace{0.2cm}
L'implémentation du couplage est généralement réalisée de manière
partielle et introduit une erreur numérique qu'il convient de
minimiser. Dans ce but, des méthodes de couplage itératives et
leur vitesse de convergence sont considérées ici, les pratiques
actuelles étant souvent équivalentes à une seule itération.
\par
	\vspace{0.2cm}
Une difficulté dans l'analyse mathématique de la convergence
des méthodes de couplage est la présence d'une couche limite
de surface entre l'océan et l'atmosphère.
Cette spécificité justifie d'étudier la convergence au niveau
discret (c'est-à-dire en prenant en compte certains choix
d'implémentation) plutôt que continu.
\par
	\vspace{0.2cm}
Par ailleurs, les paramétrisations de la couche limite
de surface s'appuient sur des hypothèses
qui ne sont pas mathématiquement imposées au sein des modèles.
Cette thèse propose de renforcer la cohérence entre
le calcul des flux turbulents intervenant dans la couche limite
et les discrétisations des équations décrivant l'océan
et l'atmosphère.
\par
	\vspace{0.2cm}
L'analyse de la couche limite de surface au sein du couplage
océan-atmosphère est réalisée ici en utilisant une hiérarchie
de modèles permettant à la fois d'obtenir des résultats
mathématiques et de reproduire des comportements numériques d'intérêt.
\end{minipage}}
&
{\begin{minipage}{18em}
Numerical models of the ocean and atmosphere
are essential for the understanding of the
associated geophysical phenomena.
The coupling of these models plays a key role
for a wide range of time scales
(diurnal cycle, tropical cyclone, global climate...)
where it is necessary to represent the interactions
between the ocean and the atmosphere.
\par
	\vspace{0.2cm}
The implementation of the coupling is generally done only partially
and introduces a numerical error that should be minimized.
For this purpose, iterative coupling methods and their convergence speed
are considered here, the current practices being often equivalent to a
single iteration.
\par
	\vspace{0.2cm}
A difficulty in the mathematical convergence analysis
of coupling methods is the presence of a surface layer
between the ocean and the atmosphere.
This specificity justifies studying the convergence at the
discrete level (i.e. taking into account some implementation choices) rather than
at the continuous level.
\par
	\vspace{0.2cm}
Moreover, the parameterizations of the surface layer
are based on assumptions which are not mathematically
enforced within the models.
This thesis proposes to consolidate the coherence between
the computation of turbulent flows in the surface layer
and the discretizations of the equations describing the ocean
and the atmosphere.
\par
	\vspace{0.2cm}
The analysis of the surface boundary layer within the ocean-atmosphere coupling
is carried out here using a hierarchy
of models allowing to obtain both mathematical results and the
replication of numerical behaviors of interest.
\end{minipage}}
\end{tabular}
