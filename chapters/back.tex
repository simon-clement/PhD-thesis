\clearpage
\phantomsection
\pdfbookmark[0]{Abstract (Français/Anglais)}{abstract}
\thispagestyle{empty}
\begin{tabular}{c|c}
{\begin{minipage}{15em}
Le couplage océan-atmosphère est une composante essentielle
des modèles de prévision numérique. Il repose sur
le calcul et l'échange de flux turbulents à la surface au
sein de méthodes itératives.
\par
D'une part, le calcul des flux turbulents s'appuie sur
des paramétrisations de l'interface air-mer utilisant
des hypothèses sur les données qui ne sont pas
mathématiquement imposées au sein des modèles.
Cette thèse propose de renforcer le lien entre
le calcul des flux turbulents et les
représentations numériques -ou discrétisations-
des équations décrivant l'océan et l'atmosphère.
\par
D'autre part, les méthodes itératives de couplage utilisées
dans les modèles opérationnels introduisent une erreur
numérique qu'il convient de minimiser. Dans cet
objectif, la convergence de ces méthodes est ici analysée
sur des modèles simplifiés, tout en restant au plus près
de la discrétisation.
\par
En particulier, la couche limite de surface où s'échangent
les flux turbulents est intégrée dans l'analyse de la
convergence de ces méthodes de couplage.
Les caractéristiques de cette couche limite sont liées
aux choix dans la discrétisation, ce qui justifie d'étudier
au préalable l'effet de la discrétisation
dans la convergence des méthodes itératives de couplage.
\end{minipage}}
&
{\begin{minipage}{15em}
Ocean-atmosphere coupling is an essential component of numerical forecasting models.
It relies on the computation and exchange of turbulent flows at the surface within iterative methods.  
\par
On one hand, the computation of turbulent fluxes relies
on parameterizations of the air-sea interface that use
assumptions about the data that are not
mathematically imposed within the models.
This thesis proposes to strengthen the link between the computation
of turbulent fluxes and the numerical representations
(discretizations) of the equations describing the ocean and
atmosphere.
\par
On the other hand, the iterative coupling methods used
in operational models introduce a numerical error that
must be minimized. With this in mind, the convergence of
these methods is analyzed
here on simplified models, while remaining as close
as possible to the discretization.
\par
In particular, the surface boundary layer where turbulent flows
are exchanged is integrated in the
analysis of the convergence of these coupling methods.
The characteristics of this boundary layer are linked to
the choices in the discretization, which justifies to study
the effect of the latter in the convergence of iterative
coupling methods.
\end{minipage}}
\end{tabular}
