\chapter{Approximations of the discrete convergence factor of Schwarz methods }
\label{ch:approximatedDiscreteSchwarz}
\minitoc
In Chapter \ref{ch:discreteSchwarzAnalysis} the semi-discrete and
discrete convergence factors of Schwarz methods were studied.
As the complexity of the discretizations increase, it becomes
tedious to compute those convergence factors.
The goal of this chapter
is to ease the computations with approximations.
An article in preparation is reported here:
it discusses two different methods to approximate the discrete
or semi-discrete convergence factor without actually having to
compute them.
\begin{itemize}
\item The first method approximates a semi-discrete
(in space or time) convergence factor by using
\textit{modified equations}.
\item The second method combines the semi-discrete convergence
factors and the continuous one to get an approximation of the
discrete in space and time convergence factor.
\end{itemize}
Those two methods are then used in the optimization of the
convergence to evaluate the potential of these approximations
in the acceleration of Schwarz methods.
\par
The main results of this chapter are listed below.
\begin{itemize}
\item The convergence study with modified equations
	includes the main features of the semi-discrete
		schemes for low-frequencies.
\item Using the modified equations technique
	on a multi-step time scheme amounts to ignoring
	the presence of the intermediate step. We conclude
	that this technique should not be used for
	the approximation of the semi-discrete in time
	convergence factor.
\item For space schemes, using the modified equations simplifies
	the analysis of the convergence factor
	(compared to the semi-discrete level) only
	for a particular type of differential equations.
\item The combination of the semi-discrete analyses can provide
	a good approximation of the discrete convergence factor.
	For a large range of the problem parameters,
	the optimization performs better on the \textit{combined}
	convergence than on the continuous one. However,
	this improvement is not systematic and using this method
	on other problems without further investigations
	would be uncertain.
\end{itemize}

\setlength{\footskip}{110pt}
\invisiblesection{Introduction}
\includepdf[pages=1,pagecommand={}, scale=0.95, offset=0 -10]{paperApproximatedDiscreteSchwarz.pdf}
\invisiblesection{Discretisation and Schwarz methods studied}
\includepdf[pages=2-3,pagecommand={}, scale=0.95, offset=0 -10]{paperApproximatedDiscreteSchwarz.pdf}
\invisiblesection{Semi-discrete convergence factor from the modified equation technique}
\invisiblesubsection{Derivation of the modified convergence factor}
\includepdf[pages=4-5,pagecommand={}, scale=0.95, offset=0 -10]{paperApproximatedDiscreteSchwarz.pdf}
\invisiblesubsection{Frequency range of validity for the modified equation technique}
\includepdf[pages=6,pagecommand={}, scale=0.95, offset=0 -10]{paperApproximatedDiscreteSchwarz.pdf}
\invisiblesection{Combining semi-discrete analyses}
\includepdf[pages=7-8,pagecommand={}, scale=0.95, offset=0 -10]{paperApproximatedDiscreteSchwarz.pdf}
\invisiblesection{Effect on the optimisation on free parameters}
\invisiblesubsection{One-parameter optimisation}
\includepdf[pages=9-10,pagecommand={}, scale=0.95, offset=0 -10]{paperApproximatedDiscreteSchwarz.pdf}
\invisiblesubsection{Robin two-sided optimisation}
\invisiblesubsection{Robustness}
\includepdf[pages=11,pagecommand={}, scale=0.95, offset=0 -10]{paperApproximatedDiscreteSchwarz.pdf}
\invisiblesection{Conclusion}
\includepdf[pages=12-,pagecommand={}, scale=0.95, offset=0 -10]{paperApproximatedDiscreteSchwarz.pdf}
\setlength{\footskip}{30pt}
% \section{Limitations of the discrete convergence factor}
% \subsection{Bounded domains in space}
% \subsection{The effect of the size of the time window}
% \subsection{Random initialisation: the effect of the white noise choice}
% \section{Partial conclusion}

