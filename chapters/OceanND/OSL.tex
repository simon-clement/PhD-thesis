\section{Oceanic surface layer}
\label{sec:OceanND_OSL}
This section focuses on the discretization of an ocean
column and its surface layer.
Indeed, we consider from now on that the bulk formulation
takes into account the surface layer of the ocean and that
this surface layer is based on
\begin{equation}
	\forall z \in [\delta_o, 0],
	\begin{cases}
	|K_u \partial_z u| &= {\frac{\rho_a}{\rho_o}}
	u_\star^2\\
	K_\theta \partial_z \theta &=
	{\frac{\rho_{\rm a} c_{\rm a}^p}
			{\rho_{\rm o} c_{\rm o}^p}}
	\theta_\star u_\star- \frac{Q_{lw} + Q_{sw}(z)}
		{\rho_{\rm o} c_{\rm o}^p}
	\end{cases}
\end{equation}
where $c_{\rm a}^p$, $c_{\rm o}^p$ are the heat capacities
of air and water and
$Q_{lw}, Q_{sw}$ are radiative fluxes defined in Section
\ref{sec:ND_Ocean_radiativeFluxes}.
\par
As it was discussed in Chapter \ref{ch:ND}, it can be important
to take into account the hypotheses of the surface layer within
the discretization. The first part of this section is dedicated
to the extention of the surface flux scheme of Chapter \ref{ch:ND}
for the oceanic surface layer.
\par
The second part of this section focuses on a specificity of the
oceanic surface layer: the radiative fluxes. The latter penetrate
in the first few meters of the ocean, creating an external forcing
which affects the surface layer. In particular, the flux
$\langle w' \theta'\rangle = -K_\theta \partial_z \theta$ is no more
constant along the vertical when considering the radiative fluxes.
\par
As a first step towards a discretization including the radiative
fluxes, we will derive in Section \ref{sec:ND_Ocean_radiativeFluxes}
a discretization based on an approximative reconstruction of the
potential temperature.
\par
% 
% If a \textit{two-sided} bulk is used then the
% discretization of the oceanic surface layer can be derived
% similarly to the discretization of the atmospheric surface
% layer in Chapter \ref{ch:ND}.
% \par
% In the ocean modeling community, it is unusual to use
% wall modeling at the surface. The Sea Surface Temperature
% (SST) is often taken at the first grid level inside the
% ocean: this corresponds to the assumption that 
% the profiles are constant in the surface layer.
% \cite{zeng_prognostic_2005} proposed a scheme to avoid this
% issue for weather forecasting, climate modeling and
% data assimilation.
% %
% \par
% The goal of this section is to investigate the effects
% of the surface layer discretization in a coupled situation.
We first (\S \ref{sec:ND_Ocean_differencesWithAtmosphere})
specify the differences between the ocean and
atmosphere models in use,
then focus on the radiative fluxes in
\S \ref{sec:ND_Ocean_radiativeFluxes}.
Finally, Section \ref{sec:ND_Ocean_bulkRadiativeFluxes}
presents the sensitivity to the discretization obtained of the
surface layer in a forced setting and in a coupled setting.
\subsection{Differences with the atmosphere and derivation of
	a symmetric surface flux scheme}
\label{sec:ND_Ocean_differencesWithAtmosphere}
In this section, the numerical model of a stratified column
of ocean is described. The objective is to obtain a model
similar to the atmosphere while including the specificity
of the ocean.
In particular, the main changes compared to the atmosphere
model are:
\begin{itemize}
	\item the density which mainly affects the exchanges
		between ocean and atmosphere;
	\item the universal stability functions which are
		taken in \cite{large_similarity_2019};
	\item the vertical coordinate $z$ is negative;
	\item the time and space scales (the motion in the ocean
		is slower and smaller space steps are used with
		larger time steps).
\end{itemize}
\subsubsection{The Ocean model}
We describe here the continuous model in use for the ocean.
The equations for momentum, potential temperature and
turbulent kinetic energy in the inner domains are similar
to those of the atmosphere
except that the geostrophic momentum is not included in the ocean:
\begin{equation}
	\begin{aligned}
	&(\partial_t + if) u - \partial_z (K_u \partial_z u) = 0
		,~~~~~ z \leq \delta_{o} \\
	&\partial_t \theta -\partial_z (\partial_z K_{\theta} \theta)
	= F_{\theta},~~~~~~~~~~~~~~ z \leq \delta_{o} \\
		&\partial_t e =
    \underbrace{\partial_z \left(K_e
    \partial_z e\right)}_{\text{diffusion}}
    + \underbrace{K_u ||\partial_z u||^2}_{\text{shear}} 
    - \underbrace{K_{\theta} N^2 }_{\text{buoyancy}}
    - \underbrace{c_{\epsilon}
    \frac{e^{3/2}}{l_{\epsilon}(z)}}_{\text{dissipation}}
	\end{aligned}
\end{equation}
$F_\theta = -\frac{\partial_z Q_{sw}}{\rho_{\rm o} C_p}$ is a
forcing corresponding to the penetration of a shortwave radiadive
flux coming from the sun in a diurnal cycle.
As in the atmospheric case the buoyancy $N^2$ is
given by a linear equation of state
$N^2 = g \alpha \partial_z \theta$.
The mixing lengths follow the description in 
\ref{sec:ND_StratifiedCase_turbulentVisc}
except that $l_{up}$ and $l_{down}$ are swapped and the shear
is neglected in \eqref{eq:ND_StratifiedCase_lD80}.
%
\paragraph{Initialization and boundary conditions.}
At initialization, the TKE is set to $e=e_{\rm min}$, 
the temperature is set to a constant $\theta = 280 \; {\rm K}$
and the initial momentum is set to $u = 0 \; {\rm m}.{\rm s}^{-1}$.
At the bottom boundary,
$\partial_z u = 0$,
$\partial_z \theta = 0$ and $e=e_{\rm min}$.

\subsubsection{The surface boundary condition}
We derive here the discretization ``FV free" applied to the oceanic
column model. Let $k$ be the space index such that
$z_{k-1} < \delta_o \leq z_k$;
in this chapter, we will assume that $\delta_o$ is
inside the first cell under the surface as it was done
in the beginning of Chapter \ref{ch:ND}. The index k hence
corresponds to the surface index $z_k=0$.
we note $\widetilde{u}, \widetilde{\theta}$
the averaged variables over the interval $(z_{k-1}, \delta_o)$.
In the case of the atmosphere column, equation
\eqref{eq:ND_StratifiedCase_relation_tilde_bar}
gives the relation between the averaged variables
$\overline{u}, \overline{\theta}$ over the volume $(z_{k-1}, z_{k})$.
The same relation can be written here:
\begin{equation}
\begin{aligned}
\label{eq:ND_Ocean_relation_tilde_bar}
\alpha_{o, u}\widetilde{u} = \overline{u}_{k-1/2} -
\widetilde{h}
	\left(\frac{\phi_{\delta}}{3} + \frac{\phi_{k-1}}{6}\right)
	\left(\alpha_{o, u} - \frac{\widetilde{h}}{h_{k-1/2}}\right)
	- (1 - \alpha_{o, u})u(0)\\
\alpha_{o, \theta}
\widetilde{\theta}
= \overline{\theta}_{k-1/2} -
	\widetilde{h}\left(\frac{{(\partial_z \theta)}_{\delta}}{3}
	+ \frac{{(\partial_z \theta)}_{k-1}}{6}\right)
	\left(\alpha_{o, \theta}-\frac{\widetilde{h}}{h_{k-1/2}}\right)
 - (1 - \alpha_{o, \theta})\theta_s
\end{aligned}
\end{equation}
with for $x = u, \theta$:
\begin{equation}
	\alpha_{o, x} = \frac{\widetilde{h}}{h_{k-1/2}} +
	\frac{\frac{1}{{h_{k-1/2}}}\int_{\delta_{o}}^{z_k} x(0) - x(z)
	dz}{x(0) - x(\delta_{o})}
\end{equation}
Appendix \ref{sec:ND_Ocean_stabilityFunctionIntegration}
details the derivation of the explicit formula of $\alpha_{o}$
with the oceanic universal stability functions.
Finally, the scheme at the first grid level above
the surface layer is:
\begin{equation}
	\label{eq:ND_Ocean_semiDiscreteEkmanEqFVfree}
	(\partial_t+if) \widetilde{u}
	= \frac{K_{u, \delta_o}\phi_{\delta_o}
	- K_{u,k-1} \phi_{k-1}}{\widetilde{h}}
\end{equation}
\begin{equation}
	\label{eq:ND_Ocean_semiDiscreteEkmanEqPTFVfree}
	\partial_t \widetilde{\theta}
	= \frac{K_{\theta, \delta_o}{(\partial_z \theta)}_{\delta_o} -
	K_{\theta,k-1} {(\partial_z \theta)}_{k-1}}{\widetilde{h}}
	+ \widetilde{F}_\theta 
\end{equation}
where for $x=u,\theta$:
\begin{equation}
	\label{eq:ND_Ocean_semiDiscreteEkmanEqPTFVfree}
	\widetilde{x} = \frac{1}{\alpha_{\rm o, x}(t)}
	\left(
	\overline{x}_{k-1/2} -
	\widetilde{h}(\frac{{(\partial_z x)}_{\delta_o}}{3} +
	\frac{{(\partial_z x)}_{k-1}}{6})
	\left(\alpha_{o, x}-\frac{\widetilde{h}}{h_{k-1/2}}\right)
	 - (1 - \alpha_{o, x})x(0)
	\right)
\end{equation}
In the surface layer, the hypothesis that the potential
temperature is quasi-stationary gives that
$\forall z \in [\delta_o, 0], ~~
\partial_z \left(K_\theta \partial_z \theta \right)
= \frac{\partial_z Q_{sw}}{\rho_{\rm o} C_p}$.
The scheme uses as boundary conditions at the surface layer:
\begin{equation}
\label{eq:ND_Ocean_boundaryConditionFVfree}
\begin{aligned}
	K_{u,\delta_o} \phi_{\delta_o} &= \frac{\rho_{\rm a}}
	{\rho_{\rm o}}
	u_\star^2 e_\tau
	\\
	K_{\theta, \delta_o} (\partial_z \theta)_{\delta_o} &= 
	\frac{\rho_{\rm a} c_{\rm a}^p}{\rho_{\rm o} c_{\rm o}^p}
	\theta_\star u_\star - \frac{Q_{lw} + Q_{sw}(\delta_{o})}{\rho_{\rm o} c_{\rm o}^p}
  \end{aligned}
\end{equation}
The surface flux scheme is summarized in Figure
\ref{fig:ND_Ocean_nouvelle_dis_neutre}.
\begin{figure}
	\subimport{images/}{nouvelle_dis_neutre_ocean.pdf_tex}
	\caption{Surface layer scheme ``FV free" in the ocean model}
	\label{fig:ND_Ocean_nouvelle_dis_neutre}
\end{figure}
As it will be explained
in Section \ref{sec:ND_Ocean_radiativeFluxes} this surface flux scheme
would require an integration of the radiative flux and we will instead
use an evolution equation for $\theta$ together with an approximated
reconstruction.
\subsection{Radiative fluxes, another surface flux scheme}
\label{sec:ND_Ocean_radiativeFluxes}
We note $Q_{sw}$ and $Q_{lw}$ the shortwave and longwave (positive
downward) radiative fluxes.
To include those fluxes in the
bulk formula, \cite{pelletier_two-sided_2021} introduced
a variable $\theta_\star^{\rm rad}$
that is similar to a friction scale but depends on $z$:
\begin{equation}
\theta_\star^{\rm rad}(z) =
	\theta_\star -
	\frac{Q_{sw}(z) + Q_{lw}}{u_{\star}\rho_{\rm a} c_{\rm a}^p}
\end{equation}
This variable is used instead of $\theta_\star$
in the ocean part of the two-sided bulk procedure
to include the radiative fluxes.
To be fully consistent between the computational domain and
the oceanic surface layer, the boundary condition for the temperature
should take into account the radiative fluxes: in
\eqref{eq:ND_Ocean_boundaryConditionFVfree} the flux
$K_\theta \partial_z \theta$ is exactly equal to
$\frac{\rho_{\rm a} c_{\rm a}^p}{\rho_{\rm o} c_{\rm o}^p}
\theta_\star^{\rm rad}(\delta_o) u_\star$.
\par
The reconstruction of $\theta$
requires an vertical integration of a combination of $Q_{sw}$
and the stability functions of MOST. The integral involves
a special functions (the exponential integral)
and it would need to be integrated a second time
for the Finite Volume representation. These difficulties
are left for future work.
\par
Instead, as an intermediate step
to avoid the double integration of $Q_{sw}$
in the Finite Volume discretization,
we use an evolution equation
inside the surface layer despite the contradiction with
the quasi-stationarity. This intermediate discretization
features a different reconstruction than outside the
surface layer:
it cannot be fully coherent with the surface layer
hypotheses because of the quasi-stationarity
but it might be more adapted than the
quadratic spline reconstruction used outside the surface layer.
% \begin{equation}
% 	\label{eq:ND_Ocean_skinbulk}
% 	\frac{\kappa (\theta(0) - \theta(\delta_o))}
% 	{\theta_o^{\star}} = (1 -
% 	\frac{Q_{lw}}{Q_H})
% 	\left(\ln (1 - \frac{\delta_o}{z_{\theta}}) -
% 	\psi_h(\frac{-\delta_o}{L_o})\right)
% 	- \frac{\lambda_\theta Q_{sw}(0)}
% 	{Q_H} E(\delta_o)
% \end{equation}
\par
%\subsubsection{Adding radiative fluxes: derivation without
%molecular sub-layer}
We neglect the molecular sub-layer and
integrate between $\delta_o$ and 0:
the flux at $z=0$ is
$K_\theta \partial_z \theta = \frac{Q_H - Q_{lw}}{\rho_{\rm o} c_{\rm o}^p}$
where $Q_H = \theta_\star u_\star \rho_{\rm a} c_{\rm a}^p$
\begin{equation}
\label{eq:evolEqOSL}
h_{osl}\partial_t \overline{\theta}_{osl} =
	\frac{Q_H - Q_{lw}}{\rho_{\rm o} c_{\rm o}^p}
- \left. K_\theta \partial_z \theta 
\right|_{\delta_{o}}
- h_{osl}\int_{\delta_o}^0
	\frac{\partial_z Q_{sw}}{\rho_{\rm o} c_{\rm o}^p} dz
\end{equation}
where $h_{osl} = |\delta_o|$
is the size of the surface layer and
$\overline{\theta}_{osl}$ is the average potential
temperature in the surface layer.
\subsubsection*{Approximate reconstruction of $\theta(z)$}
We will use the following reconstruction of $\theta$
inside the SL (taken from \citep{zeng_prognostic_2005}
where the molecular surface layer was neglected to simplify
the expressions):
\begin{equation}
	\label{eq:ND_Ocean_reconstructionSimpleTheta}
    \theta(z) = \theta_s -
    \left(\frac{z}{\delta_{o}}\right)^\nu \left( \theta_s - 
    \theta_{\delta_o}\right)
\end{equation}
where $\nu$ is a constant parameter that can be set to 1 to
recover a linear reconstruction of $\theta(z)$.
\cite{zeng_prognostic_2005} choose $\nu=0.3$ and argue that
the choice of $\nu$ is linked to the size of the molecular
sub-layer.
We formulate the reconstruction in terms of $\overline{\theta}_{osl}$
and $\left.\partial_z \theta\right|_{\delta_o}$:
\begin{equation}
    \overline{\theta}_{osl} = \frac{1}{\nu+1}
    (\theta_{\delta_o} + \nu \theta_s)
	, ~~~ \left.\partial_z \theta \right|_{\delta_o}
= \frac{\nu}{h_{osl}} (\theta_s - \theta_{\delta_o})
\end{equation}
We get that the surface temperature
is $\theta_s = \overline{\theta}_{osl}
+ \frac{h_{osl}}{\nu(\nu+1)} \partial_z \theta$ and
the difference of temperature between
the surface and the bottom of the surface layer is
$\theta_s - \theta_{\delta_o} = 
\frac{h_{osl}}{\nu}\partial_z \theta$:
\begin{equation}
    \theta(z) = \overline{\theta}_{osl} +
    \frac{h_{osl}}{\nu} \partial_z \theta_{\delta_o} \left(
    \frac{1}{\nu+1} - \left(\frac{z}{\delta_{o}}\right)^\nu
\right)
\end{equation}
\begin{remark}
It would be possible to consider that $\overline{\theta}_{osl}$ is
the average in an interval $(\delta_o, z_k)$ with $z_k \neq 0$.
However the radiative forcing would affect each individual cell
inside the surface layer, leading to a contradiction with the simple
reconstruction \eqref{eq:ND_Ocean_reconstructionSimpleTheta}
of the surface layer.
We instead assume here that $|\delta_{o}| < |z_{k-1}|$ and $z_k=0$.
\end{remark}
\subsubsection*{Link with the quadratic spline}
Let us assume now that  $\theta(z)$ is a quadratic spline between
$z_{-1}$ and $\delta_o$.
Using the continuity at $z=\delta_o$ and Chasles' relation,
we link the surface layer profile with the spline:
\begin{equation}
\underbrace{
    \overline{\theta}_{osl} -
    \frac{h_{osl}}{\nu+1}\left. \partial_z \theta
    \right|_{\delta_o}}_{\theta(\delta_o^{+})}
    =
    \underbrace{
    \widetilde{\theta}
    + \frac{\widetilde{h}}{3}
    \left.\partial_z \theta\right|_{\delta_o}
    + \frac{\widetilde{h}}{6}
	\left.\partial_z \theta\right|_{z_{-1}}}_{
	\theta(\delta_o^{-})}
	,
    ~~~~
	\underbrace{
    h_{-1/2} \overline{\theta} = 
    h_{osl}\overline{\theta}_{osl} +\widetilde{h}
    \widetilde{\theta}}_{
    \int_{z_{-1}}^{0}\theta(z) dz
    }
\end{equation}
The reconstruction for $|z|<|\delta_o|$ (inside the surface layer) is hence
\begin{equation}
    \theta(z) =
    \overline{\theta}_{-\frac{1}{2}}
    +
	\left(\frac{h_{osl}\widetilde{h}}{-z_{-1}(\nu+1)} +
	\frac{\widetilde{h}^2}{-3 z_{-1}}
    +
    \frac{h_{osl}}{\nu}\left(
    \frac{1}{\nu+1} + \left(\frac{z}{h_{osl}}
    \right)^\nu
    \right)\right) \partial_z \theta_{\delta_o}
    +
	\left(\frac{\widetilde{h}^2}{-6 z_{-1}}\right)
    \partial_z \theta_{-1}
\end{equation}
for $|\delta_o| < |z| < |z_{-1}|$
(in the quadratic region) we have
\begin{equation}
    \theta(z) =
        \widetilde{\theta}
        +
        \left(
        z-\delta_o + 
        \frac{(z-\delta_o)^2}{2\widetilde{h}}
        + \frac{\widetilde{h}}{3}
        \right)\partial_z \theta_{\delta_o}
        +
        \left(
        \frac{\widetilde{h}^2 - 
        3(z-\delta_o)^2}{6 \widetilde{h}}
        \right)\partial_z \theta_{-1}
\end{equation}
where $\widetilde{\theta}$ is computed with
\begin{equation}
\label{eq:formulaTildeTheta}
\widetilde{\theta} = \overline{\theta}_{-1/2}
	-\frac{\delta_o}{z_{-1}}\left(
\frac{h_{osl}}{\nu+1} + \frac{\widetilde{h}}{3}
\right)\partial_z \theta_{\delta_o}
	- \frac{\widetilde{h}\delta_o}{6z_{-1}}
\partial_z \theta_{-1}
\end{equation}
Now that the reconstruction in all the first cell is known,
the only thing left to do is to derive the evolution equations
to integrate in time the potential temperature.
\subsubsection*{The discretization}
The continuity equation at $z_{-1}$
uses $\widetilde{h}$
as the space step:
\begin{equation}
    \widetilde{\theta}
    - \frac{\widetilde{h}}{3}
    \partial_z \theta_{-1}
    - \frac{\widetilde{h}}{6}
    \partial_z \theta_{\delta_o}
    = \overline{\theta}_{-3/2}
    + \frac{\widetilde{h}}{3}
    \partial_z \theta_{-1}
    + \frac{\widetilde{h}}{6}
    \partial_z \theta_{-2}
\end{equation}
The evolution equation of the first volume is:
\begin{equation}
h_{-\frac{1}{2}}\partial_t
\overline{\theta}_{-\frac{1}{2}} =
	\frac{Q - Q_{lw}}{\rho_{\rm o} c_{\rm o}^p}
- K_{\theta, -1} \partial_z \theta_{-1}
- h_{-\frac{1}{2}}\int_{z_{-1}}^0
	\frac{\partial_z Q_{sw}}{\rho_{\rm o} c_{\rm o}^p} dz
\end{equation}
We need a last equation to close the system.
Starting from \eqref{eq:formulaTildeTheta}, we use the
evolution equation
$\widetilde{h}\partial_t \widetilde{\theta}K_\theta \partial_z
\theta_{\delta_o}- K_\theta\partial_z \theta_{-1} - \widetilde{h}
\int^{\delta_o}_{z_{-1}}
	\frac{\partial_z Q_{sw}}{\rho_{\rm o} c_{\rm o}^p} dz$
to obtain
\begin{equation}
\partial_t
\left(
\widetilde{h}\overline{\theta}_{-1/2}
	-\frac{\delta_o}{z_{-1}}\left(
\frac{\widetilde{h}h_{osl}}{\nu+1} + \frac{\widetilde{h}^2}{3}
\right)\partial_z \theta_{\delta_o}
	- \frac{\widetilde{h}^2\delta_o}{6z_{-1}}
\partial_z \theta_{-1}
\right)
	= K_\theta\partial_z \theta_{\delta_o}-
K_\theta\partial_z \theta_{-1}
- \widetilde{h}
\int^{\delta_o}_{z_{-1}}
	\frac{\partial_z Q_{sw}}{\rho_{\rm o} c_{\rm o}^p} dz
\end{equation}
Note that the reconstruction we are using is not differentiable
in $z=0$.
If greater regularity is needed, it is possible to link
the reconstruction with a proper modeling of the molecular sub-layer
(e.g. $\rho_{\rm o} c_{\rm o}^p K_{mol} \partial_z \theta(z) =
Q - Q_{lw} - \int_z^0 \partial_z Q_{sw}(z')dz'$ for $|z|\ll|\delta_o|$)
% It is hence more appealing to take into account
% the molecular sublayer to add some regularity to the solution.
% \subsubsection{Oceanic surface flux scheme with molecular sublayer}
% We now imitate the previous derivation, using a more regular
% reconstruction for $\theta(z)$ inside the surface layer:
% we take exactly the reconstruction used by
% \cite{zeng_prognostic_2005}: let $d << |\delta_o|$ be the
% depth of the molecular sublayer:
% \begin{equation}
% 	\theta(z) = \theta_{-d} -
% 	\left(\frac{z+d}{\delta_o + d}\right)^\nu
% 	\left(\theta_{-d} - \theta_{\delta_o}\right)
% \end{equation}
% where the temperature at the bottom of the molecular sublayer
% $\theta_{-d}$ is linked to the surface temperature through an
% integration of
% $K_{mol}\partial_z \theta = \frac{Q-Q_{lw}}{\rho_{\rm o} c_{\rm o}^p} -
% \frac{1}{\rho_{\rm o} c_{\rm o}^p}\int_{-d}^0\partial_z Q_{sw}(z)dz$:
% \begin{equation}
% 	\theta_{-d} = \theta_s - \frac{d}{\rho_{\rm o} c_{\rm o}^p K_{\theta, mol}}
% 	\left(
% 	\frac{Q-Q_{lw} - Q_{sw}(0)f_s}{\rho_{\rm o} c_{\rm o}^p}
% 	\right).
% \end{equation}
% $f_s$ is here the fraction of solar radiation absorbed in the
% sublayer, described in \cite{zeng_prognostic_2005}.
% $d$ can be parametrized by $u_\star$ and the fluxes.
\subsection{Sensitivity to the discretization
	of the surface layers}
\label{sec:ND_Ocean_bulkRadiativeFluxes}
Figure \ref{fig:OceanND_OSL_Forced}
show the profiles obtained in wind and
potential temperature with several discretisations.
As in Chapter \ref{ch:ND} a high resolution simulation
where every cell is divided into three cells
is also performed.
\par
For the velocity, the differences between
		the different cases are located between
		$z=-1\;{\rm m}$ and $z=0\;{\rm m}$.
		For the temperature, significant differences
		are found down to the depth of $10$ meters.
\par
The output of the normal simulation is not extremely sensitive
to the discretization (the continuous lines are close with each
others); the Finite Difference and the Finite Volume discretizations
give coherent results.
\par
The size of the oceanic surface layer is $|\delta_o|=0.17$.
This choice actually corresponds to first grid level
of the high resolution cutted in half.
Indeed, the ``FV free" discretization does not allow
for inactive grid levels like in the atmosphere case.
\par
The distance between the dashed line and the
continuous line for the same color indicates
how consistent is the surface layer discretization.
Contrarily to the atmosphere case, the
least consistent of the discretizations appears to
be the ``FV free".
\par
Figure \ref{fig:OceanND_OSL_Coupled} is the same as Figure 
\ref{fig:OceanND_OSL_Forced} except that the ocean column is coupled
with the atmosphere column described in Chapter \ref{ch:ND}.
\begin{figure}
	\centering
\includegraphics[scale=0.6]{images/oce_Forced.pdf}
	\caption{Forced case: dashed lines indicate
	a high resolution simulation and solid lines
	show the normal simulation. Numerical parameters
	are the same as in \S \ref{sec:ND_Consistency_Coupled}.}
	\label{fig:OceanND_OSL_Forced}
\end{figure}
\begin{figure}
	\centering
\includegraphics[scale=0.6]{images/oce_Coupled.pdf}
	\caption{Coupled case. The observed convergence factor
	of the Schwarz method was around $10^{-2}$
	(i.e. convergence in 2 or 3 iterations).
	Dashed lines indicate
	a high resolution simulation and solid lines
	show the normal simulation.}
	\label{fig:OceanND_OSL_Coupled}
\end{figure}
\par
Similarly to the sensitivity to the use of a \textit{two-sided}
bulk (Figure \ref{fig:OceanND_twoSidedBulk_difference}),
the difference in temperature is bigger than the difference in
velocity. Moreover, as in Figure \ref{fig:OceanND_OSL_Forced}
the differences between the cases are in the direct
neighborhood to the surface for $u$ whereas they are also
present further from the interface for $\theta$.
\par
The difference between the \textit{one-sided} bulk ($\delta_o= 0$)
and the \textit{two-sided} bulk also appears in the potential
temperature in the two ``FV pure" cases.
\par
The biggest difference between the normal simulation and the
high-resolution simulation is again the scheme ``FV free".
A possible reason for this undesirable behavior is the
lack of accuracy of the approximation
of the profile of $\theta$ in the oceanic surface layer.
\section{Partial conclusion}
In this chapter, we extended the ``FV free" discretization
for the oceanic surface layer. The radiative fluxes inclusion
was studied and we proposed a discretization to take them
into account. They represent an instance of additional
difficulties encountered when developping coherent
numerical treatment of the surface layer.
\par
The discretization proposed in this chapter does not
perform better than the other ones when comparing
the consistencies in a coupled case and in a forced
case.
\par
No stability nor accuracy studies were conducted on the proposed
discretisations. Although they seem to behave correctly in the
experiments, further analyses of the ``FV free" scheme are
necessary before they can be used. In particular,
the Finite Volume approximation between the first cell and the second
corresponds to the use of a strongly unstructured grid,
the second cell being typically twofold bigger than
the explicit part of the first cell.
