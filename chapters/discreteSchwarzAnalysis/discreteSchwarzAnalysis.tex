\chapter{Discrete Analysis of Schwarz methods for a diffusion reaction problem with discontinuous coefficients}
\label{ch:discreteSchwarzAnalysis}
\minitoc
In this chapter the focus is on the effect of the discretization
on the convergence speed of Schwarz methods. The reaction-diffusion
equations coupling problem presented in
Chapter \ref{ch:airseaSCM} is used : the goal is
to present the methodology
that can be used to study the convergence
of Schwarz methods in a discrete setting.
\par
For this purpose, the article
\citep{clement_discrete_2022-1} was published in
\textit{SMAI Journal of Computational Mathematics} and is reported
here.
This article:
\begin{itemize}
	\item presents a Finite Volume discretization based
	on quadratic spline reconstruction in space;
	\item presents a methodology to study the convergence
	speed of Schwarz methods;
	\item examinates the difference betweeen
	the discrete convergence factor and the continuous one;
	\item proves the stability of the discrete schemes considered;
	\item exposes an operator $\gamma$ showing the specificity
	of multi-stage time schemes within Schwarz methods.
	\item investigates on the interactions between the time
		and space discretisations.
\end{itemize}
\paragraph{Code availability}
The code used in the numerical experiments is available in a Zenodo
archive (https://doi.org/10.5281/zenodo.7092357,
\citep{clement_code_2022-1}).
\setlength{\footskip}{110pt}
\invisiblesection{Introduction}
\includepdf[pages=1,pagecommand={}, scale=0.95, offset=0 -10]{papierDiscreteSchwarz.pdf}
%\includepdf{papierDiscreteSchwarz.pdf}
	%\value{thepagenumberSMAI2022}}
\invisiblesection{Model problem and Schwarz Waveform
relaxation algorithm}
\invisiblesubsection{Model problem}
\invisiblesubsection{Schwarz Waveform
relaxation algorithm}
\includepdf[pages=2,pagecommand={}, scale=0.95, offset=0 -10]{papierDiscreteSchwarz.pdf}
\invisiblesubsection{General form of the continuous convergence rate}
\includepdf[pages=3,pagecommand={}, scale=0.95, offset=0 -10]{papierDiscreteSchwarz.pdf}
\invisiblesection{Semi-discrete and discrete convergence rates}
\invisiblesubsection{Time discretisation}
\includepdf[pages=4-7,pagecommand={}, scale=0.95, offset=0 -10]{papierDiscreteSchwarz.pdf}
\invisiblesubsection{Space discretisation}
\includepdf[pages=8-15,pagecommand={}, scale=0.95, offset=0 -10]{papierDiscreteSchwarz.pdf}
\invisiblesection{Discrete case}
\invisiblesubsection{Stability analysis}
\includepdf[pages=16-17,pagecommand={}, scale=0.95, offset=0 -10]{papierDiscreteSchwarz.pdf}
\invisiblesubsection{Convergence rates}
\includepdf[pages=18-19,pagecommand={}, scale=0.95, offset=0 -10]{papierDiscreteSchwarz.pdf}
\invisiblesection{Numerical examples and optimisation of convergence rates}
\invisiblesubsection{Comparison between numerical and theoretical convergence rates}
\includepdf[pages=20,pagecommand={}, scale=0.95, offset=0 -10]{papierDiscreteSchwarz.pdf}
\invisiblesubsection{Optimisation of the two-sided Robin interface conditions}
\includepdf[pages=21,pagecommand={}, scale=0.95, offset=0 -10]{papierDiscreteSchwarz.pdf}
\invisiblesection{Conclusion}
\includepdf[pages=22-,pagecommand={}, scale=0.95, offset=0 -10]{papierDiscreteSchwarz.pdf}
\setlength{\footskip}{30pt}

% \section{On the difficulty of an analytical optimisation}
% \subsection{Optimisation of the continuous case}
% \subsection{Optimisation of the semi-discrete overlapping case}
% \subsection{No semi-discrete optimisation without overlap}
% \section{On the monolithic solution}
% \subsection{Necessary precision when projecting the interface conditions}
% \subsection{Difference with the monolithic solution}
