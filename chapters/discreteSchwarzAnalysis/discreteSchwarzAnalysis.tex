\chapter{Discrete Analysis of Schwarz methods for a diffusion reaction problem with discontinuous coefficients}
\label{ch:discreteSchwarzAnalysis}
\minitoc
As seen in the previous chapter (\S \ref{sec:airseaSCM_SWR})
the convergence of Schwarz methods is usually studied on the
continuous problem. For the inclusion of the surface layer
in the convergence study
it seems important to switch to the semi-discrete level or even
to the totally discrete level.
In this chapter the focus is on the effect of the discretization
on the convergence speed of Schwarz methods. The reaction-diffusion
equations coupling problem presented in
Chapter \ref{ch:airseaSCM} is used : the goal is
to present the methodology that can be used to study the convergence
of Schwarz methods in a discrete setting.
\par
For this purpose, the article
\citep{clement_discrete_2022-1} was published in
\textit{SMAI Journal of Computational Mathematics} and is reported
here.
This article is built upon the following steps:
\begin{itemize}
	\item Description of a methodology to study the convergence
	speed of Schwarz methods.
	\item Introduction of space and time schemes,
		notably a Finite Volume discretization based
	on quadratic spline reconstruction in space and
		a Diagonally Implicit Runge Kutta scheme in time.
	\item Numerical experiments and optimization of continuous
		and discrete convergence factors.
\end{itemize}
The key points of this chapter are:
\begin{itemize}
	\item The difference between
	the discrete convergence factor and the continuous one
		and the expression of those convergence factors.
	\item The stability of the discrete schemes considered;
	\item An operator showing the specificity
	of multi-stage time schemes within Schwarz methods.
	\item The interactions between the time and space
		discretizations.
	\item The numerical optimization of the convergence factors.
\end{itemize}
\paragraph{Code availability}
All the theoretical results were numerically validated:
the code used for the experiments in this chapter
and all the others is available in a Zenodo
archive (https://doi.org/10.5281/zenodo.7092357,
\citep{clement_code_2022-1}).
\setlength{\footskip}{110pt}
\invisiblesection{Introduction}
\includepdf[pages=1,pagecommand={}, scale=0.95, offset=0 -10]{papierDiscreteSchwarz.pdf}
%\includepdf{papierDiscreteSchwarz.pdf}
	%\value{thepagenumberSMAI2022}}
\invisiblesection{Model problem and Schwarz Waveform
relaxation algorithm}
\invisiblesubsection{Model problem}
\invisiblesubsection{Schwarz Waveform
relaxation algorithm}
\includepdf[pages=2,pagecommand={}, scale=0.95, offset=0 -10]{papierDiscreteSchwarz.pdf}
\invisiblesubsection{General form of the continuous convergence rate}
\includepdf[pages=3,pagecommand={}, scale=0.95, offset=0 -10]{papierDiscreteSchwarz.pdf}
\invisiblesection{Semi-discrete and discrete convergence rates}
\invisiblesubsection{Time discretisation}
\includepdf[pages=4-7,pagecommand={}, scale=0.95, offset=0 -10]{papierDiscreteSchwarz.pdf}
\invisiblesubsection{Space discretisation}
\includepdf[pages=8-15,pagecommand={}, scale=0.95, offset=0 -10]{papierDiscreteSchwarz.pdf}
\invisiblesection{Discrete case}
\invisiblesubsection{Stability analysis}
\includepdf[pages=16-17,pagecommand={}, scale=0.95, offset=0 -10]{papierDiscreteSchwarz.pdf}
\invisiblesubsection{Convergence rates}
\includepdf[pages=18-19,pagecommand={}, scale=0.95, offset=0 -10]{papierDiscreteSchwarz.pdf}
\invisiblesection{Numerical examples and optimisation of convergence rates}
\invisiblesubsection{Comparison between numerical and theoretical convergence rates}
\includepdf[pages=20,pagecommand={}, scale=0.95, offset=0 -10]{papierDiscreteSchwarz.pdf}
\invisiblesubsection{Optimisation of the two-sided Robin interface conditions}
\includepdf[pages=21,pagecommand={}, scale=0.95, offset=0 -10]{papierDiscreteSchwarz.pdf}
\invisiblesection{Conclusion}
\includepdf[pages=22-,pagecommand={}, scale=0.95, offset=0 -10]{papierDiscreteSchwarz.pdf}
\setlength{\footskip}{30pt}

% \section{On the difficulty of an analytical optimisation}
% \subsection{Optimisation of the continuous case}
% \subsection{Optimisation of the semi-discrete overlapping case}
% \subsection{No semi-discrete optimisation without overlap}
% \section{On the monolithic solution}
% \subsection{Necessary precision when projecting the interface conditions}
% \subsection{Difference with the monolithic solution}
