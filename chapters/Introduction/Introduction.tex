\chapter*{Introduction}
\label{ch:introduction}
\addcontentsline{toc}{chapter}{\nameref{ch:introduction}}
% Qu'est-ce qu'un modèle numérique
Numerical models are ubiquitous in oceanography, climatology and
meteorology. They become more and more accurate as the
computational power increases and as we refine our knowledge on
both physical phenomena and numerical behaviour.
As long as we want to encompass more phenomena in the simulations,
both mathematical models and their implementations must be
adapted to new scales and new challenges.
\par % Le couplage océan-atmosphère
The interactions between the ocean and the atmosphere are crucial
for climate projections and mid- to long-term weather forecasts.
It means that the ocean and atmosphere must be jointly simulated;
however, there is no ocean-atmosphere model in a single
block to our knowledge.
Indeed, the scales and dynamics involved in those two systems are
sufficiently different to justify the use of separate models,
and there are historically two separate communities behind the
models. For those reasons simulations of the air-sea system
always rely on the \textit{coupling} between numerical models describing
the oceanic and atmospheric circulations.
\par % Qu'est-ce qu'une discrétisation
A numerical implementation does not directly solve the mathematical
model (the \textit{continuous} equations) but an approximation of it
(the \textit{discrete} equations). There exist many of these
approximations (called \textit{discretizations}), the choice of which
is not to be taken lightly: a discretization introduces some
numerical error, and may also enforce desirable mathematical
or physical properties.
\par
A numerical simulation cannot explicitly represent all the
scales involved in geophysical phenomena, because of the
colossal cost it would induce. In climate simulations,
scales smaller than an hour in time
and dozens of kilometers in space are only represented
through their effects on larger scales.
Among the parameterizations of small scales, the \textit{turbulence}
accounts for sub-grid chaotic motions.
\par % La couche limite de surface
Moreover, the interface between the models in the case of
the ocean-atmosphere coupling contains
specific turbulent motions. The discretization of this zone
called \textit{surface layer} is a key for a proper coupling.
\par % Les méthodes de Schwarz
Because of cost constraints, the ocean-atmosphere coupling algorithms
are in practice quite simple and many of them actually
correspond to a single step of a \textit{Schwarz method}
\citep{lemarie_analysis_2015}. The latter domain decomposition methods consist in solving
iteratively the models (here, the ocean model and the
atmosphere model) until the solutions at the interface match.
These coupling methods, widely used for number of applications
outside the ocean-atmosphere context, are known for being relatively
slow. However, the mismatch at the interface is multiplied
at each iteration by a \textit{convergence factor} which
can be optimized to accelerate the convergence of the
Schwarz method.
\par % Selon Marti, ça merde un peu des fois
\citep{marti_schwarz_2021} used several Schwarz iterations
instead of a single one to evaluate the loss of precision
generated by the coupling within the operational climate model
IPSL-CM. A significant difference was found
\textit{``after sunrise
and before sunset, when the external forcing (insolation at
the top of the atmosphere) has the fastest pace of change}"
and recent additional ensemble simulations tend to highlight
a significant long-term impact of performing iterations.
Although it would be unaffordable to use multiple
iterations of Schwarz methods in actual climate simulations,
this approach is useful to evaluate current methods
used in this context.
Studying the convergence properties of Schwarz methods is
hence of interest, first for the evaluation of present simulations,
and then for the reduction of the
error in the first iteration\footnote{However, note that
in the overwhelming majority of convergence studies
(including this thesis) the convergence factor of
Schwarz methods does not characterize the very first iteration
because the initial value is not the solution of the
considered differential equation.}.
\par % Optimisation discrète
The discretizations interact with the coupling methods:
if the study of the convergence factor is pursued at the
\textit{continuous} level, the coupling of the \textit{discrete}
equations may converge differently than expected. Moreover,
if the convergence is optimized at the discrete level, one expects
to obtain a faster numerical convergence.
Besides, the continuous equations describing the surface
layer are actually linked with the numerical implementation.
It is hence appropriate to directly study at the \textit{discrete}
level the coupling methods when taking into account
the surface layer.
\par % objectifs
In this context, the objectives of this thesis are:
\begin{itemize}
\item Improve our knowledge on how the discretization affects
the convergence factor of Schwarz methods.
\item Discuss the numerical treatment of the surface layer and propose
	improvements.
\item Study the effect of the surface layer within
	the ocean-atmosphere coupling.
\end{itemize}
\subsubsection*{The surface layer}
The surface layer contains strong disparities and can be modeled by a so-called
\textit{law of the wall}. A review of wall modelling was done by
\citep{larsson_large_2016}, focusing on Large Eddy Simulations
(which are simulations for small scales).
In particular, it is emphasized that the height of the surface layer
should be chosen based on both physical and numerical criteria.
\par
Recent advances in \textit{bulk formulations}
(which characterize the law of the wall in the atmosphere models)
partially underpin this thesis:
one is the introduction by \citep{pelletier_two-sided_2021}
of an oceanic part of the surface layer in the \textit{bulk formulations}.
The other is the discovery of a systematic bias
created in some discretizations because the variables are
``used as the center point value in the model grid
to estimate the surface fluxes even though
the variables are volume-averaged values"
\citep{nishizawa_surface_2018}.
In the latter case, the bias can be avoided by
adapting the bulk formulations to averaged
data.
\par
Moreover, the hypotheses on which the bulk formulations rely
are generally not included in the discretization.
We propose to reunite the hypotheses in the bulk formulations
and in the discretization in order to
approximate more rigorously the continuous equations.
\par
From a more theoretical point of view,
the modeling of the surface layer leads to
non-linearities in the surface fluxes. This raises
some questions about the well-posedness of the mathematical
problem.
The existence and regularity of solutions for
the ocean-atmosphere coupling is proved
in \citep{lions_mathematical_1995} with linear conditions at
the interface and a part of the non-linearities
inside the computational domains.
More recently, a study focused only
on the ocean domain \citep{chacon-rebollo_existence_2014}
shows the existence and unicity of a solution in the neighborhood
of a steady state. It is proven on a one-dimensional model that
includes a parameterization of the turbulence.
We will discuss the well-posedness of the ocean-atmosphere coupling
in the presence of a parameterized surface layer.
%
\subsubsection*{Schwarz methods}
The choice of the coupling methods in the ocean-atmosphere context is
not straightforward.
In addition to the trade-off between the required computational time and
the achieved precision, the coupling methods
also face some constraints:
\begin{itemize}
	\item some quantities must be conserved by the method
		(e.g. water, energy);
	\item the atmosphere and ocean models being very
		sophisticated, the methods should not be intrusive
		(i.e. they should consider
		the models as black boxes);
	\item it is generally unaffordable to call multiple times
		the ocean and atmosphere models.
\end{itemize}
With wisely chosen interface conditions, Schwarz methods can
satisfy those constraints. In the context of making a very
small number of iterations, the acceleration of the convergence
can be used to increase the precision of the coupling.
Discrete transparent boundary conditions
\citep{zisowsky_discrete_2006} allow to converge up to the
numerical precision in two iterations.
However, those boundary conditions are non-local in time.
Schwarz methods can be instead accelerated
by an optimization \citep{gander_optimized_2006} where
the boundary conditions at interface are local in time.
\par
This optimization, when carried in a semi-discrete in space
\citep{wu_semi-discrete_2014-1},
in time \citep{arnoult_discrete-time_2022}
or in a fully discrete setting
\citep{wu_optimized_2017} yields a theoretical convergence rate which is
closer to the one observed in numerical simulations and
generally leads to faster convergence.
However, even if they also consider reaction-diffusion equations,
\citep{wu_semi-discrete_2014-1} and \citep{wu_optimized_2017}
focus on electric circuits where
there are overlaps between the domains. \citep{gander_analysis_2018}
analyze the presence of an overlap but to our knowledge no previous study
stands in the discrete case without overlap (note that the
semi-discrete 2D stationary case was done by
\citep{gerardo-giorda_optimized_2005}).
Discrete Schwarz Waveform Relaxation studies with other equations
can be found in \citep{haynes_fully_2020} together with
the finite domain case.
In the continuous case, a lot of efforts have been made
to study the convergence of Schwarz methods
(e.g. \citep{thery_analysis_2021} where the viscosity
is varying and is discontinuous at interface or
\citep{haberlein_optimized_2014} who give an overview of
nonlinear systems).
\subsubsection*{Outline}
This thesis contains six chapters.
Chapter \ref{ch:airseaSCM} focuses on the derivation of the equations
driving the ocean-atmosphere coupling. It presents in particular
a hierarchy of models which is used in the other chapters:
two couplings of reaction-diffusion equations (with linear
or nonlinear conditions at interface), and a more sophisticated
model using a turbulence parameterization.
\par
%
\citep{clement_discrete_2022-1} (Chapter
\ref{ch:discreteSchwarzAnalysis}) studies the effect
of the discretisation in space and time of the linear
reaction-diffusion coupling problem without overlap.
In particular, it gives a methodology to study the convergence factor
of Schwarz methods at the discrete and semi-discrete levels
and highlights some features of these convergence factors.
It also presents a \textit{Finite Volume} discretization
used throughout all the other chapters.
\par
%
Chapter \ref{ch:approximatedDiscreteSchwarz}
introduces alternative methods to estimate the convergence
speed of the discrete problem with less computations.
The ultimate goal would be to obtain the convergence speed of
the discrete or semi-discrete optimization with a continuous study.
The linear reaction-diffusion coupling is taken as an illustration
to show how much the convergence can be accelerated with the
proposed approximations and to discuss their correctness.
\par
%
Using the sophisticated model based on a parameterization of turbulence,
Chapter \ref{ch:ND} and \ref{ch:OceanND} discuss the discretization of
the surface layer. We propose a discretization that is more coherent
with the physical theory of the surface layer.
Chapter \ref{ch:ND} introduces the ideas
and discretization for the atmosphere surface layer and
Chapter \ref{ch:OceanND} extends the
discussion to the ocean surface layer and focuses on handling
specific features of the ocean surface layer.
%
\par
Finally, Chapter \ref{ch:OASchwarz} treats the nonlinearity due
to the presence of the surface layer.
The well-posedness of the nonlinear coupling of
reaction-diffusion equations is discussed.
Nonlinearities
at interface were introduced in domain decomposition methods by
\citep{caetano_schwarz_2011} where they created interface
conditions resembling to linear ones specifically to match
with the existing studies.
We start instead with an existing nonlinear interface
and study its convergence properties.
