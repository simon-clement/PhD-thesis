\chapter*{Résumé détaillé}
Cette thèse s'intéresse à la représentation
numérique des interactions air-mer, notamment au sein
du couplage entre un modèle d'océan et un modèle d'atmosphère.
Nous étudierons donc conjointement des méthodes de couplage
et la représentation numérique de la
\textit{couche limite de surface} :
une zone cruciale pour le calcul des interactions.
%
\section*{Chapitre \ref{ch:airseaSCM}: Modélisation d'une colonne
océan-atmosphère}
Le chapitre \ref{ch:airseaSCM} présente les équations et
principes physiques utilisés
au sein de cette thèse.
\par
Tout d'abord, le système des \textit{équations primitives}
décrivant les écoulements atmosphériques et océaniques est présenté.
Une attention particulière est portée à la représentation de la
\textit{turbulence}, qui induit une séparation entre des échelles
``résolues" et des plus petites échelles dites ``turbulentes".
\par
Cette séparation d'échelle est nécessaire d'un point de vue
numérique : les échelles résolues correspondent à celles représentées
dans les grilles de calcul des schémas numériques (actuellement,
de l'ordre de centaines de mètres). Au contraire, les échelles
turbulentes \textit{sous-maille} ne peuvent pas être représentées
explicitement par les grilles de calcul et doivent être paramétrisées.
\par
Le concept de couche limite de surface
(basé sur une hypothèse de flux constants) est ensuite introduit.
Celui-ci est au centre du sujet de la thèse : il permet de décrire,
à travers un formalisme nommé ``loi du mur",
le comportement turbulent des écoulements aux abords de la surface.
\par
Dans les modèles d'atmosphère, la couche limite de surface est
représentée par les \textit{formulations bulk},
qui permettent de calculer les composantes
turbulentes des flux air-mer à partir des données autour de
l'interface.
La stratification (variation verticale de la densité) joue un rôle
important dans la couche limite de surface. Elle affecte la
``loi du mur" et est une composante essentielle des formulations
bulk.
\par
Cette thèse s'articule autour d'une hiérarchie de modèles présentés
au chapitre \ref{ch:airseaSCM}. Ceux-ci
décrivent tous l'évolution temporelle d'une colonne d'atmosphère
au-dessus d'une colonne d'océan en une seule dimension spatiale
(verticale).
\par
\begin{enumerate}
	\item Le plus bas niveau de la hiérarchie est un couplage
	de simples équations réaction-diffusion, utilisé aux chapitres
		\ref{ch:discreteSchwarzAnalysis} et
		\ref{ch:approximatedDiscreteSchwarz} pour étudier
		et optimiser des méthodes itératives de
		couplage (dites \textit{méthodes de Schwarz}).
		Ce modèle ne prend pas en compte les aspects
		de turbulence sous-maille.
	\item Le niveau intermédiaire (utilisé au chapitre
		\ref{ch:OASchwarz}) est un autre couplage
		d'équations réaction-diffusion.
		Cependant, ce couplage utilise des conditions de bord
		à l'interface décrivant une couche limite de surface.
\item Le plus haut niveau de la hiérarchie est un modèle utilisant
une paramétrisation de la turbulence et qui inclut une
stratification en température.
A l'interface, de véritables formulations bulk sont utilisées
(au lieu d'une formulation simplifiée dans le niveau intermédiaire).
Ce modèle est utilisé aux chapitres \ref{ch:ND}
et \ref{ch:OceanND} pour
s'intéresser à l'impact des discrétisations de couche limite,
notamment dans le couplage océan-atmosphère.
\end{enumerate}
Les objectifs de cette thèse correspondent aux trois niveaux de la
hiérarchie :
\begin{enumerate}
	\item étudier l'effet de la discrétisation sur la vitesse de
		convergence des méthodes de Schwarz~;
	\item étudier la vitesse de convergence des méthodes de
		Schwarz en présence d'une couche limite de surface
		(i.e. avec des conditions de transmission
		non-linéaires) ;
	\item développer une discrétisation permettant de représenter
		au mieux les couches limites de surface, en prenant
		en compte les paramétrisations au sein de ces
		couches limites.
\end{enumerate}
\section*{Chapitre \ref{ch:discreteSchwarzAnalysis} : Analyse discrète des méthodes de Schwarz avec des équations de réaction-diffusion}
Le chapitre \ref{ch:discreteSchwarzAnalysis} a été publié sous la forme de l'article
\citep{clement_discrete_2022-1} dans le journal
\textit{SMAI Journal of Computational Mathematics}.
\begin{itemize}
\item \textbf{Modèle continu et méthode de Schwarz} :
Le modèle utilisé dans les deux premiers chapitres est le
suivant:
\begin{subequations}
\begin{align}
\partial_t u_j +( r - \nu_j \partial_x^2) u_j &= f_j ~~~~~~~~
(j=o,a) &\qquad&
	(x,t) \in \widetilde{\Omega}_j \times ]0,T] \label{eq:dr1} \\
u_j(x,0) &= u_{j,0}(x)   &\qquad&  x \in \widetilde{\Omega}_j  \\
u_o(0^-,t) &=  u_a(0^+,t) &\qquad& t \in [0,T] \label{eq:interface-dir} \\
\nu_o \partial_x u_o(0^-,t) &= \nu_a \partial_x u_a(0^+,t) &\qquad& t \in [0,T] \label{eq:interface-neu} 
\end{align}
\label{eq:resume_francais_model-problem}
\end{subequations}
où $r$ et $\nu_j$ sont les coefficients de réaction et de diffusion.
Pour résoudre ce problème couplé, les domaines
$\widetilde{\Omega}_o = \mathbb{R_{-}}$ et
et $\widetilde{\Omega}_a = \mathbb{R_{+}}$ voient leurs équations
correspondantes \eqref{eq:dr1} être résolues tour à tour
dans $\widetilde{\Omega}_j \times [0,T]$
en utilisant les équations
\eqref{eq:interface-dir} et \eqref{eq:interface-neu}
en tant que conditions de bord.
Cette méthode s'appelle \textit{Relaxation d'onde de Schwarz avec
des conditions de transmission Dirichlet-Neumann}.
% Les conditions de Dirichlet et Neumann \eqref{eq:interface-dir} et
% {eq:interface-neu} peuvent être remplacées par des combinaisons
% linéaires de ces conditions afin d'obtenir des conditions de Robin.
L'étude de la convergence de cette méthode se fait en deux étapes :
\begin{itemize}
\item Les équations \eqref{eq:dr1} sont résolues dans l'espace de
	Fourier : la dimension temporelle devient un espace
	de fréquences. On obtient une formule analytique de
	la différence entre $u_j^k$
	($k$ note l'itération de Schwarz courante)
	et la solution couplée, c'est-à-dire
	de l'évolution de l'erreur en fin des itérations.
\item Les conditions de transmission à l'interface permettent
	de quantifier l'évolution de cette différence au fur et
	à mesure des itérations.
\end{itemize}
 La convergence est \textit{linéaire},
 c'est-à-dire que la différence entre $u_j^k$ et la solution couplée
 est multipliée à chaque itération par un
 \textit{facteur de convergence} ne dépendant
 pas de $k$. Dans le cas continu avec des conditions
 de transmission Dirichlet-Neumann, le facteur de convergence ne
 dépend pas non plus de la fréquence ni de $r$ et vaut
 $\rho_{DN}^{(c,c)}=\sqrt{\frac{\nu_o}{\nu_a}}$.
\item \textbf{Facteur de convergence semi-discret en temps} :
la convergence observée lors de l'implémentation d'une méthode
de Schwarz dépend des discrétisations en temps et en espace utilisées.
On utilise la transformée en Z au lieu de la transformée de Fourier
pour étudier un signal semi-discret en temps.
Si le passage du continu au discret est aisé pour les schémas
en temps qui ne comportent qu'une étape, les schémas
à plusieurs étapes demandent une attention particulière. En effet,
les conditions d'interface sont interpolées durant les étapes
intermédiaires. Cette interpolation
modifie la vitesse de convergence de la méthode de Schwarz,
particulièrement dans les hautes fréquences temporelles.
\item \textbf{Discrétisation en espace} :
la convergence des méthodes de Schwarz a été étudiée pour
deux discrétisations en espace. La première est une discrétisation
de référence utilisant des différences finies (FD)
centrées d'ordre 2 ;
la deuxième est une discrétisation en volumes finis (FV) basée sur
des splines paraboliques.
Chaque maille est ainsi caractérisée par un polynôme d'ordre 2
et l'approximation volumes finis est déduite du raccord des
polynômes entre les mailles.
Retrouver la solution numérique mono-domaine s'avère immédiat
lorsque la discrétisation FV est utilisée.
Avec la discrétisation FD, utiliser une condition de transmission
particulière dans ce but diminue
drastiquement la vitesse de convergence.
\item \textbf{Analyse discrète} :
une méthode d'analyse est présentée et appliquée pour les combinaisons
de deux discrétisations en temps et en espace. Ces combinaisons
sont démontrées stables en étudiant les valeurs propres des
matrices à inverser.
\end{itemize}
Finalement, en utilisant des conditions de transmission
contenant des degrés de liberté, la vitesse de convergence
des méthodes de Schwarz peut être accélérée en optimisant les
paramètres introduits.
Si l'optimisation se fait au niveau continu, les paramètres
choisis ne seront en général pas optimaux pour la vitesse de
convergence observée numériquement. Au contraire, en optimisant
sur le facteur de convergence discret on obtient des vitesses de
convergence supérieures dans les expériences numériques.
\par
Le facteur de convergence discret peut cependant s'avérer
contraignant à calculer, en particulier pour des discrétisations
en temps à plusieurs étapes ou pour des discrétisations en espace
d'ordre élevé.
\section*{Chapitre \ref{ch:approximatedDiscreteSchwarz} : Approximation du facteur de convergence discret des méthodes de Schwarz}
L'objet du chapitre \ref{ch:approximatedDiscreteSchwarz} est donc
de proposer de nouvelles approximations qui rendent le
facteur de convergence observé numériquement plus facile à estimer
que par une approche d'analyse discrète complète (comme au chapitre
\ref{ch:discreteSchwarzAnalysis}), tout en étant plus précis que le
facteur de convergence continu.
La première approximation étudiée est la méthode des
\textit{équations modifiées} qui introduit un terme représentant
les principaux effets de la discrétisation.
La seconde est une combinaison des analyses semi-discrètes
pour approcher l'analyse complètement discrète.
\subsection*{Équations modifiées}
La méthode des équations modifiées consiste à étudier au niveau
continu non pas l'équation différentielle originale mais celle
qui est résolue par le schéma numérique.
Ces équations modifiées sont obtenues à l'aide d'un développement
de Taylor de la discrétisation.
\subsubsection*{Équations modifiées en temps}
Lors du calcul du facteur de convergence, l'utilisation d'équations
modifiées en temps se traduit par un changement de la variable
fréquentielle $\omega$ utilisée dans la transformée de Fourier :
\begin{itemize}
	\item
La complexité de l'étude continue des équations modifiées en temps
est similaire à celle semi-discrète des schémas à une étape.
	\item
Les équations modifiées en temps simplifient l'étude de convergence
des schémas à deux étapes. Cependant, l'utilisation du développement
de Taylor cache l'opération d'interpolation des données de
transmission réalisée dans les étapes intermédiaires. Ainsi, lorsque la
différence entre les facteurs de convergence continu
et semi-discret en temps vient de cette opération d'interpolation,
les équations modifiées ne permettent pas d'approcher le facteur
de convergence semi-discret.
\end{itemize}
\subsubsection*{Équations modifiées en espace}
L'utilisation des équations modifiées sur les schémas en espace
suscite d'autres observations~:
\begin{itemize}
	\item 
	le développement de Taylor introduit des dérivées
d'ordre plus élevé que dans l'équation originale. Pour
obtenir le caractère bien posé des équations modifiées, il est donc
nécessaire d'ajouter des conditions d'interface. Celles-ci peuvent
être judicieusement choisies à partir de la discrétisation en espace
proche du bord.
	\item Il est en général insuffisant de n'étudier que
		l'effet de la discrétisation de
		l'équation aux dérivées partielles. En effet,
		les conditions d'interface et de bord sont
		elles aussi affectées par la discrétisation et
		leurs versions discrètes doivent être utilisées
		dans le calcul du facteur de convergence.
	\item Dans le cas général, l'utilisation des équations
	modifiées en espace augmente l'ordre de l'équation aux
	dérivées partielles et ne rend pas plus aisé
	le calcul du facteur de convergence par rapport
		au calcul semi-discret.
	\item Dans le cas particulier d'une équation ne faisant
	intervenir
	qu'une seule différentiation en espace de n'importe
	quel ordre :
	une astuce de calcul permet de se ramener au cas où un simple
	changement de variable fréquentielle suffit pour caractériser
		l'équation modifiée.
	L'analyse de la convergence en prenant en compte la
		discrétisation devient alors similaire à l'analyse
		continue.
\end{itemize}
%
\subsection*{Combinaison des facteurs de convergence}
Il est possible de combiner les facteurs
de convergence semi-discrets (S-D) et continu pour approcher le
facteur de convergence discret selon la formule:
\begin{equation}
	\text{DISCRET} \approx
	\text{S-D EN ESPACE} +
	\text{S-D EN TEMPS} -
	\text{CONTINU}
\end{equation}
%
\par
Des expériences numériques présentées à la fin
du chapitre \ref{ch:approximatedDiscreteSchwarz} montrent dans quels
cas et dans quelle mesure :
\begin{itemize}
	\item
ces diverses approximations sont efficaces pour approcher le facteur
de convergence discret ;
	\item
la convergence
peut être accélérée en optimisant ces approximations.
\end{itemize}
\section*{Chapitre \ref{ch:ND} : Vers une discrétisation de la couche limite atmosphérique cohérente avec la théorie physique}
Dans le chapitre \ref{ch:ND}, nous étudions une colonne atmosphérique et sa discrétisation.
À cause de considérations numériques, la colonne
d'atmosphère doit se diviser en deux parties :
\begin{enumerate}
	\item
	la couche limite de surface, exclue du domaine
	de calcul et paramétrisée par une ``loi du mur" ;
	\item
	le reste de la colonne, qui réagit plus lentement
	aux variations des conditions à la surface.
\end{enumerate}
L'objectif de ce chapitre est d'évaluer les discrétisations possibles
de cette couche limite.
\begin{itemize}
	\item
Dans une discrétisation FD typique (telle que celle qui sera
utilisée au chapitre \ref{ch:OASchwarz}), ce découpage est implicite :
l'équation d'évolution est utilisée à partir
du premier point de grille et on considère la zone
sous ce point comme la couche limite.
Les flux turbulents sont calculés à l'aide de la solution
	au premier point de grille, ce qui correspond donc à
	l'extrémité supérieure de la couche limite.
\item La construction de discrétisations en volumes finis
	est moins souple dans sa gestion de la couche limite.
	La solution au premier point de grille correspond
	alors à une moyenne sur une cellule qui se trouve
	partagée entre deux zones.
	Des travaux antérieurs ont montré qu'une gestion
	de la couche limite calquée sur les méthodes FD
	conduit à un biais systématique dans l'estimation
	des flux turbulents.
\end{itemize}
Ainsi, \citep{nishizawa_surface_2018} proposent
d'étendre la couche limite de surface à l'entièreté de
la première cellule et d'utiliser un algorithme bulk
adapté aux données moyennées.
\par
Ce chapitre \ref{ch:ND} propose d'implémenter
directement dans la discrétisation FV les hypothèses
existantes dans les formulations bulk.
	Pour ce faire, la reconstruction à l'intérieur
	de la première maille est réalisée à l'aide des fonctions
	particulières intervenant dans les lois du mur et
	non seulement avec des polynômes.
\par
Ces reconstructions particulières permettent notamment
	d'étendre la hauteur de la couche limite au-delà
	de la première maille.
	En effet, le \textit{log-layer mismatch},
	un problème numérique bien connu dans les
	simulations à haute résolution (Large Eddies
	Simulations),
	provient d'une couche limite trop mince.
\par
En étant capable de choisir l'épaisseur de la couche
	limite de surface sur des critères physiques et non
	seulement numériques, la consistance des schémas
	s'en trouve améliorée. La résolution verticale
	peut ainsi être raffinée sans pour autant changer
	les équations continues résolues par la
			discrétisation.
\par
En utilisant un critère de consistance, les stratégies
	de gestion de la couche limite sont
	comparées pour différents types de stratification.
	Une comparaison de ces stratégies est également effectuée
	à l'aide d'un couplage avec une colonne océanique.
\section*{Chapitre \ref{ch:OceanND} : Discrétisation de
la couche limite oceanique}
Le chapitre \ref{ch:OceanND} est tourné vers la définition et l'implémentation
d'une couche limite de surface océanique.
\par
Il est maintenant bien connu que les algorithmes
	bulk doivent utiliser des vents relatifs aux
	courants de surface pour calculer plus précisément
	les flux turbulents.
	Des travaux récents \citep{pelletier_two-sided_2021}
	se sont intéressés à la prise en compte des mouvements
	turbulents dans la description des courants de
	surface. Par symétrie avec les caractéristiques
	de la couche limite atmosphérique, les formulations
	bulk peuvent également intégrer une couche
	limite océanique.
\par
L'implémentation de cette couche limite peut suivre
	les mêmes principes que son homologue atmosphérique.
	Une différence notable est la présence de la
	pénétration du flux solaire, très importante dans
	les premiers mètres sous la surface.
	Ce flux solaire vient contredire la notion de flux
	constant utilisée dans les lois du mur.
\par
Les stratégies de gestion de la couche limite
	océanique sont comparées dans des simulations numériques,
	dans un cas forcé et dans un cas couplé.
\section*{Chapitre \ref{ch:OASchwarz}: Méthodes de Schwarz pour le couplage
discret océan-atmosphère}
Le dernier chapitre de cette thèse met en relation l'étude discrète
des méthodes de Schwarz et l'utilisation d'une couche limite de
surface.
	\par Comme mentionné précédemment,
	la hauteur de la couche limite de surface est
	choisie dans les modèles actuels en fonction de la
	résolution de la grille.
		Pour cette raison, étudier la convergence de
		l'algorithme de Schwarz
		au niveau semi-discret en espace semble pertinent,
		puisque les conditions de transmission sont
		liées au choix de discrétisation.
	\par Le chapitre \ref{ch:OASchwarz} se concentre sur la discrétisation FD
		qui a le mérite d'être à la fois simple à
		manipuler et représentative de l'état de l'art.
		Les conditions de transmission sont simplifiées :
		on formule le flux à l'interface comme égal à
		$C_D |\Delta U| \Delta U$ où $\Delta U$ est le saut
		des solutions à travers l'interface et $C_D$ est
		une constante
		(au lieu d'être une fonction non-linéaire de
		$\Delta U$).
	\par Nous prouvons l'existence et l'unicité
	d'une solution du problème couplé au voisinage d'un état
		stationnaire. Cette preuve est réalisée au niveau
		semi-discret en espace, pour lequel l'état
		stationnaire est unique pour notre choix de
		paramètres.
	\par La convergence des méthodes de Schwarz sur une version
		linéaire du problème (où on pose
		$\alpha=C_D |\Delta U| = {\rm const}$)
		est tout d'abord discutée. L'optimisation d'un
		paramètre de relaxation $\theta$ introduit dans
		la condition de transmission donne un paramètre
		optimal $\theta \approx 1$.
	\par Le problème non-linéaire est ensuite étudié en passant
		par une linéarisation de la condition de transmission
		autour de l'état stationnaire. Au contraire
		de ce qu'on peut trouver habituellement dans
		les études de convergence (qui utilisent dans
		l'immense majorité des cas des conditions de
		transmission linéaires),
		le facteur de convergence pour une fréquence donnée
		change d'une itération à une autre. 
		Une optimisation est également réalisée dans ce cas
		et donne un paramètre optimal $\theta \approx 1.5$.
	\par Finalement, des expériences numériques
		valident les résultats obtenus analytiquement.

\chapter*{Introduction}
\label{ch:introduction}
% Qu'est-ce qu'un modèle numérique
Numerical models are ubiquitous in oceanography, climatology and
meteorology. They become more and more accurate as the
computational power increases and as we refine our knowledge on
both physical phenomena and numerical behaviour.
As long as we want to encompass more phenomena in the simulations,
both mathematical models and their implementations must be
adapted to new scales and new challenges.
\par % Le couplage océan-atmosphère
The interactions between the ocean and the atmosphere are crucial
for climate projections and mid-term to long-term weather forecasts.
In terms of simulations, it means that the numerical models of
ocean and of atmosphere must be \textit{coupled}.
Indeed, there is no ocean-atmosphere model in a single
block to our knowledge:
the scales and dynamics involved in those two systems are
sufficiently different to be simulated by specific tools
and there are historically two separate communities behind the
models.
\par % Qu'est-ce qu'une discrétisation
A numerical implementation do not solve directly the mathematical
model (the \textit{continuous} equations) but an approximation of it
(the \textit{discrete} equations). There exist many of these
approximations (called \textit{discretizations}), the choice of which
is not taken lightly: a discretization introduces some
numerical error and sometimes enforces desirable mathematical
properties.
\par % La couche limite de surface
The interface between the models in the case of
the ocean-atmosphere coupling is specifically treated because
of the turbulent motions that appear around the sea surface.
The numerical treatment of this \textit{surface layer}
is both physically justified and numerically driven.
Improving the discretization of the surface
layer is a key for the accuracy of air-sea exchanges.
\par % Les méthodes de Schwarz
\textit{Schwarz domain decomposition methods} consist in solving
iteratively the models (here, the ocean model and the
atmosphere model) until the solutions at the interface match.
These coupling methods are known for being relatively slow.
However, the mismatch at the interface is multiplied
at each iteration by a \textit{convergence factor} which
can be optimized to accelerate the convergence of the
Schwarz method.
Several ocean-atmosphere coupling are implemented with
a method corresponding to a single step of a Schwarz method.
\par % Selon Marti, ça merde un peu des fois
\citep{marti_schwarz_2021} used several Schwarz iterations
instead of a single one to evaluate the loss of precision
generated by the coupling. A significant difference was found
"\textit{after sunrise
and before sunset, when the external forcing (insolation at
the top of the atmosphere) has the fastest pace of change}".
Although it would be unpractical to use multiple
steps of Schwarz methods in actual climate simulations,
the study of convergence of those methods is of interest,
first for the evaluation and then for the reduction of the
error in the first iteration\footnote{However, note that
in the overwhelming majority of convergence studies
(including this thesis) the convergence factor of
Schwarz methods do not characterize the very first iteration
because the initial value is not the solution of the
differential equation considered.}.
\par % Optimisation discrète
The discretizations interact with the coupling methods:
if the study of the convergence factor is pursued at the
\textit{continuous} level, the coupling of the \textit{discrete}
equations may converge differently than expected. Moreover,
if the convergence is optimized at the discrete level, one expect
to obtain a faster numerical convergence.
Besides, the numerical treatment of the surface layer already
appears in the \textit{continuous} equations.
It can appear more natural to directly study at the \textit{discrete}
level the coupling methods when taking into account
the surface layer.
\par % objectifs
In this context, the objectives of this thesis are:
\begin{enumerate}
\item to improve our knowledge on the effect of the discretization
on the convergence factor of Schwarz methods;
\item to discuss the numerical treatment of the surface layer;
\item to study the presence of a surface layer at the interface in
the heterogeneous coupling ocean-atmosphere.
\end{enumerate}
\subsubsection*{The surface layer}
The vertical area next to the sea surface contains
strong disparities and can be modeled by a so-called
"law of the wall". A review of the wall modelling was done by
\citep{larsson_large_2016}, focusing the very high resolutions.
In particular, it is emphased that the height of the surface layer
should be chosen based on both physical and numerical criteria.
\par
\citep{pelletier_two-sided_2021} introduced an oceanic surface layer
in the \textit{bulk formulations} which characterize the law of
the wall in the atmosphere models.
\citep{nishizawa_surface_2018} discovered a systematic bias
created in some discretizations when using the vertically
averaged data in the \textit{bulk formulations}. They propose to
include the averaging of the data inside the bulk.
\par
Moreover, the hypotheses on which relies the bulk formulations
are generally not included in the discretization.
We propose to reunite the hypotheses in the bulk formulations
and in the discretization in order to
approximate more rigorously the continuous equations.
\par
The bulk formulations are strongly nonlinear.
The well-posedness of the ocean-atmosphere coupling is discussed
in \citep{lions_mathematical_1995} with linear conditions at
the interface and non-linearities inside the computational domains;
A study of the well-posedness focused on the ocean domain
\citep{chacon-rebollo_existence_2014} shows the existence and
unicity in the neighborhood of a steady state. The method
used in this article is applied to a nonlinearity
representative of bulk methods in Chapter
\ref{ch:OASchwarz}.
%
% \subsubsection*{Nonlinearities due to the surface layer}
% Finally, Chapter \ref{ch:OASchwarz} focuses on the convergence
% of Schwarz methods for the Ekman problem. We use idealized
% equations in the inner domains to focus on the non-linear
% condition at interface representing the bulk algorithm.
% Non-linearities in convergence study of Schwarz methods
% can be found in \myTD{sources: non-linear schwarz}.
% We also prove the well-posedness of the Ekman problem
% at a discrete level, following
\subsubsection*{Schwarz methods}
The choice of the coupling methods to compute the interactions
between the ocean and atmosphere is not simple.
In addition to the trade-off between the computational time required and
the precision achieved, the coupling methods
also face some constraints:
\begin{itemize}
	\item some quantities must be conserved by the method
		(e.g. water, energy);
	\item the atmosphere and ocean models being very
		sophisticated, the methods should not be intrusive;
	\item it is generally unaffordable to call multiple times
		the ocean and atmosphere models.
\end{itemize}
With wisely chosen interface conditions, Schwarz methods can
satisfy those constraints. In the context of making a very
small number of iterations, accelerating the convergence
corresponds to increasing the precision of the coupling.
Discrete transparent boundary conditions
\citep{zisowsky_discrete_2006} allow to converge up to the
numerical precision in two iterations.
However, those boundary conditions are non-local in time.
Schwarz methods can be instead accelerated
by an optimization \citep{gander_optimized_2006} where
the boundary conditions at interface are local in time.
\par
This optimization, when carried in a semi-discrete in space
\citep{wu_semi-discrete_2014-1} or in a fully discrete setting
\citep{wu_optimized_2017-2} yields a convergence rate which is
closer to the one observed in numerical simulations and
generally leads to faster convergence.
However, the last two study focus on electric circuits where
there are overlaps between the domains. \citep{gander_analysis_2018}
analyze the presence of an overlap but no previous study
stands in the discrete case without overlap (note that the
semi-discrete 2D stationary case was done by
\citep{gerardo-giorda_optimized_2005}).
Discrete Schwarz Waveform Relaxation studies with other equations
can be found in \citep{haynes_fully_2020} together with
the finite domain case.
In the continuous case, a lot of efforts have been made
(e.g. \citep{thery_analysis_2021} where the viscosity
is varying and is discontinuous at interface or
\citep{haberlein_optimized_2014} who give an overview of
nonlinear systems).
\par
\citep{clement_discrete_2022-1} (Chapter
\ref{ch:discreteSchwarzAnalysis}) studies the effect
of the discretisation in space and time of the reaction-diffusion
problem described in
\S\ref{sec:airseaSCM_reactionDiffusionSection}.
\par
Chapter \ref{ch:approximatedDiscreteSchwarz}
introduces alternative methods to estimate the convergence
speed of the discrete problem with less computations.
The ultimate goal would be to obtain the convergence speed of
the discrete or semi-discrete optimization with a continuous study.
\par
Finally, Chapter \ref{ch:OASchwarz} treats the nonlinearity due
to the presence of the surface layer. Nonlinearities
at interface were introduced in domain decomposition methods by
\citep{caetano_schwarz_2011} where they created interface
conditions resembling to linear ones to keep the convergence
properties. We start instead with an existing nonlinear interface
and study its convergence properties.
\subsubsection*{Outline}
This thesis contains six chapters.
Chapter \ref{ch:airseaSCM} focuses on the derivation of the equations
driving the ocean-atmosphere coupling.
In Chapter \ref{ch:discreteSchwarzAnalysis}, Schwarz algorithms are studied at a discrete
levels in a very idealized setting.
Chapter \ref{ch:approximatedDiscreteSchwarz} presents some approximations simplifying the
convergence study of Chapter \ref{ch:discreteSchwarzAnalysis}.
Chapter \ref{ch:ND} and \ref{ch:OceanND} discuss the discretization of the
surface layer. Chapter \ref{ch:ND} introduces the ideas and discretization
for the atmosphere surface layer and Chapter \ref{ch:OceanND} extends the
discussion to the ocean surface layer.
Finally, a study of Schwarz methods is applied
in the presence of a surface layer in Chapter \ref{ch:OASchwarz}.
