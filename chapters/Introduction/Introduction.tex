\chapter*{Résumé détaillé}
\section*{Chapitre 1: Modélisation d'une colonne océan-atmosphère}
\section*{Chapitre 2: Analyse discrète des méthodes de Schwarz avec des équations de réaction-diffusion}
Le chapitre 2 a été publié sous la forme de l'article
\citep{clement_discrete_2022-1} dans le journal
\textit{SMAI Journal of Computational Mathematics}.
\begin{enumerate}
\item \textbf{Modèle continu et méthode de Schwarz}:
\myTD{manque une phrase ici}
\begin{subequations}
\begin{align}
\partial_t u_j +( r - \nu_j \partial_x^2) u_j &= f_j ~~~~~~~~
(j=o,a) &\qquad&
(x,t) \in (-\infty,0) \times ]0,T] \label{eq:dr1} \\
u_j(x,0) &= u_{j,0}(x)   &\qquad&  x \in (-\infty,0)  \\
u_o(0^-,t) &=  u_a(0^+,t) &\qquad& t \in [0,T] \label{eq:interface-dir} \\
\nu_o \partial_x u_o(0^-,t) &= \nu_a \partial_x u_a(0^+,t) &\qquad& t \in [0,T] \label{eq:interface-neu} 
\end{align}
\label{eq:resume_francais_model-problem}
\end{subequations}
Pour résoudre ce problème couplé, les domaines
$\widetilde{\Omega_o} = \mathbb{R_{-}}$ et
et $\widetilde{\Omega_a} = \mathbb{R_{+}}$ voient leurs équations
correspondantes \eqref{eq:dr1} être résolues tour à tour
dans $\widetilde{\Omega_j} \times [0,T]$
en utilisant les équations
\eqref{eq:interface-dir} et {eq:interface-neu}
en tant que conditions de bord.
Cette méthode s'appelle \textit{Relaxation d'onde de Schwarz avec
des conditions de transmission Dirichlet-Neumann}.
% Les conditions de Dirichlet et Neumann \eqref{eq:interface-dir} et
% {eq:interface-neu} peuvent être remplacées par des combinaisons
% linéaires de ces conditions afin d'obtenir des conditions de Robin.
L'étude de la convergence de cette méthode se fait en deux étapes :
\begin{itemize}
\item Les équations \eqref{eq:dr1} sont résolues dans l'espace de
	Fourier : la dimension temporelle devient un espace
	de fréquences. On obtient une formule analytique de
	la différence entre $u_j^k$
	($k$ note l'itération de Schwarz courante)
	et la solution couplée.
\item Les conditions de transmission à l'interface permettent
	de quantifier l'évolution de cette différence au fur et
	à mesure des itérations.
\end{itemize}
 La convergence est \textit{linéaire},
 c'est à dire que la différence entre $u_j^k$ et la solution couplée
 est multipliée par un \textit{facteur de convergence} ne dépendant
 pas de $k$. Dans le cas continu avec des conditions
 de transmissions Dirichlet-Neumann le facteur de convergence ne
 dépend pas non plus de la fréquence ni de $r$ et vaut
 $\rho_{DN}^{(c,c)}=\sqrt{\frac{\nu_o}{\nu_a}}$.
\item \textbf{Facteur de convergence semi-discrete en temps}:
la convergence observée lors de l'implémentation d'un méthode
de Schwarz dépend des discrétisations en temps et en espace utilisées.
On utilise la transformée en Z au lieu de la transformée de Fourier
pour étudier un signal semi-discret en temps.
Si le passage du continu au discret est aisé pour les discrétisations
en temps qui ne comportent qu'une étape, les discrétisations
à plusieurs étapes demandent une attention particulière. En effet,
les conditions d'interface sont interpolées durant les étapes
intermédiaires. Cette interpolation (que nous nommons $\gamma$)
modifie la vitesse de convergence de la méthode de Schwarz,
particulièrement dans les hautes fréquences temporelles.
\item \textbf{Discrétisation en espace} :
la convergence des méthodes de Schwarz a été étudiée pour
deux discrétisations en espace. La première est une discrétisation
de référence utilisant des différences finies (FD)
centrées d'ordre 2 ;
la deuxième est une discrétisation en volumes finis (FV) basée sur
des splines paraboliques.
Chaque maille est ainsi caractérisée par un polynôme d'ordre 2
et l'approximation volumes finis est déduite du raccord des polynômes
entre les mailles.
Retrouver numériquement la solution mono-domaine s'avère plus
naturel lorsque la discrétisation FV est utilisée;
Avec la discrétisation FD, utiliser une condition de transmission
particulière dans ce but diminue
drastiquement la vitesse de convergence.
\item \textbf{Analyse discrète}:
une méthode d'analyse est donnée et appliquée pour les combinaisons
de deux discrétisations en temps et en espace. Ces combinaisons
sont montrées stables en étudiant les valeurs propres des
matrices à inverser.
\end{enumerate}
Finalement, en utilisant des conditions de transmissions
contenant des degrés de liberté, la vitesse de convergence
des méthodes de Schwarz peut être accélérée en optimisant les
paramètres introduits.
Si l'optimisation se fait au niveau continu, les paramètres
choisis ne seront en général pas optimaux pour la vitesse de
convergence observée numériquement. Au contraire, en optimisant
sur le facteur de convergence discret on obtient des vitesses de
convergence supérieures dans les expériences numériques.
\par
Le facteur de convergence discret peut cependant s'avérer
contraignant à calculer, en particulier pour des discrétisations
en temps à plusieurs étapes ou pour des discrétisations d'ordre
élevé.
\section*{Chapitre 3: Approximation du facteur de convergence discret des méthodes de Schwarz}
 L'objet du chapitre 3 est donc d'étudier des approximations
qui rendent le facteur de convergence plus facile à calculer
que dans le cas discret tout en étant plus précis que le facteur
de convergence continu dans l'approximation de la vitesse
de convergence observée numériquement.
La première approximation étudiée est la méthode des
\textit{équations modifiées} dans le calcul du facteur de convergence.
La seconde est la combinaison des analyses semi-discrètes
pour approcher l'analyse complètement discrète.
\subsection*{Equations modifiées}
La méthode des équations modifiées consiste à étudier au niveau
continu non pas l'équation différentielle originale mais celle
qui est résolue par le schéma numérique. \myTD{manque le actual :(}
Ces équations modifiées sont obtenues à l'aide d'un développement
de Taylor de la discrétisation.
\subsubsection*{Equations modifiées en temps}
Lors du calcul du facteur de convergence, l'utilisation d'équation
modifiées en temps se traduit par un changement de la variable
fréquencielle $\omega$ utilisée dans la transformée de Fourier :
\begin{itemize}
	\item
La complexité de l'étude continue des équations modifiées en temps
est similaire à celle semi-discrète des schémas à une étape.
	\item
Les équations modifiées en temps simplifient l'étude de convergence
des schémas à deux étapes. Cependant, l'utilisation du développement
de Taylor cache l'opération d'interpolation des données de
transmission dans les étapes intermédiaires. Ainsi, lorsque la
différence entre les facteurs de convergence continu
et semi-discret en temps vient de cette opération d'interpolation,
les équations modifiées ne permettent pas d'approcher le facteur
de convergence semi-discret.
\end{itemize}
\subsubsection*{Equations modifiées en espace}
L'utilisation des équations modifiées sur les schémas en espace
génère d'autres observations :
\begin{itemize}
	\item 
	le développement de Taylor introduit des dérivées
d'ordre plus élevées quand dans l'équation originale. Pour
obtenir le caractère bien posé des équations modifiées, il est donc
nécessaire d'ajouter des conditions d'interface. Celles-ci peuvent
être judicieusement choisies à partir de la discrétisation en espace
proche du bord.
	\item Il n'est en général pas suffisant de n'étudier
		l'effet de la discrétisation seulement sur
		l'équation aux dérivées partielles. En effet,
		les conditions d'interface et de bord sont
		elles aussi affectées par la discrétisation et
		leur version discrète doit être utilisée
		dans le calcul du facteur de convergence.
	\item Dans le cas général, l'utilisation des équations
	modifiées en espace augmente l'ordre de l'équation aux
	dérivées partielle et ne rend pas plus aisé
	le calcul du facteur de convergence par rapport
		au calcul semi-discret.
	\item Dans le cas particulier d'une équation ne faisant
	intervenir
	qu'une seule différentiation en espace de n'importe
	quel ordre :
	une astuce de calcul permet de se ramener au cas où un simple
	changement de variable fréquencielle suffit pour caractériser
		l'équation modifiée.
\end{itemize}
%
\subsection*{Combinaison des facteurs de convergence}
Il est possible de combiner les facteurs
de convergence semi-discrets (S-D) et continu pour approcher le
facteur de convergence discret selon la formule:
\begin{equation}
	\text{DISCRET} \approx
	\text{S-D EN ESPACE} +
	\text{S-D EN TEMPS} -
	\text{CONTINU}
\end{equation}
%
\par
Pour ces approximations, des expériences numériques constituent la fin
du chapitre 3 et montrent dans quels cas et dans quelle mesure :
\begin{itemize}
	\item
Ces approximations sont efficaces pour approcher le facteur
de convergence discret ;
	\item
la convergence
peut être accélérée en optimisant ces approximations.
\end{itemize}
\section*{Chapitre 4: Vers une discrétisation de la couche limite atmosphérique cohérente avec la théorie physique}
\section*{Chapitre 5: La couche limite océanique}
\section*{Chapitre 6: Méthodes de Schwarz pour le couplage
discret océan-atmosphère}

\chapter*{Introduction}
\label{ch:introduction}
% Qu'est-ce qu'un modèle numérique
Numerical models are ubiquitous in oceanography, climatology and
meteorology. They become more and more accurate as the
computational power increases and as we refine our knowledge on
both physical phenomena and numerical behaviour.
As long as we want to encompass more phenomena in the simulations,
both mathematical models and their implementations must be
adapted to new scales and new challenges.
\par % Le couplage océan-atmosphère
The interactions between the ocean and the atmosphere are crucial
for climate projections and mid- to long-term weather forecasts.
It means that the ocean and atmosphere must be conjointly simulated;
however, there is no ocean-atmosphere model in a single
block to our knowledge.
Indeed, the scales and dynamics involved in those two systems are
sufficiently different to justify the use of separate models
and there are historically two separate communities behind the
models. For those reasons simulations of the air-sea system
always relies on the \textit{coupling} between numerical models describing
ocean and atmosphere.
\par % Qu'est-ce qu'une discrétisation
A numerical implementation do not directly solve the mathematical
model (the \textit{continuous} equations) but an approximation of it
(the \textit{discrete} equations). There exist many of these
approximations (called \textit{discretizations}), the choice of which
is not to be taken lightly: a discretization introduces some
numerical error and sometimes enforces desirable mathematical
properties.
\par % La couche limite de surface
The interface between the models in the case of
the ocean-atmosphere coupling is specifically treated because
of the turbulent motions that appear around the sea surface.
The numerical treatment of this \textit{surface layer}
is both physically justified and numerically driven.
Improving the discretization of the surface
layer is a key for the accuracy of air-sea exchanges.
\par % Les méthodes de Schwarz
\textit{Schwarz domain decomposition methods} consist in solving
iteratively the models (here, the ocean model and the
atmosphere model) until the solutions at the interface match.
These coupling methods are known for being relatively slow.
However, the mismatch at the interface is multiplied
at each iteration by a \textit{convergence factor} which
can be optimized to accelerate the convergence of the
Schwarz method.
Several ocean-atmosphere coupling are implemented with
a method corresponding to a single step of a Schwarz method.
\par % Selon Marti, ça merde un peu des fois
\citep{marti_schwarz_2021} used several Schwarz iterations
instead of a single one to evaluate the loss of precision
generated by the coupling. A significant difference was found
\textit{``after sunrise
and before sunset, when the external forcing (insolation at
the top of the atmosphere) has the fastest pace of change}".
Although it would be unaffordable to use multiple
steps of Schwarz methods in actual climate simulations,
the study of convergence of those methods is of interest,
first for the evaluation and then for the reduction of the
error in the first iteration\footnote{However, note that
in the overwhelming majority of convergence studies
(including this thesis) the convergence factor of
Schwarz methods do not characterize the very first iteration
because the initial value is not the solution of the
differential equation considered.}.
\par % Optimisation discrète
The discretizations interact with the coupling methods:
if the study of the convergence factor is pursued at the
\textit{continuous} level, the coupling of the \textit{discrete}
equations may converge differently than expected. Moreover,
if the convergence is optimized at the discrete level, one expects
to obtain a faster numerical convergence.
Besides, the numerical treatment of the surface layer already
appears in the \textit{continuous} equations.
It is hence appropriate to directly study at the \textit{discrete}
level the coupling methods when taking into account
the surface layer.
\par % objectifs
In this context, the objectives of this thesis are:
\begin{itemize}
\item to improve our knowledge on how the discretization affects
the convergence factor of Schwarz methods;
\item to discuss the numerical treatment of the surface layer;
\item to study the surface layer within the ocean-atmosphere coupling.
\end{itemize}
\subsubsection*{The surface layer}
The vertical area next to the sea surface contains
strong disparities and can be modeled by a so-called
``law of the wall". A review of the wall modelling was done by
\citep{larsson_large_2016}, focusing the very high resolutions.
In particular, it is emphased that the height of the surface layer
should be chosen based on both physical and numerical criteria.
\par
\citep{pelletier_two-sided_2021} introduced an oceanic surface layer
in the \textit{bulk formulations} which characterize the law of
the wall in the atmosphere models.
\citep{nishizawa_surface_2018} discovered a systematic bias
created in some discretizations when using the vertically
averaged data in the \textit{bulk formulations}. They propose to
include the averaging of the data inside the bulk.
\par
Moreover, the hypotheses on which relies the bulk formulations
are generally not included in the discretization.
We propose to reunite the hypotheses in the bulk formulations
and in the discretization in order to
approximate more rigorously the continuous equations.
\par
The bulk formulations are strongly nonlinear.
The well-posedness of the ocean-atmosphere coupling is discussed
in \citep{lions_mathematical_1995} with linear conditions at
the interface and non-linearities inside the computational domains;
A study of the well-posedness focused on the ocean domain
\citep{chacon-rebollo_existence_2014} shows the existence and
unicity in the neighborhood of a steady state. The method
used in this article is applied to a nonlinearity
representative of bulk methods in Chapter
\ref{ch:OASchwarz}.
%
% \subsubsection*{Nonlinearities due to the surface layer}
% Finally, Chapter \ref{ch:OASchwarz} focuses on the convergence
% of Schwarz methods for the Ekman problem. We use idealized
% equations in the inner domains to focus on the non-linear
% condition at interface representing the bulk algorithm.
% Non-linearities in convergence study of Schwarz methods
% can be found in \myTD{sources: non-linear schwarz}.
% We also prove the well-posedness of the Ekman problem
% at a discrete level, following
\subsubsection*{Schwarz methods}
The choice of the coupling methods to compute the interactions
between the ocean and atmosphere is not simple.
In addition to the trade-off between the computational time required and
the precision achieved, the coupling methods
also face some constraints:
\begin{itemize}
	\item some quantities must be conserved by the method
		(e.g. water, energy);
	\item the atmosphere and ocean models being very
		sophisticated, the methods should not be intrusive;
	\item it is generally unaffordable to call multiple times
		the ocean and atmosphere models.
\end{itemize}
With wisely chosen interface conditions, Schwarz methods can
satisfy those constraints. In the context of making a very
small number of iterations, accelerating the convergence
corresponds to increasing the precision of the coupling.
Discrete transparent boundary conditions
\citep{zisowsky_discrete_2006} allow to converge up to the
numerical precision in two iterations.
However, those boundary conditions are non-local in time.
Schwarz methods can be instead accelerated
by an optimization \citep{gander_optimized_2006} where
the boundary conditions at interface are local in time.
\par
This optimization, when carried in a semi-discrete in space
\citep{wu_semi-discrete_2014-1} or in a fully discrete setting
\citep{wu_optimized_2017-2} yields a convergence rate which is
closer to the one observed in numerical simulations and
generally leads to faster convergence.
However, the last two study focus on electric circuits where
there are overlaps between the domains. \citep{gander_analysis_2018}
analyze the presence of an overlap but no previous study
stands in the discrete case without overlap (note that the
semi-discrete 2D stationary case was done by
\citep{gerardo-giorda_optimized_2005}).
Discrete Schwarz Waveform Relaxation studies with other equations
can be found in \citep{haynes_fully_2020} together with
the finite domain case.
In the continuous case, a lot of efforts have been made
(e.g. \citep{thery_analysis_2021} where the viscosity
is varying and is discontinuous at interface or
\citep{haberlein_optimized_2014} who give an overview of
nonlinear systems).
\par
\citep{clement_discrete_2022-1} (Chapter
\ref{ch:discreteSchwarzAnalysis}) studies the effect
of the discretisation in space and time of the reaction-diffusion
problem described in
\S\ref{sec:airseaSCM_reactionDiffusionSection}.
\par
Chapter \ref{ch:approximatedDiscreteSchwarz}
introduces alternative methods to estimate the convergence
speed of the discrete problem with less computations.
The ultimate goal would be to obtain the convergence speed of
the discrete or semi-discrete optimization with a continuous study.
\par
Finally, Chapter \ref{ch:OASchwarz} treats the nonlinearity due
to the presence of the surface layer. Nonlinearities
at interface were introduced in domain decomposition methods by
\citep{caetano_schwarz_2011} where they created interface
conditions resembling to linear ones to keep the convergence
properties. We start instead with an existing nonlinear interface
and study its convergence properties.
\subsubsection*{Outline}
This thesis contains six chapters.
Chapter \ref{ch:airseaSCM} focuses on the derivation of the equations
driving the ocean-atmosphere coupling.
In Chapter \ref{ch:discreteSchwarzAnalysis}, Schwarz algorithms are studied at a discrete
levels in a very idealized setting.
Chapter \ref{ch:approximatedDiscreteSchwarz} presents some approximations simplifying the
convergence study of Chapter \ref{ch:discreteSchwarzAnalysis}.
Chapter \ref{ch:ND} and \ref{ch:OceanND} discuss the discretization of the
surface layer. Chapter \ref{ch:ND} introduces the ideas and discretization
for the atmosphere surface layer and Chapter \ref{ch:OceanND} extends the
discussion to the ocean surface layer.
Finally, a study of Schwarz methods is applied
in the presence of a surface layer in Chapter \ref{ch:OASchwarz}.
