\chapter*{Résumé détaillé}
\section*{Chapitre 1: Modélisation d'une colonne océan-atmosphère}
\section*{Chapitre 2: Analyse discrète des méthodes de Schwarz avec des équations de réaction-diffusion}
Le chapitre 2 a été publié sous la forme de l'article
\citep{clement_discrete_2022-1} dans le journal
\textit{SMAI Journal of Computational Mathematics}.
\begin{enumerate}
\item \textbf{Modèle continu et méthode de Schwarz}:
\myTD{manque une phrase ici}
\begin{subequations}
\begin{align}
\partial_t u_j +( r - \nu_j \partial_x^2) u_j &= f_j ~~~~~~~~
(j=o,a) &\qquad&
(x,t) \in (-\infty,0) \times ]0,T] \label{eq:dr1} \\
u_j(x,0) &= u_{j,0}(x)   &\qquad&  x \in (-\infty,0)  \\
u_o(0^-,t) &=  u_a(0^+,t) &\qquad& t \in [0,T] \label{eq:interface-dir} \\
\nu_o \partial_x u_o(0^-,t) &= \nu_a \partial_x u_a(0^+,t) &\qquad& t \in [0,T] \label{eq:interface-neu} 
\end{align}
\label{eq:resume_francais_model-problem}
\end{subequations}
Pour résoudre ce problème couplé, les domaines
$\widetilde{\Omega_o} = \mathbb{R_{-}}$ et
et $\widetilde{\Omega_a} = \mathbb{R_{+}}$ voient leurs équations
correspondantes \eqref{eq:dr1} être résolues tour à tour
dans $\widetilde{\Omega_j} \times [0,T]$
en utilisant les équations
\eqref{eq:interface-dir} et {eq:interface-neu}
en tant que conditions de bord.
Cette méthode s'appelle \textit{Relaxation d'onde de Schwarz avec
des conditions de transmission Dirichlet-Neumann}.
% Les conditions de Dirichlet et Neumann \eqref{eq:interface-dir} et
% {eq:interface-neu} peuvent être remplacées par des combinaisons
% linéaires de ces conditions afin d'obtenir des conditions de Robin.
L'étude de la convergence de cette méthode se fait en deux étapes :
\begin{itemize}
\item Les équations \eqref{eq:dr1} sont résolues dans l'espace de
	Fourier : la dimension temporelle devient un espace
	de fréquences. On obtient une formule analytique de
	la différence entre $u_j^k$
	($k$ note l'itération de Schwarz courante)
	et la solution couplée.
\item Les conditions de transmission à l'interface permettent
	de quantifier l'évolution de cette différence au fur et
	à mesure des itérations.
\end{itemize}
 La convergence est \textit{linéaire},
 c'est à dire que la différence entre $u_j^k$ et la solution couplée
 est multipliée par un \textit{facteur de convergence} ne dépendant
 pas de $k$. Dans le cas continu avec des conditions
 de transmissions Dirichlet-Neumann le facteur de convergence ne
 dépend pas non plus de la fréquence ni de $r$ et vaut
 $\rho_{DN}^{(c,c)}=\sqrt{\frac{\nu_o}{\nu_a}}$.
\item \textbf{Facteur de convergence semi-discrete en temps}:
la convergence observée lors de l'implémentation d'un méthode
de Schwarz dépend des discrétisations en temps et en espace utilisées.
On utilise la transformée en Z au lieu de la transformée de Fourier
pour étudier un signal semi-discret en temps.
Si le passage du continu au discret est aisé pour les discrétisations
en temps qui ne comportent qu'une étape, les discrétisations
à plusieurs étapes demandent une attention particulière. En effet,
les conditions d'interface sont interpolées durant les étapes
intermédiaires. Cette interpolation (que nous nommons $\gamma$)
modifie la vitesse de convergence de la méthode de Schwarz,
particulièrement dans les hautes fréquences temporelles.
\item \textbf{Discrétisation en espace} :
la convergence des méthodes de Schwarz a été étudiée pour
deux discrétisations en espace. La première est une discrétisation
de référence utilisant des différences finies (FD)
centrées d'ordre 2 ;
la deuxième est une discrétisation en volumes finis (FV) basée sur
des splines paraboliques.
Chaque maille est ainsi caractérisée par un polynôme d'ordre 2
et l'approximation volumes finis est déduite du raccord des polynômes
entre les mailles.
Retrouver numériquement la solution mono-domaine s'avère plus
naturel lorsque la discrétisation FV est utilisée;
Avec la discrétisation FD, utiliser une condition de transmission
particulière dans ce but diminue
drastiquement la vitesse de convergence.
\item \textbf{Analyse discrète}:
une méthode d'analyse est donnée et appliquée pour les combinaisons
de deux discrétisations en temps et en espace. Ces combinaisons
sont montrées stables en étudiant les valeurs propres des
matrices à inverser.
\end{enumerate}
Finalement, en utilisant des conditions de transmissions
contenant des degrés de liberté, la vitesse de convergence
des méthodes de Schwarz peut être accélérée en optimisant les
paramètres introduits.
Si l'optimisation se fait au niveau continu, les paramètres
choisis ne seront en général pas optimaux pour la vitesse de
convergence observée numériquement. Au contraire, en optimisant
sur le facteur de convergence discret on obtient des vitesses de
convergence supérieures dans les expériences numériques.
\par
Le facteur de convergence discret peut cependant s'avérer
contraignant à calculer, en particulier pour des discrétisations
en temps à plusieurs étapes ou pour des discrétisations d'ordre
élevé.
\section*{Chapitre 3: Approximation du facteur de convergence discret des méthodes de Schwarz}
 L'objet du chapitre 3 est donc d'étudier des approximations
qui rendraient le facteur de convergence plus facile à calculer
que dans le cas discret tout en étant plus précis que le facteur
de convergence continu dans l'approximation de la vitesse
de convergence observée numériquement.
\section*{Chapitre 4: Vers une discrétisation de la couche limite atmosphérique cohérente avec la théorie physique}
\section*{Chapitre 5: La couche limite océanique}
\section*{Chapitre 6: Méthodes de Schwarz pour le couplage
discret océan-atmosphère}
