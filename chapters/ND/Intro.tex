\section{Introduction}
\begin{itemize}
\item The solution obtained with a discretization
	should be similar to the continuous solution.
\item The continuous solution should be recovered when
	the space step tends to 0.
\item The height of the surface layer
	is usually determined by the discretization.
\item This chapter presents a discretization such that
	the surface layer can be chosen for physical reasons,
		instead of numerical ones.
\item We want the discretization to be robust with regards to
	the space step. In particular, when the space step
		tends to 0, the surface layer should still
		be parametrized with the same wall law.
\item Since the characteristic
	size of the turbulent eddies in the surface layer are
		proportional to the distance to the surface (\cite{kawai_wall-modeling_2012}),
		it should not be assumed that we solve
		correctly the mechanisms in the first levels
		of a finite differences classical approach
\item The model used to derive this discretization is a 1D Ekman
	equation, with a forcing representing a geostrophic guide.
		The turbulent viscosity is computed with a
		one-equation turbulence closure model. It was
		tested against the code used in \cite{lemarie_simplified_2021}
\item We begin with a neutral stratification for pedagogical purposes,
	then apply the same concepts to a stratified column of
		atmosphere.
\end{itemize}
