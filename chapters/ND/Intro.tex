\section{Introduction}
{\color{red} [J'ai pas encore rédigé l'intro]}
\begin{itemize}
\item The solution obtained with a discretization
	should be similar {\color{red} en quel sens ? }
		to the continuous solution.
\item The continuous solution should be recovered when
	the space step tends to 0.
\item The height of the surface layer
	is usually determined by the discretization.
\item This chapter presents a discretization such that
	the surface layer can be chosen for physical reasons,
	in addition to numerical ones.
\item We want the discretization to be robust with regard to
	the space step. In particular, when the space step
	tends to 0, the surface layer should still
	be parametrized with the same wall law.
	\cite{basu_cautionary_2017} warns that
	using a small space step while keeping
	the same boundary treatment in LES contradicts
		the Monin-Obukhov
	hypothesis of being above the roughness sub-layer.
		{\color{red} TODO dire en chap1 que M-O
		n'est pas universellement admis, et détailler
		un peu plus ici ce que veut dire Basu2017}
\item Since the characteristic
	size of the turbulent eddies in the surface layer are
		proportional to the distance to the surface (\cite{kawai_wall-modeling_2012}),
		it should not be assumed that we solve
		correctly the mechanisms in the first levels
		of a finite difference classical approach.
		\cite{kawai_wall-modeling_2012} propose to use other
		levels than the first one to parameterize the
		wall law in the neutral case.
	\item \cite{maronga_improved_2020} proposed an improved
		boundary condition for stratified Large-Eddy Simulations
		which consists in using as the bulk input
		horizontally averaged values at
		$\delta_{sl}$ chosen for physical and numerical
		reasons (within the surface layer, above the
		roughness sub-layer and in a region where flows are
		well-resolved).
\item Comment est gérée la condition de surface dans les modèles:
	couches à flux constant, les mélanges entre paramétrisation
	de couche limite et d'atm libre, etc.
\item \cite{nishizawa_surface_2018} have shown that a bias is created
	in some models
	because of an underlying assumption that the log layer
		is linear. {\color{red} dit comme ça ça fait étrange}
	\item The model used to derive this discretization is a 1D Ekman
	equation, with a forcing representing a geostrophic guide.
		The turbulent viscosity is computed with a
		one-equation turbulence closure model. It was
		tested against the code used in \cite{lemarie_simplified_2021}
\item {\color{red} objectifs du chapitre}
\item We begin with a neutral stratification for pedagogical purposes,
	then apply the same concepts to a stratified column of
		atmosphere.
\end{itemize}
