\section{Introduction}
Schwarz methods have been studied in Chapters
\ref{ch:discreteSchwarzAnalysis} and
\ref{ch:approximatedDiscreteSchwarz}; we now focus
on the discretisation of the atmospheric surface layer.
The dynamics presented in chapter \ref{ch:airseaSCM} are
recalled in \S \ref{sec:ND_Intro_continuousModel} and
the specific problem of the treatment of the surface layer
is raised in \S \ref{sec:ND_Intro_treatmentSL}.
\subsection{Physical considerations}
\label{sec:ND_Intro_continuousModel}
We consider in this chapter the turbulent Ekman model presented
in Chapter \ref{ch:airseaSCM}.
This 1D vertical model includes the Coriolis effect
and a vertical turbulent term (the stratification
will be introduced later in \S\ref{sec:ND_StratifiedCase}):
\begin{equation}
	\label{eq:ND_Intro_physics_evolutionEq}
	\partial_t u + i f u + \partial_z \langle w'u'\rangle
	= i f u_G 
\end{equation}
\begin{remark}
As it was detailed in \S \ref{sec:airseaSCM_hierarchy_Ekman}
$u$ is considered to be a complex-valued function for an easier
representation of the Coriolis effect.
The vector norm $||\cdot||$ is equivalent to the modulus
$|\cdot|$.
\end{remark}
In \eqref{eq:ND_Intro_physics_evolutionEq}, the
Boussinesq hypothesis is applied on the turbulent flux
$\langle w'u' \rangle = - K_u \partial_z u$ where
$K_u$ is the turbulent viscosity.
In this chapter we focus on this flux and its bottom boundary
expression.
\par
We present a continuous model which describes the atmosphere
column as the coupling of two regions:
\myTD{Voir intro du papier qu'a envoyé Florian: SL=1h pour changer d'état ? Alors
qu'ailleurs c'est +long ?}
\begin{itemize}
		\item the surface layer where the turbulent motion
			comes from the proximity of the ocean surface;
		\item the remainder of the atmosphere where other
			processes can have a significant effect
			on the vertical mixing.
\end{itemize}
In fluid dynamics, the presence of a rough surface makes strong
gradients appear:
due to the no-slip boundary condition
the scales of motion close to the surface are much smaller than
elsewhere in the domain and it is numerically
intractable in most applications
to refine the mesh enough for those small scales.
The models hence exclude from the computational domain
a part of the surface layer, using an adapted boundary
condition.
\cite{mohammadi_rough_1998} show that bulk boundary
conditions can be derived with domain decomposition arguments;
the computational domain is split into two parts:
the surface layer $(0,\delta_{sl})$ and the remainder of
the atmosphere column, $(\delta_{sl}, +\infty)$
(see Figure \ref{fig:ND_NeutralCase_EkmanContinuous}).
%
\par
The surface layer has two main characteristics:
\begin{itemize}
	\item it is well mixed. The governing equation
		is usually stationary because the surface layer
		immediately adjusts to the external parameters.
		As a consequence (and because the Coriolis effect
		and the geostrophic forcing are neglected),
		the flux $K \partial_z u$
		is a constant along the vertical axis.
	\item The vertical profile of $K$ strongly depends 
		on $z$.
\end{itemize}
The equations in the atmosphere column are hence (see Figure
\ref{fig:ND_NeutralCase_EkmanContinuous}):
\begin{subequations}
	\label{eq:ND_NeutralCase_continuousModel}
	\begin{equation}
	\label{eq:ND_NeutralCase_EkmanEq}
  (\partial_t + if) u - \partial_z (K_u \partial_z u) = if u_G
		,~~~~~ z \geq \delta_{sl}
	\end{equation}
	\begin{equation}
	\label{eq:ND_NeutralCase_ConstantFlux}
	K_u \partial_z u
	= u_\star^2
	e_\tau, ~~~~~~~~~~~ z < \delta_{sl}
	\end{equation}
\end{subequations}
where $u_G$ is a constant value representing the geostrophic guide and
$f$ is the Coriolis parameter.
%The discretized profile of $K_u$ will be
%detailed in Section \ref{sec:ND_StratifiedCase_turbulentVisc}.
The orientation of $\partial_z u$ is contained in
$e_\tau = \frac{u(\delta_{\rm sl})-u(0)}
	{||u(\delta_{\rm sl})-u(0)||}$
where the surface current $u(0, t)$ is set to 0 or given by
the ocean model in a coupled situation.
In the neutral case (without stratification) we assume
$u(0,t)=0$ to shorten the
expressions\footnote{
Setting $u(0,t)=0$ amounts to consider the "absolute" wind speed
rather than the "relative" wind w.r.t. the surface currents:
the implementation of current feedbacks in ocean-atmosphere
coupled models is discussed in \cite{renault_implementation_2019}.
}.
The initial and top boundary conditions are
\begin{subequations}
	\label{eq:ND_NeutralCase_initialBdCond}
	\begin{equation}
	\label{eq:ND_NeutralCase_initial}
		\left.u\right|_{t=0} = u_G
	\end{equation}
	\begin{equation}
	\label{eq:ND_NeutralCase_topBdcond}
		\left.\partial_z u\right|_{z=z_{\rm top}} = 0
	\end{equation}
\end{subequations}
%
\begin{figure}
	\centering
	\subimport{images/}{notations_EkmanContinuous.pdf_tex}
	\caption{Continuous equations where the surface layer
	is excluded from the computational domain.}
	\label{fig:ND_NeutralCase_EkmanContinuous}
\end{figure}
\par
% \myTD{ici on en dit trop ou pas assez: on peut citer
% Schlichting (1960) ptet sur le proportional to $z$}
% In the surface layer, the size of the turbulent eddies at height $z$
% is proportional to the distance to the surface $z$
% (e.g. \cite{kawai_wall-modeling_2012}).
%The turbulent viscosity
%linearly scales with $u_\star z$, where the coefficient of
%proportionality is the Von Karman constant $\kappa = 0.40$.
%Combining molecular and turbulent viscosities in the surface layer
The viscosity in the surface layer is
$K_u(z\leq \delta_{sl}) = \kappa u_\star (z + z_{u})$ where
$z_{u} = z_u({K_{{\rm mol}}}, u_\star)$ depends on the
geometry of the space domain; $\kappa=0.40$ is the Von
Karman constant and $K_{\rm mol}$ is the molecular viscosity.
It follows from \eqref{eq:ND_NeutralCase_ConstantFlux} that
\begin{equation}
\label{eq:ND_NeutralCase_WallLaw}
	||u(z, t)|| = \frac{{u_\star}}{\kappa}
	\ln(1+\frac{z}{z_{u}})
\end{equation}
which is called \textit{logarithmic profile} (or in the
stratified case \textit{MOST profile}).
The expression \textit{wall law} can refer to either the
logarithmic profile,
the surface layer viscosity $K_u(z\leq \delta_{sl})$
and the boundary condition \eqref{eq:ND_NeutralCase_ConstantFlux}.
\paragraph{BULK routines}
The routine used to compute the friction velocity $u_\star$
is called ${\rm BULK}$. It takes as parameters:
\begin{itemize}
	\item the (relative) wind speed $u(\delta_{sl})$;
	\item the height of the surface layer $\delta_{sl}$;
	\item the temperature difference
		$\theta(\delta_{sl}) - \theta(0)$
		(for the stratified case);
\end{itemize}
In the neutral case, it starts with a first guess of $z_u$ then
uses \eqref{eq:ND_NeutralCase_WallLaw} to compute iteratively
$u_\star$ and $z_u(K_{\rm mol}, u_\star)$ (see
\cite{fairall_bulk_2003} for more precisions).
\paragraph{Bottom boundary condition}
A typical integration in time from $u(z, t^{n})$ to
$u(z, t^{n+1})$ is:
\begin{enumerate}
	\item Neutral Bulk: Use \eqref{eq:ND_NeutralCase_WallLaw}
		to compute $u_\star = {\rm BULK}(u(\delta_{sl}, t^n))
		:= \displaystyle \frac{\kappa ||u(\delta_{sl}, t^n)||}
			{\ln(1+\frac{\delta_{sl}}{z_{u}})}$
  \item Integrate in time \eqref{eq:ND_NeutralCase_EkmanEq}
  using \eqref{eq:ND_NeutralCase_ConstantFlux} as a boundary condition
		either explicitly: 
	\begin{equation}
		\left.K_u \partial_z u
		\right|_{z\leq \delta_{\rm sl}, t^{n+1}}
		= u_\star^2 \frac{u(\delta_{\rm sl},
		t^{\emphase{n}})}
		{||u(\delta_{\rm sl}, t^n)||}
	\end{equation}
		or implicitly:
	\begin{equation}
		\label{eq:ND_NeutralCase_implicitCond}
		\left.K_u \partial_z u
		\right|_{z\leq \delta_{\rm sl}, t^{n+1}}
		= u_\star^2 \frac{u(\delta_{\rm sl},
		t^{\emphase{n+1}})}
		{||u(\delta_{\rm sl}, t^n)||}
	\end{equation}
	\cite{lemarie_stability_2015} showed that the explicit
	numerical interface condition is not necessarily
	stable: we choose to use the implicit condition and note
		$u_\star^2 e_\tau$ the right-hand side of
		\eqref{eq:ND_NeutralCase_implicitCond}.
	% citation: IFS Part IV end of page 48
	This implicit implementation is used for instance in the
	ECMWF model (see Section 3.5 of \citep{ecmwf_ifs_2020}).
\end{enumerate}
\subsection{Treatment of the surface layer}
\label{sec:ND_Intro_treatmentSL}
The exclusion of the surface layer from the computational
domain is justified by the lack of numerical resolution
close to the surface: the continuous model
\eqref{eq:ND_NeutralCase_continuousModel} is split
according to the numerical ressources. The height
of the surface layer $\delta_{sl}$ is hence generally chosen as the
first grid level.
However, in present numerical models,
the evolution equation \eqref{eq:ND_NeutralCase_EkmanEq}
is also solved inside the surface layer which contradicts the
quasi-stationary hypothesis.
In order to improve the treatment of the surface layer,
we can identify three issues (subjectively in decreasing
order of importance and ease of correction):
\myTD{equation pour clarifier 1,2,3 ?}
\begin{enumerate}
	\item $\delta_{sl}$ is too small. This issue is well-known
		in the LES community and is called
		\textit{log-layer mismatch}.
		Since the scales of motion
		in the surface layer are proportional to the distance
		to the surface, it should not be assumed that we solve
		correctly the mechanisms in the first levels
		of a finite difference classical approach.
		\cite{kawai_wall-modeling_2012} propose to use other
		levels than the first one to parameterize the
		wall law in the neutral case.
		\cite{maronga_improved_2020} give an improved
		boundary condition for stratified Large-Eddy Simulations
		which consists in using as the bulk input
		horizontally averaged values at
		$\delta_{sl}$ chosen for physical and numerical
		reasons (within the surface layer, above the
		roughness sub-layer and in a region where flows are
		well-resolved).
	\item The quasi-stationary hypothesis of the wall law is
		not compatible with treating the first cell like
		the other ones.
		\cite{nishizawa_surface_2018} have shown that a
		bias is created in some Finite Volume models
		because the average of the first cell is considered
		to be the value in its center.
		The discretization should take into account the
		surface layer and adapt to the wall law.
	\item It is expected from most discretizations to be
		\textit{consistent}: the continuous
		model should be recovered when the space step tends
		to zero.  In particular, the surface layer should
		still be parametrized with the same wall law which
		is not the case when $\delta_{sl}$ is chosen
		as the first grid
		level. \cite{basu_cautionary_2017} warn that
		in several recent LES studies the space step is
		so small that the Monin-Obukhov hypotheses
		(that the height of the roughness elements is small:
		$z_u \ll \delta_{sl}$)
		do not apply: the consistency problem is not only
		a theoretical consideration.
\end{enumerate}
This chapter aims to study the treatment of
the surface layer and to address those issues.
In Section \ref{sec:ND_NeutralCase}, the Finite Volume
discretisation used in Chapter \ref{ch:discreteSchwarzAnalysis}
is recalled and we introduce several strategies to treat
the surface layer.
In particular, one of the strategies ("FV free") enforces
the wall law in the surface layer and does not
assume a specific $\delta_{sl}$.
The discussion is extended to a stratified case
in Section \ref{sec:ND_StratifiedCase} then
the strategies are compared in Section \ref{sec:ND_Consistency}.
