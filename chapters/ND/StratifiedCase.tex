\section{Stratified case}
\label{sec:ND_StratifiedCase}
Now that all the ideas have been presented in the neutral case,
we extend the discussion to the case of a stratified column.
A stratified fluid has layers of different densities.
We assume for simplicity that the atmosphere
is dry and that the density variation is
proportional to the variation of the temperature:
\begin{equation}
	\label{eq:ND_StratifiedCase_rhoProptoTheta}
	\partial_z \rho \propto
	- \partial_z \theta
\end{equation}
The stratification is hence given by a temperature profile:
if the temperature increases with $z$, the column is said
to be stable and the vertical mixing is reduced. This is
typically the case above sea ice or during the night.
On the contrary an unstable stratification comes from
a decreasing temperature with $z$: it happens
during daytime of diurnal cycles and enhance the
vertical mixing.
%
Finally, if the temperature is a constant, the neutral case is
recovered.
\par
In this section, we first discuss the continuous and semi-discrete in space
equations, then extend the previous discussion on the wall law to
Monin-Obukhov Similarity Theory (MOST).
Finally, Section \ref{sec:ND_StratifiedCase_turbulentVisc} derives the
profile of viscosities depending on the Turbulent Kinetic Energy
(TKE).
\subsection{Continuous model and Finite Volume discretization}
The stratification in the column is given by the
potential temperature: it will be integrated in time
together with $u$.
The continuous equations are given in \S
\ref{sec:ND_StratifiedCase_continuousModel}
and their discretisation is detailed in
\S \ref{sec:ND_StratifiedCase_FVDis}.
The stratification affects $u$
in two ways:
\begin{itemize}
	\item
As explained in Chapter \ref{ch:airseaSCM}
the bulk routine in the stratified case
also gives a friction scale $\theta_\star$
for the potential temperature:
\begin{equation}
	u_\star, \emphase{\theta_\star} =
	{\rm BULK}(\emphase{\theta(\delta_{sl})},
	u(\delta_{sl}))
\end{equation}
where we omitted the BULK dependency on
$u(0)$ and on the surface temperature $\theta_s$.
The bulk routine is based on equations
\eqref{eq:ND_StratifiedCase_MOST} of
\S \ref{sec:ND_StratifiedCase_continuousModel}.
\item The TKE (detailed in \S \ref{sec:ND_StratifiedCase_TKE})
	has a buoyancy term which involves the Brunt-Väisälä frequency
	$N^2 = - \frac{g\partial_z \rho}{\rho_0}$.
	With the assumption
	\eqref{eq:ND_StratifiedCase_rhoProptoTheta}
	that $ \partial_z \rho \propto - \partial_z \theta$
	we use instead that $N^2 = - \frac{g\partial_z \theta}
	{\theta_{\rm ref}}$ where $\theta_{\rm ref}=283K$.
\end{itemize}
\subsubsection{Continuous model}
\label{sec:ND_StratifiedCase_continuousModel}
The continuous equations remain the same for $u$
(see \eqref{eq:ND_NeutralCase_continuousModel}),
except that the turbulent viscosity and the
friction scale $u_\star$ depend on the potential temperature
$\theta$, which is governed by:
\begin{subequations}
	\label{eq:ND_StratifiedCase_continuousModel}
	\begin{equation}
	\label{eq:ND_StratifiedCase_EkmanEq}
\partial_t \theta -\partial_z (K_{\theta} \partial_z \theta)
	= F_{\theta},~~~~~~~~~~~~~~ z \geq \delta_{sl}
	\end{equation}
	\begin{equation}
	\label{eq:ND_StratifiedCase_ConstantFlux}
	\left.K_\theta \partial_z \theta
	\right|_{z\leq \delta_{\rm sl}}
		= \emphase{\theta_\star u_\star} = C_H
	||u(\delta_{\rm sl}) - u(0)||
	\left(\theta(\delta_{\rm sl}) - \theta_{s}\right)
	,~~~ z < \delta_{sl}
	\end{equation}
\end{subequations}
The viscosities $K_u, K_\theta$ will be detailed with
the discretization of the turbulent kinetic energy in section
\ref{sec:ND_StratifiedCase_turbulentVisc}.
In the surface layer, Monin-Obukhov Similarity Theory (MOST)
profiles for $u$ and $\theta$ are
\begin{equation}
\label{eq:ND_StratifiedCase_MOST}
\begin{aligned}
	||u(z)-u(0)|| = \frac{u_\star}{\kappa}
    \left(
	\ln(1+\frac{z}{z_{u}})
	\emphase{- \psi_u\left(\frac{z}{L_{MO}}\right)}
    \right)
    \\
    \theta(z) - \theta_s = 
	\frac{\emphase{\theta_\star}}{\kappa}
    \left(
	\ln(1+\frac{z}{z_{\theta}})
	\emphase{- \psi_\theta\left(\frac{z}{L_{MO}}\right)}
\right)
\end{aligned}
\end{equation}
where $z_\theta$ typically depends on $u_\star$, $K_{{\rm mol}}$
and $L_{MO} = \frac{\theta(\delta_{\rm sl})
u_\star^2}{\theta_\star \kappa g }$.
Similarly to the neutral case, the semi-discrete in time
boundary condition is not explicit:
$C_H^n||u^n(\delta_{sl})- u^n(0)||$ is computed with the values
at the current time step whereas the temperature variation
is taken at time $t^{n+1}$: the boundary condition for temperature
is
\begin{equation}
	K_\theta \partial_z \theta = C_H^{\emphase{n}}||u^\emphase{n}(\delta_{sl})- u^{\emphase{n}}(0)||
	(\theta^\emphase{n+1}(\delta_{\rm sl}) - \theta_s^\emphase{n+1}),
	~~~~ z < \delta_{\rm sl}
\end{equation}

\subsubsection{Finite Volume discretization}
\label{sec:ND_StratifiedCase_FVDis}
The discretization of $u$ is exactly the same as in the
neutral case (\eqref{eq:ND_NeutralCase_semiDiscreteEkmanEq} and \eqref{eq:ND_NeutralCase_prognosticEqFV}) and the discretization of 
the potential temperature is very similar:
the average potential temperature $\overline{\theta}_{m+1/2}$
evolves with
\begin{equation}
\label{eq:ND_StratifiedCase_semiDiscreteEkmanEqPT}
    \partial_t \overline{\theta}_{m+1/2}
    - \frac{K_{\theta, m+1} {(\partial_z \theta)}_{m+1} - K_{\theta, m} {(\partial_z \theta)}_m}{h_{m+1/2}}
	= \overline{F}_{\theta, m+1/2}
\end{equation}
And the derivative of temperature at $z_{m-1}, z_m, z_{m+1}$ solves
\begin{equation}
\begin{aligned}
\label{eq:ND_StratifiedCase_prognosticPT_FV}
\partial_t \left( \frac{h_{m-1/2}}{6} {(\partial_z \theta)}_{m-1}
+ \frac{h_{m-1/2} + h_{m+1/2}}{3} {(\partial_z \theta)}_m
	+ \frac{h_{m+1/2}}{6} {(\partial_z \theta)}_{m+1} \right)
	~~~~~~~~~~& \\
	-
    \left(
	\frac{K_{\theta, m+1}}{ h_{m+1/2}}{(\partial_z \theta)}_{m+1}
	- \frac{(h_{m-1/2} + h_{m+1/2}) K_{\theta, m}}{h_{m-1/2} h _{m+1/2}}
	{(\partial_z \theta)}_m + \frac{K_{\theta, m-1}}{ h_{m-1/2}}{(\partial_z \theta)}_{m-1}
	\right)&\\
	= \overline{F}_{\theta, m+1/2} - \overline{F}_{\theta, m-1/2}&
\end{aligned}
\end{equation}
Identically to the reconstruction of $u(z)$,
inside the $m$-th cell
$\theta(z) = {\cal T}_{m-\frac{1}{2}}(z-z_{m-\frac{1}{2}})$
where \eqref{eq:ND_NeutralCase_quadraticReconstruction}
gives ${\cal T}$ by replacing $\overline{u}, \phi$
with $\overline{\theta}, \partial_z \theta$.
We also extend the surface flux scheme strategies for the temperature:
the ideas of those strategies remain the same
(see Figure \ref{fig:ND_NeutralCase_summary_sfscheme}) and we only
need to introduce the bottom boundary condition for $\theta$.
\begin{itemize}
	\item With Finite Differences the evolution equation is
		solved at $z_{\frac{1}{2}}$ with the surface flux:
		\begin{equation}
			\underbrace{K_\theta
			\emphase{\partial_z \theta}^{n+1}_\emphase{0}}_{\text{
				Surface flux
			}} =
			C_H|\Delta u|(
			\theta_{\frac{1}{2}}^{n+1} - \theta_s),
			~~~ \text{using } {\rm BULK}
			(\underbrace{\theta_{\frac{1}{2}}^{n},
			u_{\frac{1}{2}}^{n}}_{
				\text{FD values at } z_{\frac{1}{2}}
			})
		\end{equation}
		where 
$C_H|\Delta u| = C_H||u^n(\delta_{sl})- u^n(0)|| = 
\frac{u_\star \kappa}{\ln\left(1+\frac{z}{z_{u}}\right)
    - \psi_\theta\left(\frac{z}{L_{MO}}\right) }$.

	\item With "FV pure" the evolution equation is solved in 
		the whole first grid cell with the reconstructed
		value $\theta(\delta_{sl}) = {\cal T}_{1/2}(0)$:
		\begin{equation}
			\underbrace{K_\theta
			\emphase{\partial_z \theta}^{n+1}_\emphase{0}}_{\text{
				Surface flux
			}} =
			C_H|\Delta u|({\cal T}_{1/2}^{n+1}(0)
			- \theta_s),
			~~~ \text{using } {\rm BULK}
			(\underbrace{{\cal T}_{1/2}^{n}(0),
			{\cal S}_{1/2}^{n}(0)}_{
			\text{Reconstructions at } z_{\frac{1}{2}}
			})
		\end{equation}
	\item With "FV Nishizawa" the evolution equation is solved in
		the whole first grid cell and the adapted bulk
		for averaged values is used:
		\begin{equation}
			\underbrace{K_\theta
			\emphase{\partial_z \theta}^{n+1}_\emphase{0}}_{\text{
				Surface flux
			}} =
			C_H|\Delta u| (
			\overline{\theta}_{\frac{1}{2}}^{n+1}
			- \theta_s),
			~~~ \text{using } \overline{\rm BULK}
			(\underbrace{\overline{\theta}_{\frac{1}{2}}^{n},
			\overline{u}_{\frac{1}{2}}^{n}}_{
			\text{Averages around } z_\frac{1}{2}})
		\end{equation}
		The adapted bulk routines read
		\begin{equation}
		\label{eq:ND_StratifiedCase_MOST}
		\begin{aligned}
			 ||\overline{u}_{\frac{1}{2}}-u(0)||
			 = \frac{u_\star}{\kappa}
			\left(1\emphase{+\frac{z_u}{\delta_{sl}}}\right)
		\left(\ln(1+\frac{z}{z_u})\emphase{-1}\right)
		- \emphase{\Psi_u}\left(\frac{z_1}{L_{MO}}\right)
		    \\
			\overline{\theta}_{\frac{1}{2}} - \theta_s=
			\frac{\theta_\star}{\kappa}
			\left(1\emphase{+\frac{z_\theta}
					{\delta_{sl}}}\right)
		\left(\ln(1+\frac{z}{z_\theta})\emphase{-1}\right)
		- \emphase{\Psi_\theta}\left(\frac{z_1}{L_{MO}}\right)
		\end{aligned}
		\end{equation}
		where $\Psi_x(z) = \int_0^z (\psi_x(z'))dz'$
		for $x=u,\theta$.
	\item With "FV2" in the first grid cell
		the MOST profile for $\theta$ is assumed as in
		\cite{nishizawa_surface_2018}.
		The boundary condition is the flux at the top of the
		surface layer $\delta_{sl}=z_1$,
		using the reconstructed solution at $z_1$:
		\begin{equation}
			\underbrace{K_\theta
			\emphase{\partial_z \theta}^{n+1}_\emphase{1}}_{\text{
				Flux at
			} z_1} =
			C_H|\Delta u|\left(
			{\cal T}_{3/2}^{n+1}
			\left(-\frac{h_\frac{3}{2}}{2}\right)
			- \theta_s\right),
			~~~ \text{using } {\rm BULK}
			\left(\underbrace{{\cal T}_{3/2}^{n}
			\left(-\frac{h_\frac{3}{2}}{2}\right),
			{\cal S}_{3/2}^{n}
			\left(-\frac{h_\frac{3}{2}}{2}\right)}_{
			\text{Reconstructions at } z_1 }\right)
		\end{equation}
\end{itemize}
The limitations of the strategies shown in the neutral case are the
same for the temperature: the "FV pure" and "FV Nishizawa" schemes
assume that the evolution equation stands
in the surface layer and approximate a MOST profile with
a quadratic spline.
The value of $K_\theta$ at $z_0$ cannot be set to the molecular
diffusivity in those cases for the same reason as the
viscosity $K_{u,0}$ in the neutral case.
%
Moreover, the discretizations force $\delta_{sl}$ to the first
grid level $z_{\frac{1}{2}}$ (or $z_1$ in the case of "FV2" and
"FV Nishizawa").
We now derive the stratified version of the "FV free" surface scheme
to eliminate those limitations.
\subsection{FV free}
\label{sec:ND_StratifiedCase_FVfree}
In this \S \ref{sec:ND_StratifiedCase_FVfree} we follow
step by step the derivation of the neutral "FV free" scheme and
apply it to this stratified case. The stratification adds no
complexity: we hence present directly the derivation directly for
any $\delta_{\rm sl} \geq 0$.
Let $k$ such that $z_k \leq \delta_{\rm sl} < z_{k+1}$.
As in the neutral case, the cell $[z_k, z_{k+1}]$ is split
into two parts for the "FV free" surface flux scheme.
The first part $[z_k,\delta_{\rm sl}]$ is contained
in the surface layer where the temperature and wind
follow \eqref{eq:ND_StratifiedCase_MOST}.
The second part is the
"sub-cell" $[\delta_{\rm sl}, z_{k+1}]$ of size $\widetilde{h}$, of
averages $\widetilde{u}, \widetilde{\theta}$
and of sub-grid reconstructions
\begin{equation}
\begin{aligned}
	{\cal S}_{k+1/2}(\xi) = \emphase{\widetilde{u}} +
	\frac{\phi_{k+1} + \phi_\emphase{\delta}}{2} \xi
	+ \frac{\phi_{k+1} - \phi_\emphase{\delta}}{2\emphase{\widetilde{\color{black} h}}}
	\left(\xi^2 - \frac{\emphase{\widetilde{\color{black} h}}^2}{12}\right) \\
	{\cal T}_{k+1/2}(\xi) = \emphase{\widetilde{\theta}} +
	\frac{{(\partial_z \theta)}_{k+1} + 
		{(\partial_z \theta)}_\emphase{\delta}}{2} \xi
+ \frac{{(\partial_z \theta)}_{k+1} - {(\partial_z \theta)}_\emphase{\delta}}
	{2\emphase{\widetilde{\color{black} h}}}
	\left(\xi^2 - \frac{\emphase{\widetilde{\color{black} h}}^2}{12}\right)
\end{aligned}
\end{equation}
where $\xi = z - (z_{k+1} - \frac{\widetilde{h}}{2})$ is defined
for $\delta_{\rm sl} < z < z_{k+1}$.
The relation between
$\overline{u}_{k+1/2}, \overline{\theta}_{k+1/2}$ and
$\widetilde{u},\widetilde{\theta}$ is, similarly to the neutral case:
\begin{equation}
\begin{aligned}
\label{eq:ND_StratifiedCase_relation_tilde_bar}
	\left(\emphase{\widetilde{u}} - u(0)\right)\alpha_{sl, u}
	&= \emphase{\overline{u}_{k+1/2}} - u(0) + \widetilde{h}
	\left(\frac{\emphase{\phi_\delta}}{3} +
	\frac{\emphase{\phi_{k+1}}}{6}\right)
	\left(\alpha_{sl, u} -
	\frac{\widetilde{h}}{h_{k+\frac{1}{2}}}
	\right)\\
	\left(\emphase{\widetilde{\theta}} - \theta_s\right)
	\alpha_{sl, \theta}
	&= \emphase{\overline{\theta}_{k+1/2}} - \theta_s +
	\widetilde{h}\left(
		\frac{\emphase{(\partial_z \theta)_\delta}}{3} +
		\frac{\emphase{{(\partial_z \theta)}_{k+1}}}{6}
	\right)\left(\alpha_{sl, u} -
	\frac{\widetilde{h}}{h_{k+\frac{1}{2}}}
	\right)
\end{aligned}
\end{equation}
where the non-dimensional numbers $\alpha_{sl, u},
\alpha_{sl, \theta}$ depend on the stratification. For $x=u, \theta$,
\begin{equation}
	\begin{aligned}
	\alpha_{sl, x} &= \frac{\widetilde{h}}{h_{k+\frac{1}{2}}} +
	\frac{\frac{1}{h_{k+\frac{1}{2}}}\int_{z_k}^{\delta_{sl}}
		\left(\ln(1+\frac{z}{z_{x}})
		\emphase{- \psi_x\left(\frac{z}{L_{MO}}\right)}\right)
	dz}{\ln(1+\frac{\delta_{sl}}{z_{x}})
		\emphase{- \psi_x\left(\frac{\delta_{sl}}{L_{MO}}\right)
		}} \\
		&=\frac{\widetilde{h}}{h_{k+\frac{1}{2}}} +
 \frac{\frac{1}{{h_{k+\frac{1}{2}}}}
    \left[
	    (z+z_{x})\ln(1+\frac{z}{z_{x}})-z
		+\emphase{ z \Psi_x\left(\frac{z}{L_{MO}}\right)}
	\right]_{z_k}^{\delta_{sl}}
    }{\ln(1+\frac{\delta_{sl}}{z_{x}})
		\emphase{- \psi_x\left(\frac{\delta_{sl}}{L_{MO}}\right)
		}}
\end{aligned}
\end{equation}
Note that, instead of $\psi_x$, we used $\Psi_x$,
the averaged form of the universal
profile stability functions defined in \citep{nishizawa_surface_2018}.
%
Since $\ln(1+\frac{z}{z_{x}})-
\psi_x(\frac{z}{L_{MO}})$ is non-negative and increases with $z$ even
in strongly unstable situations,
$0 < \alpha_{sl, x} \leq 1$ and as in the neutral case
$\alpha_{sl, x}=1 \iff \widetilde{h}=h_{k+1/2}$. This lets us retrieve
the "FV2" surface flux scheme by setting the height of the surface
layer to exactly $z_1$.
\par
Finally, equations \eqref{eq:ND_NeutralCase_boundaryCondFVfree} and
\eqref{eq:ND_NeutralCase_semiDiscreteEkmanEqFVfree} of the
neutral case can be used for
$u$ and the equations for $\theta$ are similar:
the boundary condition of "FV free" is
\begin{equation}
	\label{eq:ND_StratifiedCase_semiDiscreteEkmanEqFVfree}
	\underbrace{K_{\theta, \delta} (\partial_z \theta)_\delta}_{
		\text{Flux at } \delta_{sl}
	} = 
	C_H|\Delta u|\underbrace{
	\frac{
  \overline{\theta}_{k+1/2} - \frac{\widetilde{h}^2}{h_{k+1/2}}
	\left(\frac{{(\partial_z \theta)}_k}{3} +
	\frac{{(\partial_z \theta)}_{k+1}}{6}\right) 
  - \theta_s
}{\alpha_{sl, \theta}}
	}_{\text{Reconstruction at } \delta_{sl}}
\end{equation}
and the evolution equation in the first active grid cell is
\begin{equation}
    \begin{aligned}
	\label{eq:ND_StratifiedCase_semiDiscreteEkmanEqPTFVfree}
	    \partial_t \left(\frac{1}{\alpha_{\rm sl, \theta}(t)}
	    \left(
	    \overline{\theta}_{k+\frac{1}{2}} +
	\widetilde{h}\left(\frac{{(\partial_z \theta)}_\delta}{3} +
	\frac{{(\partial_z \theta)}_{k+1}}{6}\right)
		\left(\alpha_{sl, \theta} -
	    	\frac{\widetilde{h}}{h_{k+\frac{1}{2}}}\right)
	 - (1 - \alpha_{sl, \theta})\theta_s
	    \right) \right)&
	= \\
	    \frac{K_{\theta, k+1}{(\partial_z \theta)}_{k+1} -
	K_{\theta,\delta} {(\partial_z \theta)}_\delta}{\widetilde{h}}
	    + &\overline{F}_{\theta, k+1/2}
    \end{aligned}
\end{equation}
The prognostic equation \eqref{eq:ND_StratifiedCase_prognosticPT_FV}
for $\partial_z \theta$ is also adjusted for the first
cell to replace $h_{k+1/2}$ by $\widetilde{h}$.
\subsection{Turbulent viscosities and diffusivities}
\label{sec:ND_StratifiedCase_turbulentVisc}
As it was explained in Chapter \ref{ch:airseaSCM} the
viscosity $K_u$ and diffusivity $K_\theta$ follow the
Monin-Obukhov theory in the surface layer and are parameterized
above the surface layer by the Turbulent Kinetic Energy (TKE) with a
one-equation closure.
The objective of this section is to derive the discretised
turbulent closure:
\begin{enumerate}
\item
The vertical profiles of $u$ and $\theta$ depend strongly on
the discretisation of the Turbulent Kinetic Energy
and of the mixing lengths. Those discretisations are detailed in
\S \ref{sec:ND_StratifiedCase_TKE}.
\item To obtain smooth solutions, the viscosity and
the diffusivity must also be smooth.
At the top of the surface layer, the parameterization
of $K_u, K_\theta$ with the TKE should follow Monin-Obukhov
similarity theory:
the link is done through the mixing lengths in \S
\ref{sec:ND_StratifiedCase_mixing_lengths_match}.
\end{enumerate}
\subsubsection{Finite Volume discretization of TKE equation}
\label{sec:ND_StratifiedCase_TKE}
The Turbulent Kinetic Energy (TKE) evolves with the equation
\begin{equation}
\label{eq:ND_StratifiedCase_TKE}
    \begin{aligned}
    \partial_t e =
    \underbrace{\partial_z \left(K_e
    \partial_z e\right)}_{\text{diffusion}}
    + \underbrace{K_u ||\partial_z u||^2}_{\text{shear}} 
    - \underbrace{K_{\theta} N^2 }_{\text{buoyancy}}
    - \underbrace{c_{\epsilon}
    \frac{e^{3/2}}{l_{\epsilon}(z)}}_{\text{dissipation}}
    \end{aligned}
\end{equation}
The turbulent viscosities $K_u, K_{\theta}$ and $K_e$ are computed
with constants $C_m, C_s, C_e$ such that $(K_u, K_\theta, K_e) = (C_m , C_s \phi_z(z), C_e)l_m(z)
\sqrt{e(z)}$. The value $\phi_z$ is proportional to the inverse
of the so-called turbulent Prandtl number and is bounded by
$\phi_z^{\rm max} = 2.2$ \citep{lemarie_simplified_2021}.
\begin{equation}
\phi_z= \frac{1}{1+ \max\left(-0.5455,
	0.143\times\frac{ l_m l_\epsilon N^2}{e}\right)}
\end{equation}
\paragraph{Mixing lengths}
The mixing lengths $l_m(z) = \min{(l_{\rm up}(z), l_{\rm down}(z))}$
and $l_\epsilon(z) = \sqrt{l_{\rm up}(z) l_{\rm down}(z)}$
(introduced in \citep{bougeault_parameterization_1989})
are obtained with the method named $l^\star_{\rm D80}$ described in
\cite{lemarie_simplified_2021} that we explain here:
\begin{enumerate}
	\item Define
		\begin{equation}
			\label{eq:ND_StratifiedCase_lD80}
			l^\star_{\rm D80} = \frac{2\sqrt{e}}
			{c_0 ||\partial_z u|| + \sqrt{
				c_0^2 ||\partial_z u||^2 + 2 N^2
			}}, ~~ \text{with } c_0=0.2
		\end{equation}
	\item initialize $l_{\rm up}$ and $l_{\rm down}$ to
		$l^\star_{\rm D80}$;
	\item limit $l_{\rm up}$ by the distance to the top and to
		a strongly stratified area of the air column:
		this is done by ensuring $-\partial_z l_{\rm up} < 1$
		and $l_{\rm up}(z_{\rm top}) \approx 0$.
	\item limit $l_{\rm down}$ by the distance to the surface
		and to a strongly stratified area of the air column:
		this is done by ensuring $\partial_z l_{\rm down} < 1$
		and an appropriate surface layer value.
		The value here is not zero, so that the surface layer
		links correctly with the computational domain.
\end{enumerate}
\begin{figure}
	\centering
	\includegraphics[scale=0.6]{images/mixing_lengths.pdf}
	\caption{Typical vertical profiles of the mixing lengths
	(left) reported with the corresponding TKE (right).
	It is seen that $\partial_z l_{\rm up}$ is limited by $-1$
	for $z\in (50 \; {\rm m}, 150\; {\rm m})$ whereas
	$\partial_z l_{\rm down}$ is limited by 1 for $z<100 \;{\rm m}$.
	The surface flux scheme used is "FV free" which is why the
	profiles start at $\delta_{sl} = 10\;{\rm m}$.}
	\label{fig:ND_StratifiedCase_mixing_lengths}
\end{figure}
Steps 3 and 4 are implemented by applying sequentially from top
(for $l_{\rm up}$) and bottom (for $l_{\rm down}$):
\begin{equation}
	\begin{aligned}
		l_{\rm up}(z_m) &= \min\left(l^\star_{D80}(z_m),~~~
		l_{\rm up}(z_{m+1}) + h_{m+\frac{1}{2}}\right) \\
		l_{\rm down}(z_m) &= \min\left(l^\star_{D80}(z_m),~
		l_{\rm down}(z_{m-1}) + h_{m-\frac{1}{2}}\right)
	\end{aligned}
\end{equation}

Figure \ref{fig:ND_StratifiedCase_mixing_lengths} shows vertical
profiles of $l_{\rm up}, l_{\rm down}, l_m, l_{D80}^\star$. The value of
$l_{\rm down}$ in the surface layer is detailed in
\S \ref{sec:ND_StratifiedCase_mixing_lengths_match} to link
the Monin-Obukhov similarity theory to the turbulent
parameterization based on the TKE.
%
\paragraph{Turbulent Kinetic Energy in the computational domain}
The discretisation of the turbulent kinetic energy
requires some care. Instead of choosing the Finite Volume scheme
used for $u$ and $\theta$, we prefer a Finite Difference
method which has a positivity preserving property.
The energy $e$ and the lengths scales
$l_m, l_\epsilon$ are evaluated at the interfaces
$z_m$ between the cells:
\begin{equation}
\label{eq:ND_StratifiedCase_SemiDiscreteTKE}
    \left(
    \frac{c_\epsilon \sqrt{e^n_m}}{l_\epsilon}
    + \partial_t
    \right) e_m
	=\frac{(K_{e} \partial_z e)_{m+1/2} -
	(K_{e} \partial_z e)_{m-1/2}}{
	z_{m+1/2} - z_{m-1/2}
	}
    + K_u ||\partial_z u||^2
    - K_\theta N^2
\end{equation}
The discretisation of the dissipation term
$\frac{c_\epsilon \sqrt{e_m^n}e_m^{n+1}}{l_\epsilon}$
ensures the preservation
of the positivity of $e$, as long as
$K_u ||\partial_z u||^2 - K_\theta N^2$ is positive.
Indeed, ignoring the diffusion term a backward-Euler
discretisation of \eqref{eq:ND_StratifiedCase_SemiDiscreteTKE}
gives
\begin{equation}
	e^{n+1} = \frac{e^n +
	\left(K_u ||\partial_z u||^2 - K_\theta N^2\right)}
	{1 + \Delta t \frac{c_\epsilon \sqrt{e^n}}{l_\epsilon}}
\end{equation}
which is non-negative if $e^n$ is non-negative.
In the cells where the buoyancy is stronger than the shear,
$K_\theta N^2$ is multiplied by
$\frac{e^{n+1}}{e^n}$ to keep the positivity preserving property.
This multiplication is known as the "Patankar's trick"
\citep{lemarie_simplified_2021}.
%
\paragraph{Boundary conditions}
In the surface layer, because of the strong mixing, $e$ is
quasi-stationary ($\partial_t e = 0$). The energy is given by
an equilibrium between shear, buoyancy and dissipation:
\begin{equation}
	\label{eq:ND_StratifiedCase_TKE_SL_cond}
e(z<\delta_{sl}) = \left(\frac{l_\epsilon}{c_\epsilon}
(K_u||\partial_z u||^2 - K_\theta N^2)\right)^{2/3}
\end{equation}
where $K_u||\partial_z u||^2$ and $K_\theta N^2$ are given by MOST.
A homogeneous Neumann boundary condition is used at the top:
$\partial_z e (z_{\rm top}) = 0$

\subsubsection{Matching viscosities at the surface layer}
\label{sec:ND_StratifiedCase_mixing_lengths_match}
In order to obtain a regular solution from the "FV free"
discretization, we derive here the constraints on mixing length
and on TKE inside the surface layer.
A sub-grid model that suits both
surface layer and free turbulence was proposed by
\citep{redelsperger_simple_2001}, physically
justified by measurements.
In the neutral case, this sub-grid model leads to linear
mixing lengths $l_m, l_{\epsilon}$ in the surface layer;
with stratified fluids, the formulation is more sophisticated
and depends strongly on the Obukhov length.
The link between the surface layer and the regions
further away from the surface is ensured with a linear combination
between the two regimes.
%
\par
Instead of following the latter method,
we aim to keep the mixing lengths \eqref{eq:ND_StratifiedCase_lD80}
of the computational domain and
to set a particular boundary condition for $l_{\rm down}$
to satisfy the Monin-Obukhov Similarity Theory.
Assuming that 
\begin{equation}
	\label{eq:ND_StratifiedCase_viscosities_assumption}
K_m = C_m l_m \sqrt{e} ~~\text{and}~~
K_\theta = C_s l_m \phi_z \sqrt{e}
\end{equation}
inside the SL, we have
for $z\leq \delta_{\rm sl}$
\begin{enumerate}
\item
Monin-Obukhov viscosity profiles in the surface layer:
\begin{equation}
	\label{eq:ND_StratifiedCase_viscosities_SL}
	K_m = \kappa u_\star\frac{z+ z_{u}}{\phi_m(z/L_{MO})} ~~\text{and}~~
K_\theta = \kappa u_\star\frac{z+ z_{u}}{\phi_h(z/L_{MO})}
\end{equation}
\eqref{eq:ND_StratifiedCase_viscosities_assumption}
together with \eqref{eq:ND_StratifiedCase_viscosities_SL} put
a constraint on $\phi_z$ in the surface layer:
$\phi_z = \frac{C_m\phi_m(z/L_{MO})}{C_s\phi_h(z/L_{MO})}
		\forall z\leq \delta_{\rm sl}$.
\item Quasi-equilibrium of the TKE equation:
\begin{equation}
	\label{eq:ND_StratifiedCase_TKE_quasi_equilibrium}
	c_\epsilon \frac{e^{3/2}}{l_\epsilon}=K_m ||\partial_z u||^2 - \frac{g}{\theta_{ref}} K_\theta \partial_z \theta
\end{equation}
\end{enumerate}
We assume that $l_\epsilon$ in
\eqref{eq:ND_StratifiedCase_TKE_quasi_equilibrium} is taken at
time index $n$, so that the energy can be integrated in time
with a proper boundary condition before computing the mixing
length. The characteristic length $l_{\rm up}$ (resp. $l_{\rm down}$) indicates
how much the turbulent mixing is acting upwards (resp. downward).
In the surface layer, it is hence natural to follow the
procedure \eqref{eq:ND_StratifiedCase_lD80} for $l_{\rm up}$,
using the MOST profiles for the shear and buoyancy.
We derive the surface value of $l_{\rm down}$ to link Monin-Obukhov
Similarity Theory with the turbulent closure used here.
\par
Rewriting \eqref{eq:ND_StratifiedCase_TKE_quasi_equilibrium}
with MOST and $l_{\rm up}, l_{\rm down}$ gives:
\begin{equation}
	\label{eq:ND_StratifiedCase_TKE_quasi_equilibrium_simplified}
	e = \left(l_{\rm up} l_{\rm down}\right)^\frac{1}{3}\left(
	\frac{u_\star}{c_\epsilon} 
	\left({u}_\star^2
	\frac{\phi_m(z/L_{MO})}{\kappa (z+z_{u})}
	- \frac{g}{\theta_{\rm ref}}\theta_\star
	\right)
	\right)^\frac{2}{3}
\end{equation}
which is the bottom boundary condition of the TKE
\eqref{eq:ND_StratifiedCase_TKE_SL_cond}.
Now, \eqref{eq:ND_StratifiedCase_viscosities_SL} gives
\begin{equation}
	\label{eq:ND_StratifiedCase_solution_l_down1}
	\min (l_{\rm down}^2, l_{\rm up}^2) = \frac{1}{e} \left(
	\frac{\kappa u_\star}{C_m}
	\frac{z + z_{u}}{\phi_m(z/L_{MO})}
\right)^2
\end{equation}
$l_{\rm down}$ is limited by the distance to the bottom only
in the free-turbulence zone. In the SL, it is given by
\eqref{eq:ND_StratifiedCase_solution_l_down1}
so that \eqref{eq:ND_StratifiedCase_viscosities_SL} is
verified.
We inject \eqref{eq:ND_StratifiedCase_TKE_quasi_equilibrium_simplified}
into the previous equation and get
\begin{equation}
	\label{eq:ND_StratifiedCase_solution_l_down2}
	\begin{cases}
	l_{\rm down}^{7/3} = \frac{1}{l_{\rm up}^{1/3}}
		\frac{u_\star^{4/3}\left(
	\frac{\kappa}{C_m}
	\frac{z + z_{u}}{\phi_m(z/L_{MO})}
	\right)^2}{
	\left({u}_\star^2
	\frac{\phi_m(z/L_{MO})}{c_\epsilon\kappa (z+z_{u})}
	- \frac{g}{\theta_{\rm ref}c_\epsilon}\theta_\star
	\right)^{2/3}} ~~~ {\rm if } ~~l_{\rm down} < l_{\rm up}\\
	l_{\rm down}^{1/3} = \frac{1}{l_{\rm up}^{7/3}}
		\frac{u_\star^{4/3}\left(
	\frac{\kappa}{C_m}
	\frac{z + z_{u}}{\phi_m(z/L_{MO})}
	\right)^2}{
	\left({u}_\star^2
	\frac{\phi_m(z/L_{MO})}{c_\epsilon\kappa (z+z_{u})}
	- \frac{g}{\theta_{\rm ref}c_\epsilon}\theta_\star
	\right)^{2/3}} ~~~ {\rm otherwise}\\
	\end{cases}
\end{equation}
It is actually sufficient to use
\eqref{eq:ND_StratifiedCase_solution_l_down1} with
the assumption $l_{\rm down} < l_{\rm up}$
to compute $l_{\rm down}$ explicitly. It also guarantees that
the MOST viscosity and the MOST diffusivity numerically
scale with $l_m\sqrt{e}$.
However, using \eqref{eq:ND_StratifiedCase_solution_l_down1}
with the surface condition for the TKE
\eqref{eq:ND_StratifiedCase_TKE_SL_cond}
amounts to solving iteratively
(\ref{eq:ND_StratifiedCase_TKE_quasi_equilibrium_simplified}-
\ref{eq:ND_StratifiedCase_solution_l_down1});
using \eqref{eq:ND_StratifiedCase_solution_l_down2}
directly solves the system
(\ref{eq:ND_StratifiedCase_TKE_quasi_equilibrium_simplified}-
\ref{eq:ND_StratifiedCase_solution_l_down1}).
