\tableofcontents
\section{Introduction}
\begin{itemize}
\item The solution obtained with a discretization
	should be similar to the continuous solution.
\item The continuous solution should be recovered when
	the space step tends to 0.
\item The height of the surface layer
	is usually determined by the discretization.
\item This chapter presents a discretization such that
	the surface layer can be chosen for physical reasons,
		instead of numerical ones.
\item We want the discretization to be robust with regards to
	the space step. In particular, when the space step
		tends to 0, the surface layer should still
		be parametrized with the same wall law.
\item {\color{red} log-layer mismatch ref: since the characteristic
	size of the turbulent eddies in the surface layer are
		proportional to the distance to the surface,
		it should not be assumed that we solve
		correctly the mechanisms in the first levels
		of a finite differences classical approach}
\item The model used to derive this discretization is a 1D Ekman
	equation, with a forcing representing a geostrophic guide.
		The turbulent viscosity is computed with a
		one-equation turbulence closure model.
		{\color{red} TODO find reference}
\item We begin with a neutral stratification for pedagogical purposes,
	then apply the same concepts to a stratified column of
		atmosphere.
\end{itemize}
\section{Neutral case}
In this section, we assume that the stratification is neutral.
There is no effect of the stratification on the turbulent viscosity
so the wind speed $u(z, t)$ can be integrated in time
without the temperature and humidity profiles.
{\color{red} presentation of subsections}
\subsection{Continuous Model and Finite Volumes discretization}
\subsubsection{Continuous equations: two separate domains}
\label{sec:ND_NeutralCase}
The continuous equation above the surface layer
($z > \delta_{\rm sl}$) is
\begin{equation}
	\label{eq:ND_NeutralCase_EkmanEq}
  (\partial_t + if) u - \partial_z (\partial_z K_u u) = if u_G
\end{equation}
where $u_G$ is a constant value representing the geostrophic guide.
$f$ is the Coriolis parameter and $K_u$ the turbulent viscosity.
The profile of $K_u$ will be detailed
{\color{red}in section \ref{sec:ND_TurbulentViscosity}}.
inside the surface layer ($z \leq \delta_{\rm sl}$), the flux
$K_u \partial_z u$ is constant: {\color{red} TODO passer un peu plus de temps dessus ?}
\begin{equation}
	\label{eq:ND_NeutralCase_constantFlux}
	\left.K_u \partial_z u \right|_{z\leq \delta_{\rm sl}}
	= u_\star^2
	\frac{u(\delta_{\rm sl}, t)}{||u(\delta_{\rm sl}, t)||}.
\end{equation}
The size of the turbulent eddies at height $z$
is proportional to the distance to the surface $z$.
{\color{red} log-layer mismatch ref}. Combining molecular and
turbulent viscosities gives 
$K_u(z\leq \delta_{sl}) = \kappa u_\star z + K_{u, {\rm mol}}
= \kappa u_\star (z + z_{\star})$.
It follows that
\begin{equation}
\label{eq:ND_NeutralCase_WallLaw}
	||u(z, t) - u(0, t)|| = \frac{{u_\star}}{\kappa}
	\log(1+\frac{z}{z_{\star}}).
\end{equation}
A typical integration in time from $u(z, t^{n})$ to
$u(z, t^{n+1})$ is:
\begin{enumerate}
  \item Use \eqref{eq:ND_NeutralCase_WallLaw} to compute 
	  $u_\star = \frac{\kappa ||u(z, t^n)||}
			{\log(1+\frac{z}{z_\star})}$
  \item Integrate in time \eqref{eq:ND_NeutralCase_EkmanEq}
  using \eqref{eq:ND_NeutralCase_constantFlux} as a boundary condition
		either of Neumann type ($\left.K_u \partial_z u
		\right|_{z\leq \delta_{\rm sl}, t^{n+1}}
	= u_\star^2 \frac{u(\delta_{\rm sl}, t^n)}
		{||u(\delta_{\rm sl}, t^n)||}$)
		or of Robin type ($\left.K_u \partial_z u
		\right|_{z\leq \delta_{\rm sl}, t^{n+1}}
		- \frac{u_\star^2} {||u(\delta_{\rm sl}, t^n)||}
		u(\delta_{\rm sl}, t^{n+1}) = 0$)
\end{enumerate}

\subsubsection{Discretization with Finite volumes}
We define here the discretization used throughout all the present
section. The space domain is divided into $M$ cells delimited by
heights $(z_0=0, .., z_m, .., z_M)$. The size of the $m$-th cell
is $h_{m-\frac{1}{2}}=z_{m}-z_{m-1}$ and the average of $u(z, t)$
over this cell is 
$\overline{u}_{m-\frac{1}{2}}=\frac{1} {h_{m-\frac{1}{2}}}
\int_{z_{m-1}}^{z_m}u(z, t)dz$.
The space derivative of $u$ at $z_m$ is noted $\phi_{m}$.
Averaging \eqref{eq:ND_NeutralCase_EkmanEq} over a cell gives
the semi-discrete equation
\begin{equation}
\label{eq:ND_NeutralCase_semiDiscreteEkmanEq}
	(\partial_t + if) \overline{u}_{m+\frac{1}{2}} - 
	\frac{K_{u, m+1} \phi_{m+1} - K_{u, m} \phi_{m}}
		{h_{m+\frac{1}{2}}} = i f u_G.
\end{equation}
The reconstruction of $u(z, t) = {\cal S}_{m+\frac{1}{2}}
				(z - z_{m+\frac{1}{2}})$
				inside a cell must be decided
to pursue the derivation of the scheme. The simplest choice is
a quadratic polynomial,
${\cal S}_{m+\frac{1}{2}}(\xi) = r_0 + r_1 \xi + r_2 \xi^2$ where
$-\frac{h_{m+1/2}}{2} \leq \xi \leq \frac{h_{m+1/2}}{2}$.
Averaging ${\cal S}_{m+\frac{1}{2}}$ over the cell and
prescribing its space derivative at $z_{m}$ and $z_{m+1}$
($\xi=-\frac{h_{m+\frac{1}{2}}}{2}, \frac{h_{m+\frac{1}{2}}}{2}$)
lead to the following system:
\begin{equation}
    \begin{pmatrix}
    \overline{u}_{m+1/2} \\
    h_{m+\frac{1}{2}} \phi_m \\
	    h_{m+\frac{1}{2}} \phi_{m+1}
    \end{pmatrix} = 
    \begin{pmatrix}
    1 & 0 & \frac{1}{12} \\
    0 & 1 & -1 \\
    0 & 1 & 1 \\
    \end{pmatrix}
    \begin{pmatrix}
    r_0 \\
    r_1 h_{m+\frac{1}{2}} \\
    r_2 h_{m+\frac{1}{2}}^2
    \end{pmatrix}
\end{equation}
The reconstruction of $u(z)$ between $z_m$ and $z_{m+1}$
is then
\begin{equation}
\label{eq:ND_NeutralCase_quadraticReconstruction}
{\cal S}_{m+\frac{1}{2}}(\xi) =
	\Bar{u}_{m+\frac{1}{2}} + 
	\frac{\phi_{m+1}^{} + \phi_{m}^{}}{2} \xi
	+ \frac{\phi_{m+1}^{} - \phi_{m}^{}}{2h_{m+1/2}}
	\left(\xi^2 - \frac{h_{m+1/2}^2}{12}\right).
\end{equation}

The continuity of the solution at cell interfaces (${\cal S}_{m-\frac{1}{2}}\left(\frac{h_{m-1/2}}{2}\right) = {\cal S}_{m+\frac{1}{2}}\left(-\frac{h_{m+1/2}}{2}\right)$) is equivalent to
%
\begin{equation}
\label{eq:ND_NeutralCase_continuityEquationFV}
\frac{h_{m-1/2}}{6} \phi_{m-1} 
+ \frac{2h_{m}}{3} \phi_m  
+ \frac{h_{m+1/2}}{6} \phi_{m+1} = \Bar{u}_{m+\frac{1}{2}} - \Bar{u}_{m-\frac{1}{2}}
\end{equation}
where $h_m = \frac{h_{m-1/2} + h_{m+1/2}}{2}$.
Combining \eqref{eq:ND_NeutralCase_semiDiscreteEkmanEq}
and \eqref{eq:ND_NeutralCase_continuityEquationFV} finally gives
the prognostic equation to integrate
$\partial_z u$ in time:
\begin{equation}
\begin{aligned}
\label{eq:ND_NeutralCase_prognosticEqFV}
(\partial_t + if) \left( \frac{h_{m-1/2}}{6h_m} \phi_{m-1} 
+ \frac{2}{3} \phi_m  
+ \frac{h_{m+1/2}}{6h_m} \phi_{m+1} \right)& \\
-
    \left(
	\frac{K_{u, m+1}}{h_m h_{m+1/2}}\phi_{m+1} - \frac{2 K_{u,m}}{h_{m-1/2} h _{m+1/2}}\phi_m + \frac{K_{u,m-1}}{h_m h_{m-1/2}}\phi_{m-1}
    \right)
&= 0
\end{aligned}
\end{equation}
To reconstruct the solution, $\overline{u}$ should also be 
integrated in time with \eqref{eq:ND_NeutralCase_semiDiscreteEkmanEq}.
\eqref{eq:ND_NeutralCase_prognosticEqFV} and
\eqref{eq:ND_NeutralCase_semiDiscreteEkmanEq} are hence the two 
equations defining our finite volumes discretization.

\paragraph{Fourth order compact scheme}
To get a more accurate scheme,
a fourth degree polynomial can also be used. If
${\cal S}_{m+\frac{1}{2}}^4(\xi) = r_0^4 + r_1^4 \xi + r_2^4 \xi^2 
+ r_3^4 \xi^3 + r_4^4 \xi^4$, we can recover a fourth order compact
scheme by putting as a constraint that the reconstruction
on the boundaries of the cell are
${\cal S}^4_{m+\frac{1}{2}}(\frac{h_{m+\frac{1}{2}}}{2}) =
\overline{u}_{m+\frac{1}{2}} + \frac{h_{m+\frac{1}{2}}}{12}\left(
\phi_m + 5\phi_{m+1}\right) $ and
${\cal S}^4_{m+\frac{1}{2}}(-\frac{h_{m+\frac{1}{2}}}{2}) =
\overline{u}_{m+\frac{1}{2}} - \frac{h_{m+\frac{1}{2}}}{12}\left(
5\phi_m + \phi_{m+1}\right)$.
Then the reconstruction is given by
\begin{equation}
    \begin{pmatrix}
    r_0^4 \\
    r_1^4 h_{m+\frac{1}{2}} \\
    r_2^4 h_{m+\frac{1}{2}}^2 \\
    r_3^4 h_{m+\frac{1}{2}}^3 \\
    r_4^4 h_{m+\frac{1}{2}}^4
    \end{pmatrix}
     = 
    \begin{pmatrix}
    1 & 0 & \frac{1}{12} & 0 & \frac{1}{80} \\
    0 & 1 & -1 & \frac{3}{4} & -\frac{1}{2} \\
    0 & 1 & 1 & \frac{3}{4} & \frac{1}{2} \\
    1 & -\frac{1}{2} & \frac{1}{4} & -\frac{1}{8}
    & \frac{1}{16} \\
    1 & \frac{1}{2} & \frac{1}{4} & \frac{1}{8}
    & \frac{1}{16} \\
    \end{pmatrix}^{-1}
    \begin{pmatrix}
    \overline{u}_{m+1/2} \\
    h_{m+\frac{1}{2}} \phi_m \\
	    h_{m+\frac{1}{2}} \phi_{m+1} \\
	    \overline{u} - \frac{5}{12} h_{m+\frac{1}{2}} \phi_m - \frac{1}{12} h_{m+\frac{1}{2}} \phi_{m+1} \\
	    \overline{u} + \frac{1}{12} h_{m+\frac{1}{2}} \phi_m + \frac{5}{12} h_{m+\frac{1}{2}} \phi_{m+1}
    \end{pmatrix}
\end{equation}

\subsection{Wall law}
We show here several strategies for the discretization
in the surface layer. We first compare 3 state-of-the-art
discretizations then introduce an original one.

The discretization consists in three parts:
\begin{itemize}
	\item How $u(\delta_{\rm sl}, t^n)$ is computed
		(which influence the value of $u_\star$);
	\item How the boundary condition 
		\eqref{eq:ND_NeutralCase_constantFlux} is implemented.
\end{itemize}
\subsubsection{State-of-the-art strategies}
  We now present three strategies that are representative of
  what is done in models. A summary of those discretizations
  is presented in Figure \ref{fig:ND_NeutralCase_summary_sfscheme}.
  \begin{itemize}
	  \item "FV pure": $\delta_{\rm sl} = z_{\frac{1}{2}}$. Similar to a Finite Differences discretization.
	    The reconstruction of $u(z)$ is used to get 
		  $u(z_{\frac{1}{2}})$.
		  The first cell is treated like the others
		  (i.e. \eqref{eq:ND_NeutralCase_EkmanEq} is
		  applied inside the surface layer).
	The bottom boundary condition is 
		  $ K_{u,0} \phi_0^{n+1} = u_\star^2 
		  \frac{{\cal S}_{1/2}^{n+1}(0)}{||{\cal S}_{1/2}^n(0)||}$.
	  \item "FV1": $\delta_{\rm sl} = z_{\frac{1}{2}}$.
		  In state-of-the-art models using finite volumes,
		  $u_{\star}$ is often computed using
		  $u(\delta_{sl}) = \overline{u}_{\frac{1}{2}}$
		\ref{Nishizawa}.
		The corresponding bottom boundary condition is
		$ K_{u,0} \phi_0^{n+1} = u_\star^2 
		  \frac{\overline{u}^{n+1}_{\frac{1}{2}}}
		  	{\overline{u}_{\frac{1}{2}}^n}$.
			{\color{red} TODO références !!}

    \item "FV2":  $\delta_{\rm sl} = z_{1}$. In
    \ref{Nishizawa}, the "FV1" strategy is shown to systematically
	  underestimate $u(\delta_{\rm sl})$ because of the 
	  convexity of the logarithmic profile.
	A surface flux scheme is derived by assuming
	  $\delta_{\rm sl} = z_1$ and by integrating $u(z)$
	  in space in the first cell. Since the reconstruction
	  of $u(z)$ is continuous, it is here equivalent to
	  get $u_\star$ through
	  $u^n(\delta_{\rm sl}) = {\cal S}^n_{\frac{3}{2}}(
	  -\frac{h_{3/2}}{2})$. The first cell is only
	  parametrized with the wall law, and the boundary condition
		  at $z_1$ is $K_{u,1} \phi_1 - u_\star^2 
		  \frac{{\cal S}_{\frac{3}{2}}^{n+1}
		  	(-h_{\frac{3}{2}}/2)}
		  {||u^n(\delta_{\rm sl})||} = 0$
	  
  \end{itemize}
  Note that for the first two strategies, $K_{u,0}$ should actually
  take the value $\left. K_{u}\right|_{z=\delta_{sl}}$.
  {\color{red} explanation ?}

\begin{figure}
	\subimport{images/}{sf_scheme_strategies.pdf_tex}
	\caption{Summary of the surface flux schemes. The
	wind profiles shown here do not correspond to
	numerical values.}
	\label{fig:ND_NeutralCase_summary_sfscheme}
\end{figure}

The "FV pure" and "FV1" strategies are not satisfactory because
\begin{itemize}
	\item "FV1" relies on the assumption that $\overline{u}_{\frac{1}{2}} = u(z_{\frac{1}{2}})$
	\item both strategies assume that $u(z)$ is a parabolic spline in the surface layer and that $\eqref{eq:ND_NeutralCase_EkmanEq}$
		can be applied as well.
\end{itemize}
The "FV2" surface flux scheme solves those problems.
However, for all those discretizations, changing the space step
also changes the height of the surface layer $\delta_{\rm sl}$.
In the following, we derive a surface flux scheme that keep
the advantages of the "FV2" with a free $\delta_{\rm sl}$.

\subsubsection{A surface flux scheme with a free $\delta_{\rm sl}$}

