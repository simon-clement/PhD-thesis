\section{Oceanic surface layer}
We now interest ourselves to the discretization of an ocean
column and its surface layer.
Following the idea of \citep{pelletier_two-sided_2021} where
the Similarity theory is extended to the ocean surface layer
(creating a \textit{two-sided} bulk procedure) we
follow the discretization developed in the previous sections
to handle the oceanic surface layer.
%
\par
In the ocean modelling community, it is unusual to use
wall modelling at the surface. The Sea Surface Temperature
(SST) is often taken at the first grid level inside the
ocean: this corresponds to the assumption that 
the profiles are constant in the surface layer.
\cite{zeng_prognostic_2005} proposed a scheme to avoid this
issue for weather forecasting, climate modeling and
data assimilation.
%
\par
The goal of this section is to investigate the effects
of the surface layer discretization in a coupled situation.
We first (Section \ref{sec:ND_Ocean_differencesWithAtmosphere})
specify the differences between the ocean and
atmosphere models in use,
then focus on the incorporation of radiative fluxes in
Section \ref{sec:ND_Ocean_radiativeFluxes}.
Finally, Section \ref{sec:ND_Ocean_bulkRadiativeFluxes}
presents the sensitivity to the discretization of the
surface layer in the coupled setting.
\subsection{Differences with the atmosphere and derivation of
	a symmetric surface flux scheme}
\label{sec:ND_Ocean_differencesWithAtmosphere}
In this section, the numerical model of a stratified column
of ocean is described. The objective is to obtain a model
similar to the atmosphere while including the specificities
of the ocean.
In particular, the main changes compared to the atmosphere
model are:
\begin{itemize}
	\item the density which mainly affects the exchanges
		between ocean and atmosphere;
	\item the universal stability functions which are
		taken in \cite{large_similarity_2019};
	\item the vertical axis $z \leq 0$
		which is such that the surface corresponds to $z=0$;
	\item the time and space scales (the motion in the ocean
		is slower and smaller space steps are used with
		larger time steps).
\end{itemize}
\subsubsection{The Ocean model}
We describe here the continuous model in use for the ocean:
the discretization is identical to the atmospheric case
and is not detailed.
The equations for momentum, potential temperature and
turbulent kinetic energy in the inner domains are not changed
except that the geostrophic momentum is not included in the ocean:
\begin{equation}
	\begin{aligned}
	&(\partial_t + if) u - \partial_z (K_u \partial_z u) = 0
		,~~~~~ z \leq \delta_{o} \\
	&\partial_t \theta -\partial_z (\partial_z K_{\theta} \theta)
	= F_{\theta},~~~~~~~~~~~~~~ z \leq \delta_{o} \\
		&\partial_t e =
    \underbrace{\partial_z \left(K_e
    \partial_z e\right)}_{\text{diffusion}}
    + \underbrace{K_u ||\partial_z u||^2}_{\text{shear}} 
    - \underbrace{K_{\theta} N^2 }_{\text{buoyancy}}
    - \underbrace{c_{\epsilon}
    \frac{e^{3/2}}{l_{\epsilon}(z)}}_{\text{dissipation}}.
	\end{aligned}
\end{equation}
In the first grid level below the surface, $F_\theta$ includes a
radiative forcing corresponding to the shortwave flux of a
diurnal cycle. As in the atmospheric case the buoyancy $N^2$ is
given by a linear equation of state
$N^2 = g \alpha \partial_z \theta$.
The mixing lengths follow the description in 
\ref{sec:ND_StratifiedCase_turbulentVisc}
except that $l_{up}$ and $l_{down}$ are swapped and the shear
is neglected in \eqref{eq:ND_StratifiedCase_lD80}.
%
\paragraph{Initialization and boundary conditions.}
At initialization, the TKE is set to $e=e_{\rm min}$, 
the temperature is set to a constant $\theta = 280 \; {\rm K}$
and the initial momentum is set to $u = 0 \; {\rm m}.{\rm s}^{-1}$.
At the bottom boundary,
$\partial_z u = 0$,
$\partial_z \theta = 0$ and $e=e_{\rm min}$.

\subsubsection{The surface boundary condition}
We derive here the surface flux "FV free" applied to the oceanic
column model. Let $k$ be the space index such that
$z_{k-1} < \delta_o \leq z_k$;
we note $\widetilde{u}, \widetilde{\theta}$
the averaged variables over the interval $(z_{k-1}, \delta_o)$.
In the case of the atmosphere column, equation
\eqref{eq:ND_StratifiedCase_relation_tilde_bar}
gives the relation between the averaged variables
$\overline{u}, \overline{\theta}$ over the volume $(z_{k-1}, z_{k})$.
The same relation can be written here:
\begin{equation}
	\label{eq:ND_Ocean_tmprelation_tilde_bar}
\begin{aligned}
	\frac{\widetilde{h}}{h_{k-1/2}} (u(0) - \widetilde{u}) =
	(u(0) - \overline{u}_{k-1/2}) -
	(u(0) - u(\delta_{sl}))\tau_{sl, u} \\
\frac{\widetilde{h}}{h_{k-1/2}} (\theta_s - \widetilde{\theta}) =
	(\theta_s - \overline{\theta}_{k-1/2}) -
	(\theta_s - \theta(\delta_{sl}))\tau_{sl, \theta}
\end{aligned}
\end{equation}
with for $x = u, \theta$:
\begin{equation}
	\tau_{sl, x} = \frac{\frac{1}{{h_{k-1/2}}}\int_{\delta_{sl}}^{z_k} x(0) - x(z)
	dz}{x(0) - x(\delta_{sl})}.
\end{equation}
Section \ref{sec:ND_Ocean_stabilityFunctionIntegration}
gives the derivation of the explicit formula of $\tau_{sl}$
with the oceanic univeral stability functions.
Let us define the non-dimensional number
$\alpha_{sl, x} = \frac{\widetilde{h}}{h_{k-1/2}} +
\tau_{sl, x}$.
We inject $u(\delta_{sl}) =
{\cal S}(\frac{\widetilde{h}}{2})$ and
$\theta(\delta_{sl}) = {\cal T}(\frac{\widetilde{h}}{2})$:
\begin{equation}
\begin{aligned}
\label{eq:ND_Ocean_relation_tilde_bar}
\alpha_{sl, u}\widetilde{u} = \overline{u}_{k-1/2} -
	\tau_{sl, u}
\widetilde{h}
	(\phi_{\delta}/3 + \phi_{k-1}/6) - (1 - \alpha_{sl, u})u(0)\\
\alpha_{sl, \theta}
\widetilde{\theta}
= \overline{\theta}_{k-1/2} - \tau_{sl, \theta}
	\widetilde{h}(\frac{{(\partial_z \theta)}_{\delta}}{3}
	+ \frac{{(\partial_z \theta)}_{k-1}}{6})
 - (1 - \alpha_{sl, \theta})\theta_s
\end{aligned}
\end{equation}
Finally, the scheme at the first grid level above
the surface layer is:
%\begin{equation}
%\label{eq:ND_Ocean_prognosticu_FVfree}
%    \begin{aligned}
%(\partial_t + if)
%	    \left(\frac{\widetilde{h}}{6h_{k-1}} 
%    \phi_k
%    +
%    \left(
%	    \frac{\widetilde{h}}{3h_{k-1}} 
%	    + \frac{h_{k-3/2}}{3h_{k-1}}
%    \right)
%	    \phi_{k-1}
%	    + \frac{h_{k-3/2}}{6h_{k-1}} \phi_{k-2}\right)
%    = \\
%	     \frac{K_{u, k} \phi_{k} -
%	    K_{u,k-1} \phi_{k-1} }{\widetilde{h}h_{k-1}} -
%\frac{K_{u, k-1} \phi_{k-1} - K_{u, k-2} \phi_{k-2}}
%	    {h_{k-3/2}h_{k-1}} + \frac{\widetilde{F} - F_{k-3/2}}
%	    {h_{k-1}}
%    \end{aligned}
%\end{equation}
\begin{equation}
	\label{eq:ND_Ocean_semiDiscreteEkmanEqFVfree}
	(\partial_t+if) \left(\frac{1}{\alpha_{\rm sl, u}}
	\left(\overline{u}_{k-1/2} - \widetilde{h}\tau_{\rm sl, u}
	(\frac{\phi_k}{3} + \frac{\phi_{k-1}}{6}) - 
	(1-\alpha_{\rm sl, u})u(0)\right)\right)
	= \frac{K_{u, k}\phi_{k} - K_{u,k-1} \phi_{k-1}}{\widetilde{h}}
\end{equation}
%\begin{equation}
%\label{eq:ND_Ocean_prognosticPT_FVfree}
%    \begin{aligned}
%\partial_t \left(\frac{\widetilde{h}}{6h_{k-1}} 
%	    {(\partial_z \theta)}_k
%    +
%    \left(
%	    \frac{\widetilde{h}}{3h_{k-1}} 
%	    + \frac{h_{k-3/2}}{3h_{k-1}}
%    \right)
%	    {(\partial_z \theta)}_{k-1}
%	    + \frac{h_{k-3/2}}{6h_{k-1}} {(\partial_z \theta)}_{k-2}\right)
%    = \\
%	    \frac{K_{\theta, k} {(\partial_z \theta)}_{k} -
%	    K_{\theta, k-1} {(\partial_z \theta)}_{k-1} }{\widetilde{h}h_{k-1}}
%	    -\frac{K_{\theta, k-2} {(\partial_z \theta)}_{k-1} -
%	    K_{\theta, k-2} {(\partial_z \theta)}_{k-2}}
%	    {h_{k-3/2}h_{k-1}} + \frac{\widetilde{F}_\theta -
%	    F_{\theta, k-3/2}}{h_{k-1}}
%    \end{aligned}
%\end{equation}
\begin{equation}
	\label{eq:ND_Ocean_semiDiscreteEkmanEqPTFVfree}
	\partial_t \left(\frac{1}{\alpha_{\rm sl, \theta}}
	\left(
	\overline{\theta}_{k-1/2} - \tau_{sl, \theta}
	\widetilde{h}(\frac{{(\partial_z \theta)}_k}{3} +
	\frac{{(\partial_z \theta)}_{k-1}}{6})
	 - (1 - \alpha_{sl, \theta})\theta_s
	\right) \right)
	= \frac{K_{\theta, k}{(\partial_z \theta)}_{k} -
	K_{\theta,k-1} {(\partial_z \theta)}_{k-1}}{\widetilde{h}}
	+ \widetilde{F}_\theta 
\end{equation}
The scheme uses as boundary conditions at the surface layer:
\begin{equation}
\label{eq:ND_Ocean_boundaryConditionFVfree}
\begin{aligned}
	K_{u,\delta} \phi_{\delta} &=
	u_\star^2 e_\tau
	\\
	K_{\theta, \delta} (\partial_z \theta)_\delta &= 
	\theta_\star u_\star - \frac{Q_{lw} + Q_{sw}(\delta_{sl})}{\rho_0 c_p}
  \end{aligned}
\end{equation}
where $Q_{lw}, Q_{sw}$ are defined in Section
\ref{sec:ND_Ocean_radiativeFluxes}.
The surface flux scheme is summarized in Figure
\ref{fig:ND_Ocean_nouvelle_dis_neutre}.
\begin{figure}
	\subimport{images/}{nouvelle_dis_neutre_ocean.pdf_tex}
	\caption{Surface layer scheme "FV free" in the oceanic SCM}
	\label{fig:ND_Ocean_nouvelle_dis_neutre}
\end{figure}
As it will be explained
in Section \ref{sec:ND_Ocean_radiativeFluxes} this surface flux scheme
does not correctly take into account the radiative fluxes and the
interval $(\delta_o, z_k)$ needs an evolution equation for $\theta$.

\subsubsection{Computing $\tau_{\rm sl}$: stability function integration}
\label{sec:ND_Ocean_stabilityFunctionIntegration}
In the ocean, we use the universal function of \citep{large_similarity_2019} (as in \citep{pelletier_two-sided_2021})
\begin{equation}
	\phi^m(\zeta) = \phi^h(\zeta) = 1+5\zeta, ~~~ \zeta \geq 0
\end{equation}
\begin{equation}
	\phi^m(\zeta) = (1-14\zeta)^{-1/3}, ~~~ \zeta < 0
\end{equation}
\begin{equation}
	\phi^h(\zeta) = (1-25\zeta)^{-1/3}, ~~~ \zeta < 0
\end{equation}
and the first integrated form, $\psi(\zeta)= \int_0^\zeta \frac{1-\phi(\zeta')}{\zeta'}d\zeta'$ is for $\zeta<0$
\begin{equation}
	\psi^{\{m,h\}}(\zeta)  = \sqrt{3}\left[\arctan(\sqrt{3}) -
	\arctan\left(\frac{\sqrt{3}}{3}(2C_{\{m,h\}}+1)\right)\right]
	+ \frac{3}{2}\ln \left(\frac{(C_{\{m,h\}})^2 + C_{\{m,h\}} + 1}{3}\right)
\end{equation}
and $\psi^{\{m,h\}}(\zeta) = -5\zeta$ for $\zeta \geq 0$
where $C_m = (1-14\zeta)^{1/3}$, $C_h = (1-25\zeta)^{1/3}$.
We now compute its volume-averaged form $\Psi^{\{m,h\}}(\zeta)= \frac{1}{\zeta}\int_0^\zeta \psi^{\{m,h\}}(x)dx$.
\par
First,
\begin{equation}
	\int \ln (C_m^2+C_m+1)d\zeta
	= -2 \frac{\zeta}{3} - \frac{1}{28}C_m^2 - \frac{1}{14}C_m
	+\zeta\ln (C_m^2+C_m+1) + {\rm const}
\end{equation}
and
\begin{equation}
	\int \arctan (\frac{2C_m+1}{\sqrt{3}})d\zeta
	= \frac{\sqrt{3}}{56}C_m(C_m-2)
	+ \zeta \arctan (\frac{2C_m+1}{\sqrt{3}}) + {\rm const}
\end{equation}
Putting them together, we get
\begin{equation}
\begin{aligned}
	\Psi^m(\zeta)=\sqrt{3}\arctan(\sqrt{3})- 
	\frac{3}{2}\ln(3) -
	\frac{3}{56\zeta}C_m(C_m-2)
	- \sqrt{3} \arctan (\frac{2C_m+1}{\sqrt{3}})
	\\
	-1 - \frac{3}{56\zeta}C_m^2 - \frac{3}{28\zeta}C_m
	+\frac{3}{2}\ln (C_m^2+C_m+1)
	+ \frac{{\rm const}}{\zeta}.
\end{aligned}
\end{equation}
After simplification, using $\lim_\zeta\to0 \int_0^\zeta\psi(x)dx = 0$:
\begin{equation}
\begin{aligned}
\Psi^{\{m,h\}}(\zeta)= -\frac{5}{2}\zeta, ~~~~ \zeta \geq 0 \\
	\Psi^{\{m,h\}}(\zeta)=
	\psi^{\{m,h\}}(\zeta)
	- \frac{(2C_{\{m,h\}}+1)(C_{\{m,h\}} - 1)}
	{2\left((C_{\{m,h\}})^2 + C_{\{m,h\}} + 1\right)}, ~~~~ \zeta < 0
\end{aligned}
\end{equation}
We finally get for $u$:
\begin{equation}
	\tau_{sl, u} = \frac{\frac{1}{{h_{k-1/2}}}
    \left[
	    (-z+z_{0m})\ln(1+\frac{-z}{z_{0m}})+z
    -
    z \Psi_m(\frac{-z}{L_{MO}}) \right]_{\delta_{sl}}^{z_k}
    }{\ln(1+\frac{-\delta_{sl}}{z_{0m}})- \psi_u(\frac{-\delta_{sl}}{L_{MO}})
    }
\end{equation}
and for $\theta$:
\begin{equation}
	\tau_{sl, \theta} =  \frac{1}{{h_{k-1/2}}}\frac{(Q_H -
		Q_{lw})
    \left[
	    (-z+z_{0H})\ln(1+\frac{-z}{z_{0H}})+z
    -
    z \Psi_m(\frac{-z}{L_{MO}}) \right]_{\delta_{sl}}^{z_k}
	- Q_{sw} \int_{\delta_o}^{z_k}
		E(z) dz
    }{(Q_H - Q_{lw})
	    \left(\ln(1+\frac{-\delta_{sl}}{z_{0H}})-
	    \psi_\theta(\frac{-\delta_{sl}}{L_{MO}})\right)
	    -Q_{sw} E(\delta_{sl})
    }
\end{equation}
where $E(z) = \int_{z}^0 \frac{\phi^h_o(-x/L_o)
		\sum A_i \exp (k_i x)
		}{-x + z_{0M}}dx$
and $Q_H = \theta_o^{\star}u_o^{\star}\rho_0 c_p$.
In practice, $\tau_{sl, \theta}$ will not be used: in Section
\ref{sec:ND_Ocean_radiativeFluxes} we describe a better way to
include the radiative fluxes in the surface layer.
% \paragraph{Computing E(z)}
% To be computed numerically, $E(z)$ is separated into two parts,
% similarly to the MOST theory:
% \begin{equation}
% 	E(z) = \sum_i A_i\int_z^0
% 	\frac{\exp(k_i x)}{-x + z_{0M}}
% 	- \exp(k_i x)\frac{1 - \phi_0^h(\frac{-x}{L_{MO}})}
% 	{-x + z_{0M}}
% 	dx
% \end{equation}
% \begin{equation}
% 	E(z) = \sum_i A_i \exp(k_i z_{0M}) \left(E1(k_i z_{0M})
% 	- E1(k_i (-z+z_{0M)})\right)
% 	- \int_{z}^0 \sum_i A_i \exp(k_i x)
% 	\frac{1-\phi(\frac{-x}{L_{MO}})}{-x+z_{0M}} dx
% \end{equation}
% where $E1=\int_x^\infty \frac{\exp(-t)}{t}dt$ is the exponential
% integral.
% The left part $\sum_i A_i \exp(k_i z_{0M}) \left(E1(k_i z_{0M})
% 	- E1(k_i (-z+z_{0M)})\right)$ should be smaller than
% $\int_z^0 \frac{1}{-x + z_{0M}}dx$
% and the second part $\int_{z}^0 \sum_i A_i \exp(k_i x)
% 	\frac{1-\phi(\frac{-x}{L_{MO}})}{-x+z_{0M}} dx$
% 	should be smaller than $\int_{z}^0
% 	\frac{1-\phi(\frac{-x}{L_{MO}})}{-x+z_{0M}} dx$
\subsection{Radiative fluxes, another surface flux scheme}
\label{sec:ND_Ocean_radiativeFluxes}
We note $Q_{sw}$ and $Q_{lw}$ the shortwave and longwave (positive
downward) fluxes.
To include those fluxes, \cite{pelletier_two-sided_2021} introduced
a variable $\theta^\star_{o, \rm rad}$
that is similar to a friction scale but depends on $z$:
\begin{equation}
\theta^\star_{o, \rm rad}(z) =
	\theta^\star_{o}(z) -
	\frac{Q_{sw}(z) + Q_{lw}}{u^{\star}_o\rho_0 c_p}.
\end{equation}
This variable is used instead of $\theta^\star_{o}$
in the two-sided bulk procedure to include the radiative fluxes.
To be fully consistent between the computational domain and
the oceanic surface layer, the boundary condition for the temperature
should take into account the radiative fluxes: in
\eqref{eq:ND_Ocean_boundaryConditionFVfree} the flux
$K_\theta \partial_z \theta$ is exactly equal to
$\theta^\star_{o, \rm rad}(z) u^\star_{o}$.
However, it seems important to keep an evolution equation
in the surface layer {\color{red} là faut que ça mature,
je sais pas pourquoi ca serait mieux avec une équation d'évolution}.
%
\par
The bulk procedure is the one used in
\citep{pelletier_two-sided_2021}.
When computing
$\frac{\kappa (\theta(\delta_a) - \theta(\delta_o))}
	{\theta_a^{\star}}$,
	we use:
\begin{equation}
	\label{eq:ND_Ocean_skinbulk}
	\frac{\kappa (\theta(0) - \theta(\delta_o))}
	{\theta_o^{\star}} = (1 -
	\frac{Q_{lw}}{Q_H})
	\left(\ln (1 - \frac{\delta_o}{z_{0H}}) -
	\psi_h(\frac{-\delta_o}{L_o})\right)
	- \frac{\lambda_\theta Q_{sw}(0)}
	{Q_H} E(\delta_o).
\end{equation}
\subsubsection{Adding radiative fluxes: derivation without
molecular sublayer}
As it was done in 
\citep{zeng_prognostic_2005}, we use an evolution equation
in the surface layer. We neglect the molecular sublayer and
integrate between $\delta_o$ and 0:
the flux at $z=0$ is
$K_\theta \partial_z \theta = \frac{Q_H - Q_{lw}}{\rho_0 c_p}$
where $Q_H = \theta_o^\star u_o^\star \rho_0 c_p$
\begin{equation}
\label{eq:evolEqOSL}
h_{sl}\partial_t \overline{\theta}_{sl} =
\frac{Q_H - Q_{lw}}{\rho_0 c_p}
- \left. K_\theta \partial_z \theta 
\right|_{\delta_{o}}
- h_{sl}\int_{\delta_o}^0
\frac{\partial_z Q_{sw}}{\rho_0 c_p} dz
\end{equation}
where $h_{sl} = |\delta_o|$
is the size of the surface layer and
$\overline{\theta}_{sl}$ is the average potential
temperature in the surface layer.
We will use the following reconstruction of $\theta$
inside the SL:
\begin{equation}
    \theta(z) = \theta_s -
    \left(\frac{z}{\delta_{o}}\right)^\nu \left( \theta_s - 
    \theta_{\delta_o}\right)
\end{equation}
where $\nu$ is a constant parameter that can be set to 1 to
recover a linear reconstruction of $\theta$.
We formulate the reconstruction in terms of $\overline{\theta}_{sl}$
and $\left.\partial_z \theta\right|_{\delta_o}$:
{\color{red} TODO intégrer seulement jusqu'à $z_k$ !}
\begin{equation}
    \overline{\theta}_{sl} = \frac{1}{\nu+1}
    (\theta_{\delta_o} + \nu \theta_s)
, ~~~ \left.\partial_z \theta \right|_\delta
= \frac{\nu}{h_{sl}} (\theta_s - \theta_{\delta_o}).
\end{equation}
We get that the surface temperature
is $\theta_s = \overline{\theta}_{sl}
+ \frac{h_{sl}}{\nu(\nu+1)} \partial_z \theta$ and
the difference of temperature between
the surface and the bottom of the surface layer is
$\theta_s - \theta_{\delta_o} = 
\frac{h_{sl}}{\nu}\partial_z \theta$:
\begin{equation}
    \theta(z) = \overline{\theta}_{sl} +
    \frac{h_{sl}}{\nu} \partial_z \theta_{\delta_o} \left(
    \frac{1}{\nu+1} + \left(\frac{z}{h_{sl}}\right)^\nu
\right).
\end{equation}
Let us assume now that between
$z_{k-1}$ and $\delta_o$ , there is a quadratic spline.
Using the continuity at $z=\delta_o$ and Chasles' relation,
\begin{equation}
\underbrace{
    \overline{\theta}_{sl} -
    \frac{h_{sl}}{\nu+1} \partial_z \theta
    }_{\theta(\delta_o^{+})}
    =
    \underbrace{
    \widetilde{\theta}
    + \frac{\widetilde{h}}{3}
    \left.\partial_z \theta\right|_{\delta_o}
    + \frac{\widetilde{h}}{6}
	\left.\partial_z \theta\right|_{z_{k-1}}}_{
	\theta(\delta_o^{-})}
	,
    ~~~~
	\underbrace{
    h_{-1/2} \overline{\theta} = 
    h_{sl}\overline{\theta}_{sl} +\widetilde{h}
    \widetilde{\theta}}_{
    \int_{z_{k-1}}^{0}\theta(z) dz
    }.
\end{equation}
The reconstruction for $|z|<|\delta_o|$ is hence
\begin{equation}
    \theta(z) =
    \overline{\theta}_{-\frac{1}{2}}
    +
    \left(\frac{h_{sl}}{\alpha_{sl}(\nu+1)} +
    \frac{\widetilde{h}}{3\alpha_{sl}}
    +
    \frac{h_{sl}}{\nu}\left(
    \frac{1}{\nu+1} + \left(\frac{z}{h_{sl}}
    \right)^\nu
    \right)\right) \partial_z \theta_{\delta_o}
    +
    \left(\frac{\widetilde{h}}{6\alpha_{sl}}\right)
    \partial_z \theta_{k-1}
\end{equation}
with $\alpha_{sl} = 1+\frac{h_{sl}}{\widetilde{h}}$.
for $|\delta_o| < |z| < |z_{k-1}|$,
in the quadratic region, we have
\begin{equation}
    \theta(z) =
        \widetilde{\theta}
        +
        \left(
        z-\delta_o + 
        \frac{(z-\delta_o)^2}{2\widetilde{h}}
        + \frac{\widetilde{h}}{3}
        \right)\partial_z \theta_{\delta_o}
        +
        \left(
        \frac{\widetilde{h}^2 - 
        3(z-\delta_o)^2}{6 \widetilde{h}}
        \right)\partial_z \theta_{k-1}
\end{equation}
where $\widetilde{\theta}$ is computed with
\begin{equation}
\label{eq:formulaTildeTheta}
\widetilde{\theta} = \overline{\theta}_{-1/2}
-\frac{1}{\widetilde{\alpha}}\left(
\frac{h_{sl}}{\nu+1} + \frac{\widetilde{h}}{3}
\right)\partial_z \theta_{\delta_o}
- \frac{\widetilde{h}}{6\widetilde{\alpha}}
\partial_z \theta_{k-1}, ~~~~ \widetilde{\alpha} = 1+ \frac{\widetilde{h}}{h_{sl}}
\end{equation}
The continuity equation at $z_{-1}$
uses $\widetilde{h}$
as the space step:
\begin{equation}
    \widetilde{\theta}
    - \frac{\widetilde{h}}{3}
    \partial_z \theta_{-1}
    - \frac{\widetilde{h}}{6}
    \partial_z \theta_{0}
    = \overline{\theta}_{-3/2}
    + \frac{\widetilde{h}}{3}
    \partial_z \theta_{-1}
    + \frac{\widetilde{h}}{6}
    \partial_z \theta_{-2}
\end{equation}
The evolution equation of the first volume is:
\begin{equation}
h_{-\frac{1}{2}}\partial_t
\overline{\theta}_{-\frac{1}{2}} =
\frac{Q - Q_{lw}}{\rho_0 c_p}
- K_{\theta, -1} \partial_z \theta_{-1}
- h_{-\frac{1}{2}}\int_{z_{-1}}^0
\frac{\partial_z Q_{sw}}{\rho_0 c_p} dz.
\end{equation}
We need a last equation to close the system.
Starting from \eqref{eq:formulaTildeTheta}, we get
\begin{equation}
\partial_t
\left(
\widetilde{h}\overline{\theta}_{-1/2}
-\frac{1}{\widetilde{\alpha}}\left(
\frac{\widetilde{h}h_{sl}}{\nu+1} + \frac{\widetilde{h}^2}{3}
\right)\partial_z \theta_{\delta_o}
- \frac{\widetilde{h}^2}{6\widetilde{\alpha}}
\partial_z \theta_{k-1}
\right)
= K\partial_z \theta_k-
K\partial_z \theta_{k-1}
- \widetilde{h}
\int^{\delta_o}_{z_{k-1}}
\frac{\partial_z Q_{sw}}{\rho_0 c_p} dz
\end{equation}
where 
$1/\widetilde{\alpha}=
\frac{h_{sl}}{-\delta_{o}}$

\subsection{Two-sided Bulk and sensitivity to the discretization
	of the surface layers}
\label{sec:ND_Ocean_bulkRadiativeFluxes}
\begin{equation}
	C_D^{a+o} = \left(\frac{\kappa} {
		\ln(1 + \frac{\delta_a}{z_{0M}}) -
		\psi_m^a(\frac{\delta_a}{L_a})
		+ \lambda_u \left(
		\ln(1 - \frac{\delta_o}{z_{0M}}) -
		\psi_m^o(-\frac{\delta_o}{L_o})
		\right)
	} \right)^2
\end{equation}
\begin{equation}
	C_H^{a+o} = \frac{\kappa \sqrt{C_D}} {
		\left(
			\ln(1 + \frac{\delta_a}{z_{0H}}) -
			\psi_h^a(\frac{\delta_a}{L_a})
		\right)
		+
		(1 - \frac{Q_{lw}}
		{Q_H})
		\lambda_\theta \left(
			\ln(1 - \frac{\delta_o}{z_{0H}}) -
			\psi_h^o(-\frac{\delta_o}{L_o})
		\right)
		- \frac{\lambda_\theta Q_{sw}(0)}
		{Q_H} E(\delta_o)
	}
\end{equation}
In the bulk algorithm, we use $u_a^\star = \sqrt{C_D} |u(\delta_a)-
u(\delta_o)|$ and
$\theta_a^\star = \frac{C_H}{\sqrt{C_D}}(\theta(\delta_a)-
\theta(\delta_o))$ to get the friction scales.
Multiplying those equations gives the boundary condition
$K_\theta \partial_z \theta = \lambda_u \lambda_t u_a^\star \theta_a^\star$.
\footnote{
$u_a^\star$ is actually directly used because $\frac{C_H}{\sqrt{C_D}}$
can be simplified.
}
We use the following notation for $C_D$ and $C_H$:
\begin{equation}
	C_D^{a+o} = \frac{u_a^\star}{|u(\delta_a) - u(\delta_o)|},
	~~~
	C_D^{a} = \frac{u_a^\star}{|u(\delta_a) - u(0)|},
	~~~
	C_D^{o} = \frac{u_a^\star}{|u(0) - u(\delta_o)|},
\end{equation}

