\chapter*{Conclusion and perspectives}
\label{ch:conclusion}
\addcontentsline{toc}{chapter}{\nameref{ch:conclusion}}
This thesis focused on the numerical analysis of interactions between
the ocean and atmosphere within their coupling. We proposed a
new discretization of the surface layer which relies on
hypotheses already in use in the computation of turbulent fluxes.
Besides, the Schwarz methods were studied at discrete and
semi-discrete levels and we have highlighted important features
of their convergence. The analysis of convergence of Schwarz methods
was finally pursued
in the presence of a parameterized surface layer and we
proved some convergence results on this simplified air-sea coupled
problem.
\par
Let us recall the three objectives of this thesis (a hierarchy
of three models was used to achieve each of them):
\begin{itemize}
\item Improve our knowledge on how the discretization affects
the convergence factor of Schwarz methods.
\item Discuss the numerical treatment of the surface layer
	and propose improvements.
\item Study the effect of the surface layer within
	the ocean-atmosphere coupling.
\end{itemize}
\par
Chapter \ref{ch:discreteSchwarzAnalysis} discussed the convergence
analysis of Schwarz methods at the semi-discrete and discrete
levels.
The goal was do develop a methodology to be used later
on more sophisticated models.
The effect of the discretization in both
space and time was highlighted: we have notably identified
the importance of the interface conditions,
both in their interpolation within a multi-step time scheme
and in their discretization in space.
The interactions between space and time discretizations
were also investigated: they are characterized by the
parabolic Courant number.
Finally, a Finite Volume discretization was derived based
on a subgrid reconstruction with quadratic splines.
The special attention on the subgrid reconstruction
was later used for the discretization of the surface layer.
\par
Chapter \ref{ch:approximatedDiscreteSchwarz} introduced
approximations that can be used to simplify the derivation
of the theoretical semi-discrete and fully discrete
convergence factors.
Indeed, as the complexity of the discretizations increase the
theoretical convergence factors become tedious to
compute analytically.
One method approximates the semi-discrete convergence factors
relying on the modified equations technique and
the other method combines the semi-discrete analyses
to approximate the fully discrete one.
The ideas of those methods are simple and we exposed some
of their inherent strengths and weaknesses.
For instance, the effect of the interpolation of the interface
conditions discussed in Chapter \ref{ch:discreteSchwarzAnalysis}
is not present when using the modified equations technique.
We found in which case those approximations could be useful and
explained the reasons for which they can be inadequate in other
cases.
\par
In Chapter \ref{ch:ND}, the surface layer hypotheses were discussed:
we proposed a discretization based on the Monin-Obukhov Similarity
Theory.
We used the solution profiles of this Similarity Theory
as the subgrid reconstruction on which is based
the Finite Volume discretization.
This discretization
is such that the computation of turbulent fluxes is coherent
with the subgrid reconstruction of the solution.
It was recently highlighted that the height of the
surface layer should not (always) be limited to the first grid level.
We hence allowed the size of the surface layer to be freely chosen 
independently from the grid levels: as a result, the
consistency of the discretization was found to be better than
the usual methods that constrain the surface layer height.
\par
This discretization is applied to the oceanic part of the surface
layer in Chapter \ref{ch:OceanND}, even though considering an oceanic
surface layer is recent and still uncommon. We focused specifically
on the handling of radiative fluxes which lead to particular
difficulties in terms of subgrid reconstruction. The proposed
discretization of the surface layer including radiative fluxes
is a first step toward a well discretized oceanic surface layer.
We have shown that this discretization does not perform well in
terms of consistency and should be improved.
\par
Finally, Chapter \ref{ch:OASchwarz} studied at the
semi-discrete-in-space level the effect of the surface layer on the
ocean-atmosphere coupling.
The methodology presented in Chapter \ref{ch:discreteSchwarzAnalysis}
was used to treat a surface layer which was simplified compared to
Chapters \ref{ch:ND} and \ref{ch:OceanND}.
The well-posedness of the coupled problem was discussed
and the existence and unicity of a solution in the neighborhood
of the steady state was proved using the inverse function theorem.
We examined the convergence of Schwarz methods with a parameterized
surface layer as a nonlinear transmission condition:
interestingly, the convergence factor was found to change from
one iteration to another.
The convergence study was pursued by linearizing the transmission
condition and we derived upper and lower bound of the
convergence factor,
corresponding to the singular values of a transition matrix between
the iterations.
We optimized numerically the convergence with respect to a relaxation
parameter and compared the convergence of the nonlinear system
with the one with a linearized transmission condition.
% PERSPECTIVES
\subsection*{Perspectives}
The ultimate goal of this thesis is
to reduce the numerical errors linked to the ocean-atmosphere
coupling in operational models. Two perspectives
directly linked with this objective arise:
\begin{itemize}
	\item Implement and evaluate the discretization of the
		surface layer introduced in Chapter \ref{ch:ND} and
	\ref{ch:OceanND} within more realistic simulations.
	The order of accuracy would need to be improved in the
	process: in particular, splitting a cell in two parts
	breaks the possible structured grid properties.
	Furthermore, the schemes should be analyzed in terms of
	stability, monotonicity, and other mathematical or physical
		properties.
	\item Make sure that the convergence of Schwarz methods
	is attained in one or two iterations: iterative coupling is
	otherwise unaffordable in this ocean-atmosphere context.
	This can be done by choosing better transmission
	conditions and/or by improving the first guess.
	This first guess could be obtained either with a
	well-chosen extrapolation of other time windows (the latter option
	is presently studied in the AIRSEA team in collaboration
	with O. Marti for the model IPSL-CM) or
	with Schwarz iterations on a simplified internal subproblem (e.g.
	similar to the ones studied in this thesis).
\end{itemize}
\par
Besides those direct applications, the theoretical aspects
also need to be consolidated in terms of well-posedness
and in terms of convergence of Schwarz methods.
It seems that this mathematical knowledge is not going to
catch up soon with the increasing complexity
of the operational models. However,
the former could provide guidelines to wisely
parameterize and discretize the latter while keeping
good mathematical properties.
\par
Several steps forward can be pursued as the
following of the present thesis which used the
simplification that the viscosity is constant.
Convergence studies of Schwarz methods
were conducted with variable viscosities but
never to our knowledge at the semi-discrete-in-space level.
Two perspectives arise from the objective of taking the
varying viscosities into account:
\begin{itemize}
	\item Developing mathematical tools so that the
		semi-discrete-in-space convergence factor could
		be studied with varying viscosities (as it was done
		in the continuous case in \citep{thery_etude_2021})
		or varying space steps.
		A first step toward this convergence factor
		would be to use the idea of the \textit{combined}
		convergence factor in
		Chapter \ref{ch:approximatedDiscreteSchwarz}:
		instead of combining the contribution of the
		space scheme and of the time scheme, it could
		be relevant to combine the contribution of
		the space scheme and of the varying viscosities.
	\item Extending the proof of well-posedness of Chapter
		\ref{ch:OASchwarz} to varying viscosities
		that depend on the surface layer parameterization.
		The atmosphere and ocean models were developed 
		separately and are coupled through a surface
		layer whose parameterization contains additional
		hypotheses. The compatibility between all the
		components of the ocean-atmosphere coupling
		is not guaranteed and the well-posedness
		study of simplified models could teach us a lot.
\end{itemize}
The coherence between the surface layer parameterization and
the discretization is a small aspect within the much wider problem of
articulating a dynamical core with a physical parameterization.
Incoherence may appear in this conception because of the huge
complexity of the numerical models:
the development and the implementation of parameterizations
need a numerical analysis to ensure a global harmony.
