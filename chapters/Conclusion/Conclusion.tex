\chapter*{Conclusion and perspectives}
\label{ch:conclusion}
This thesis focused on the numerical analysis of interactions between
the ocean and atmosphere within their coupling. We proposed a
new discretization of the surface layer which relies on
hypotheses already in use in the computation of turbulent fluxes.
Besides, the Schwarz methods were studied at discrete and
semi-discrete levels and we have highlighted important features
of their convergence. The analysis of convergence of Schwarz methods
was finally pursued
in the presence of a parameterized surface layer and we
proved some convergence results on this simplified air-sea coupled
problem.
\par
Chapter \ref{ch:discreteSchwarzAnalysis} discussed the convergence
analysis of Schwarz methods at the semi-discrete and discrete
levels. The effect of the discretization in both
space and time was highlighted: in particular, we have identified
the importance of the interface conditions interpolation
for time schemes of multiple steps.
Finally, a Finite Volume discretization was derived based
on a subgrid reconstruction with quadratic splines.
\par
Chapter \ref{ch:approximatedDiscreteSchwarz} introduced
approximations that can be used to simplify the derivation
of the theoretical semi-discrete or fully discrete convergence factor.
One method approximates the semi-discrete convergence factors
relying on the modified equations technique and
the other method combines the semi-discrete analyses
to approximate the fully discrete one.
We found in which case those approximations could be useful and
explained the reasons for which they can be inadequate in other
cases.
\par
In Chapter \ref{ch:ND}, the surface layer hypotheses were discussed:
we proposed a discretization based on the Monin-Obukhov Similarity
Theory. This discretization based on Finite Volumes
is such that the computation of turbulent fluxes is coherent
with the subgrid reconstruction of the solution.
The discretization also takes into account the recent discovery
in surface layer parameterization that the height of the
surface layer should not (always) be limited to the first grid level.
\par
This discretization is applied to the oceanic part of the surface
layer in Chapter \ref{ch:OceanND}, even though considering an oceanic
surface layer is recent and still uncommon. We focused specifically
on the handling of radiative fluxes which lead to particular
difficulties in terms of subgrid reconstruction. The proposed
discretization of the surface layer including radiative fluxes
is a first step toward a well discretized oceanic surface layer.
We have shown that this discretization does not perform well
and should be improved.
\par
Finally, Chapter \ref{ch:OASchwarz} studied at the semi-discrete
level in space the effect of the surface layer on the
ocean-atmosphere coupling.
The well-posedness of the coupled problem studied was discussed
and the existence and unicity of a solution in the neighborhood
of the steady state was proved.
We examined the convergence of Schwarz methods with a parameterized
surface layer as a nonlinear transmission condition:
we derived upper and lower bound of the convergence factor and
optimized numerically the convergence with respect to a relaxation
parameter.
\par
% PERSPECTIVES
The coupled problem examined in Chapter
\ref{ch:OASchwarz} is idealized in two aspects. The first
simplification is that
the parameterization of the surface layer is explicit whereas
it is instead based on fixed-point algorithms in operational models.
Relaxing this simplification is difficult because the convergence
study relies on the linearization of the explicit expression.
The second simplification is the constant viscosity.
Convergence studies were conducted with variable viscosities but
never to our knowledge at the semi-discrete in space level.
Two perspectives arise from the objective of taking the
varying viscosities into account:
\begin{itemize}
	\item Developing mathematical tools so that the
		semi-discrete in space convergence factor could
		be studied with varying viscosities or
		varying space steps;
	\item Instead of combining two semi-discrete analyses
		as it was done in
	Chapter \ref{ch:approximatedDiscreteSchwarz}, 
	the semi-discrete analysis including a surface layer
	could be combined with a continuous case involving
	varying viscosities.
\end{itemize}
\par
Another set of improvements should be mentioned for the discretization
of the surface layer introduced in Chapters \ref{ch:ND} and
\ref{ch:OceanND}:
\begin{itemize}
\item The schemes could be analyzed in terms of stability, monotony,
	or other mathematical or physical properties;
\item The order of the schemes could be improved: in particular,
	splitting a cell in two parts breaks the possible structured
		grid properties;
\item The discretization could be tested more thoroughly with
	realistic numerical models;
\end{itemize}
% Finally, recent developments in Schwarz methods analysis have
% shown that the analytical optimization of the convergence factor
% can be pursued with an asymptotic analysis that gives relevant
% optimized parameters
\par
Studying Schwarz methods applied to the ocean-atmosphere interface
is a contribution for the improvement of numerical analysis of
geophysical systems. The ultimate end of this thesis is
hence to reduce the numerical errors linked to the ocean-atmosphere
coupling in operational models.
