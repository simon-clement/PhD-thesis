\chapter*{Conclusion and perspectives}
\label{ch:conclusion}
\addcontentsline{toc}{chapter}{\nameref{ch:conclusion}}
This thesis focused on the numerical analysis of interactions between
the ocean and atmosphere within their coupling. We proposed a
new discretization of the surface layer which relies on
hypotheses already in use in the computation of turbulent fluxes.
Besides, the Schwarz methods were studied at discrete and
semi-discrete levels and we have highlighted important features
of their convergence. The analysis of convergence of Schwarz methods
was finally pursued
in the presence of a parameterized surface layer and we
proved some convergence results on this simplified air-sea coupled
problem.
\par
Let us recall the three objectives of this thesis:
a hierarchy of three models was used to achieve each of them.
\begin{itemize}
\item Improve our knowledge on how the discretization affects
the convergence factor of Schwarz methods.
\item Discuss the numerical treatment of the surface layer
	and propose improvements.
\item Study the effect of the surface layer within
	the ocean-atmosphere coupling.
\end{itemize}
\par
Chapter \ref{ch:discreteSchwarzAnalysis} discussed the convergence
analysis of Schwarz methods at the semi-discrete and discrete
levels.
The goal was do develop a methodology to be used later
on more sophisticated models.
The effect of the discretization in both
space and time was highlighted: we have notably identified
the importance of the interface conditions,
both in their interpolation within a multi-step time scheme
and in their discretization in space.
The interactions between space and time discretizations
were also investigated: they are characterized by the
parabolic Courant number.
Finally, a Finite Volume discretization was derived based
on a subgrid reconstruction with quadratic splines.
The special attention on the subgrid reconstruction
was later used for the discretization of the surface layer.
\par
Chapter \ref{ch:approximatedDiscreteSchwarz} introduced
approximations that can be used to simplify the derivation
of the theoretical semi-discrete and fully discrete
convergence factors.
Indeed, as the complexity of the discretizations increase the
theoretical convergence factors become tedious to
compute analytically.
One method approximates the semi-discrete convergence factors
relying on the modified equations technique and
the other method combines the semi-discrete analyses
to approximate the fully discrete one.
The ideas of those methods are simple and we exposed some
of their inherent strengths and weaknesses.
For instance, the effect of the interpolation of the interface
conditions discussed in Chapter \ref{ch:discreteSchwarzAnalysis}
is not present when using the modified equations technique.
We found in which case those approximations could be useful and
explained the reasons for which they can be inadequate in other
cases.
\par
In Chapter \ref{ch:ND}, the surface layer hypotheses were discussed:
we proposed a discretization based on the Monin-Obukhov Similarity
Theory.
We used the solution profiles of this Similarity Theory
as the subgrid reconstruction on which is based
the Finite Volume discretization introduced in Chapter
\ref{ch:discreteSchwarzAnalysis}.
This discretization
is such that the computation of turbulent fluxes is coherent
with the subgrid reconstruction of the solution.
It was recently discovered that the height of the
surface layer should not (always) be limited to the first grid level.
We hence allowed the size of the surface layer to be freely chosen 
independently from the grid levels: as a result, the
consistency of the discretization was found to be better than
the usual methods that constraint the surface layer height.
\par
This discretization is applied to the oceanic part of the surface
layer in Chapter \ref{ch:OceanND}, even though considering an oceanic
surface layer is recent and still uncommon. We focused specifically
on the handling of radiative fluxes which lead to particular
difficulties in terms of subgrid reconstruction. The proposed
discretization of the surface layer including radiative fluxes
is a first step toward a well discretized oceanic surface layer.
We have shown that this discretization does not perform well in
terms of consistency and should be improved.
\par
Finally, Chapter \ref{ch:OASchwarz} studied at the semi-discrete
level in space the effect of the surface layer on the
ocean-atmosphere coupling.
The methodology presented in Chapter \ref{ch:discreteSchwarzAnalysis}
was used to treat a surface layer which was simplified compared to
Chapters \ref{ch:ND} and \ref{ch:OceanND}.
The well-posedness of the coupled problem studied was discussed
and the existence and unicity of a solution in the neighborhood
of the steady state was proved using the inverse function theorem.
We examined the convergence of Schwarz methods with a parameterized
surface layer as a nonlinear transmission condition:
interestingly, the convergence factor was found to change from
one iteration to another.
The convergence study was pursued by linearizing the transmission
condition and we derived upper and lower bound of the
convergence factor,
corresponding to the singular values of a transition matrix between
the iterations.
We optimized numerically the convergence with respect to a relaxation
parameter and compared the convergence of the nonlinear system
with the one with a linearized transmission condition.
\par
% PERSPECTIVES
\subsection*{Perspectives}
The coupled problem examined in Chapter
\ref{ch:OASchwarz} is idealized in two aspects. The first
simplification is that
the parameterization of the surface layer is explicit whereas
it is instead based on fixed-point algorithms in operational models.
Relaxing this simplification is difficult because the convergence
study relies on the linearization of the explicit expression.
The second simplification is the constant viscosity.
Convergence studies were conducted with variable viscosities but
never to our knowledge at the semi-discrete in space level.
Two perspectives arise from the objective of taking the
varying viscosities into account:
\begin{itemize}
	\item Developing mathematical tools so that the
		semi-discrete in space convergence factor could
		be studied with varying viscosities or
		varying space steps;
	\item Instead of combining two semi-discrete analyses
		as it was done in
	Chapter \ref{ch:approximatedDiscreteSchwarz}, 
	the semi-discrete analysis including a surface layer
	could be combined with a continuous case involving
	varying viscosities.
\end{itemize}
\par
Another set of improvements should be mentioned for the discretization
of the surface layer introduced in Chapters \ref{ch:ND} and
\ref{ch:OceanND}:
\begin{itemize}
\item The schemes could be analyzed in terms of stability, monotony,
	or other mathematical or physical properties;
\item The order of the schemes could be improved: in particular,
	splitting a cell in two parts breaks the possible structured
		grid properties;
\item The discretization could be tested more thoroughly with
	realistic numerical models;
\end{itemize}
% Finally, recent developments in Schwarz methods analysis have
% shown that the analytical optimization of the convergence factor
% can be pursued with an asymptotic analysis that gives relevant
% optimized parameters
\par
Studying Schwarz methods applied to the ocean-atmosphere interface
is a contribution for the improvement of numerical analysis of
geophysical systems. The ultimate goal of this thesis is
hence to reduce the numerical errors linked to the ocean-atmosphere
coupling in operational models.
