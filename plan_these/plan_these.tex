\documentclass{article}
\usepackage[utf8]{inputenc}
\usepackage{amssymb,amsmath,mathrsfs,epsfig,color,url,calc,hyperref,nicefrac}
\usepackage{import}
\usepackage[numbers]{natbib}
\usepackage{enumitem}
%\RequirePackage{lineno}
\usepackage{placeins}
\usepackage{tikz}
\usepackage{multirow}
\usepackage{subcaption}
\usepackage{lineno}% line numbersk
\setlength{\textwidth}{16.5cm}
\setlength{\textheight}{25cm}
\topmargin -2.5 cm
\oddsidemargin 0 pt
\evensidemargin 0 pt
\linenumbers

\title{Plan}
\author{Simon Clement}
\date{Mai 2022}

\begin{document}
\begin{enumerate}
\item Etude de convergence des algorithmes de Schwarz
	\begin{itemize}
		\item Papier SMAI "Approximations of the discrete
			convergence factor of Schwarz methods"
		\item Taille de la fenêtre de temps
		\item Domaines bornés
		\item Initialisation aléatoire: bruit blanc 	
	\end{itemize}
\item Etude détaillée semi-discrètes et discrète avec
	conditions hétérogènes.
	\begin{itemize}
		\item Papier SMAI "Discrete Analysis of SWR for a Diffusion
			Reaction problem with Discontinuous Coefficients"
		\item La difficulté d'une optimisation analytique
		\item précision nécessaire de $\gamma$ pour avoir un ordre 2
		\item Ecart à la solution monolithique avec condition $\kappa_c=1$
			et avec $\gamma$
	\end{itemize}
\item Discrétisation des couches limites de surface atmosphérique
	et océanique
	\begin{itemize}
		\item présentation des schémas de flux de surface
			"FV free", "FD pure", etc.
		\item Consistence du schéma "FV free"
		\item Importance d'avoir le choix de $\delta_a$ dans
			un modèle couplé
	\end{itemize}
\item Convergence des algorithmes de Schwarz appliqué au couplage
	discret Océan-Atmosphère
	\begin{itemize}
	\item Proceeding DD26: "Discrete analysis of SWR for a
		simplified air-sea coupling problem
		with nonlinear transmission conditions"
	\item expériences numériques sur la convergence de Schwarz
		sur les modèles 1D de la section d'avant
	\end{itemize}
\end{enumerate}

\end{document}
