\documentclass{article}
\usepackage[utf8]{inputenc}
\usepackage{amssymb,amsmath,mathrsfs,epsfig,color,url,calc,hyperref,nicefrac}
\usepackage{import}
\usepackage[numbers]{natbib}
\usepackage{enumitem}
%\RequirePackage{lineno}
\usepackage{placeins}
\usepackage{tikz}
\usepackage{multirow}
\usepackage{subcaption}
\usepackage{lineno}% line numbersk
\setlength{\textwidth}{16.5cm}
\setlength{\textheight}{25cm}
\topmargin -2.5 cm
\oddsidemargin 0 pt
\evensidemargin 0 pt
\linenumbers

\title{Plan}
\author{Simon Clement}
\date{Mai 2022}

\begin{document}
\begin{enumerate}
\item Modélisation d'une colonne verticale ocean-atmosphère
\begin{enumerate}
	\item Derivation des équations primitives avec turbulence,
		obtention du cas (1D + guide géostrophique)
		\begin{enumerate}
			\item équations primitives non turbulentes
			\item Décomposition de Reynolds
			\item Fermeture Turbulente et énergie cinétique turbulente
		\end{enumerate}
	\item La couche limite de surface
		\begin{enumerate}
			\item Loi du mur et Monin-Obukhov
			\item Bulk two-sided, le couplage complet
		\end{enumerate}
	\item Hiérarchie de modèles utilisés
		\begin{enumerate}
			\item Couplage d'équations de
				réaction-diffusion avec
				diffusions différentes
			\item Couplage d'Ekman avec une loi de friction
			\item Couplage d'Ekman stratifié ou non,
				avec bulk et énergie cinétique turbulente
		\end{enumerate}
	\item Contexte et objectifs de la thèse
\end{enumerate}
\item Discrete Analysis of SWR for a Diffusion Reaction problem
	with Discontinuous Coefficients
		\begin{enumerate}
		\item Papier SMAI "Discrete Analysis of SWR for a
			Diffusion Reaction problem with
				Discontinuous Coefficients"
			\begin{enumerate}
				\item Model problem and SWR algorithm
				\item Semi-discrete and Discrete convergence rates
				\item Discrete Case
				\item Numerical examples, optimisation
			\end{enumerate}
		\item La difficulté d'une optimisation analytique:
			\begin{enumerate}
				\item optimisation du continu
				\item optimisation discrète avec
					recouvrement (Wu, 2014)
				\item pas d'optimisation discrète sans
					recouvrement
			\end{enumerate}
		\item Sur la solution monolithique
			\begin{enumerate}
				\item précision nécessaire de
					la projection
					des conditions d'interface
					pour avoir un ordre 2
				\item Ecart à la solution monolithique
			\end{enumerate}

	\end{enumerate}
\item Etude approchée de convergence des algorithmes de Schwarz
	\begin{enumerate}
		\item Papier SMAI "Approximations of the discrete
			convergence factor of Schwarz methods"
			\begin{enumerate}
					\item Discretisation and SWR
					\item Modified equations
					\item Combining Semi-discrete analyses
					\item Effect on optimisation
			\end{enumerate}
		\item Limitations du facteur de convergence discret
			\begin{enumerate}
				\item Domaines bornés
				\item Taille de la fenêtre de temps
				\item Initialisation aléatoire :
					influence du choix
					du bruit blanc 	
			\end{enumerate}
		\item Conclusion partielle
			% avant de passer à des modèles + sophistiqués
	\end{enumerate}
\item Discrétisation des couches limites de surface atmosphérique
	et océanique cohérente avec la théorie physique
	\begin{enumerate}
		\item Cas neutre atmosphérique
			\begin{enumerate}
				\item Rappels de loi du mur + splines
				\item Stratégies "classiques"
				\item Stratégie avec reconstruction
			\end{enumerate}
		\item Cas stratifié atmosphérique
			\begin{enumerate}
				\item Rappels de Monin-Obukhov
					et discretisation FV pour 
					la température
				\item Extension de la reconstruction
				\item Longueurs de mélanges et Monin-Obukhov
			\end{enumerate}
		\item Consistence du schéma "FV free"
			et sensibilité à $\delta_a$
			\begin{enumerate}
				\item Cas neutre
				\item Cas stable
				\item Cas instable
			\end{enumerate}
		\item Couche limite océanique
			\begin{enumerate}
				\item Différences par rapport à
					l'atmosphère,
					dérivation similaire
				\item flux radiatifs et reconstruction
				\item Bulk two-sided et sensibilité à
					la profondeur de couche limite
			\end{enumerate}
	\end{enumerate}
\item Convergence des algorithmes de Schwarz appliqué au couplage
	discret Océan-Atmosphère
	\begin{enumerate}
	\item Proceeding DD26: "Discrete analysis of SWR for a
		simplified air-sea coupling problem
		with nonlinear transmission conditions"
		\begin{enumerate}
			\item Model Problem for ocean-atmosphere coupling
			\item Discretized coupled problem
			\item Convergence analysis
			\item Numerical experiments
		\end{enumerate}
	\item expériences numériques sur la convergence de Schwarz
		sur les modèles 1D de la section d'avant
	\end{enumerate}
\end{enumerate}

\end{document}
